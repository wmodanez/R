\documentclass[]{article}
\usepackage{lmodern}
\usepackage{amssymb,amsmath}
\usepackage{ifxetex,ifluatex}
\usepackage{fixltx2e} % provides \textsubscript
\ifnum 0\ifxetex 1\fi\ifluatex 1\fi=0 % if pdftex
  \usepackage[T1]{fontenc}
  \usepackage[utf8]{inputenc}
\else % if luatex or xelatex
  \ifxetex
    \usepackage{mathspec}
  \else
    \usepackage{fontspec}
  \fi
  \defaultfontfeatures{Ligatures=TeX,Scale=MatchLowercase}
\fi
% use upquote if available, for straight quotes in verbatim environments
\IfFileExists{upquote.sty}{\usepackage{upquote}}{}
% use microtype if available
\IfFileExists{microtype.sty}{%
\usepackage{microtype}
\UseMicrotypeSet[protrusion]{basicmath} % disable protrusion for tt fonts
}{}
\usepackage[margin=1in]{geometry}
\usepackage{hyperref}
\hypersetup{unicode=true,
            pdftitle={Noções Básicas de R - Aula 4},
            pdfborder={0 0 0},
            breaklinks=true}
\urlstyle{same}  % don't use monospace font for urls
\usepackage{color}
\usepackage{fancyvrb}
\newcommand{\VerbBar}{|}
\newcommand{\VERB}{\Verb[commandchars=\\\{\}]}
\DefineVerbatimEnvironment{Highlighting}{Verbatim}{commandchars=\\\{\}}
% Add ',fontsize=\small' for more characters per line
\usepackage{framed}
\definecolor{shadecolor}{RGB}{248,248,248}
\newenvironment{Shaded}{\begin{snugshade}}{\end{snugshade}}
\newcommand{\KeywordTok}[1]{\textcolor[rgb]{0.13,0.29,0.53}{\textbf{#1}}}
\newcommand{\DataTypeTok}[1]{\textcolor[rgb]{0.13,0.29,0.53}{#1}}
\newcommand{\DecValTok}[1]{\textcolor[rgb]{0.00,0.00,0.81}{#1}}
\newcommand{\BaseNTok}[1]{\textcolor[rgb]{0.00,0.00,0.81}{#1}}
\newcommand{\FloatTok}[1]{\textcolor[rgb]{0.00,0.00,0.81}{#1}}
\newcommand{\ConstantTok}[1]{\textcolor[rgb]{0.00,0.00,0.00}{#1}}
\newcommand{\CharTok}[1]{\textcolor[rgb]{0.31,0.60,0.02}{#1}}
\newcommand{\SpecialCharTok}[1]{\textcolor[rgb]{0.00,0.00,0.00}{#1}}
\newcommand{\StringTok}[1]{\textcolor[rgb]{0.31,0.60,0.02}{#1}}
\newcommand{\VerbatimStringTok}[1]{\textcolor[rgb]{0.31,0.60,0.02}{#1}}
\newcommand{\SpecialStringTok}[1]{\textcolor[rgb]{0.31,0.60,0.02}{#1}}
\newcommand{\ImportTok}[1]{#1}
\newcommand{\CommentTok}[1]{\textcolor[rgb]{0.56,0.35,0.01}{\textit{#1}}}
\newcommand{\DocumentationTok}[1]{\textcolor[rgb]{0.56,0.35,0.01}{\textbf{\textit{#1}}}}
\newcommand{\AnnotationTok}[1]{\textcolor[rgb]{0.56,0.35,0.01}{\textbf{\textit{#1}}}}
\newcommand{\CommentVarTok}[1]{\textcolor[rgb]{0.56,0.35,0.01}{\textbf{\textit{#1}}}}
\newcommand{\OtherTok}[1]{\textcolor[rgb]{0.56,0.35,0.01}{#1}}
\newcommand{\FunctionTok}[1]{\textcolor[rgb]{0.00,0.00,0.00}{#1}}
\newcommand{\VariableTok}[1]{\textcolor[rgb]{0.00,0.00,0.00}{#1}}
\newcommand{\ControlFlowTok}[1]{\textcolor[rgb]{0.13,0.29,0.53}{\textbf{#1}}}
\newcommand{\OperatorTok}[1]{\textcolor[rgb]{0.81,0.36,0.00}{\textbf{#1}}}
\newcommand{\BuiltInTok}[1]{#1}
\newcommand{\ExtensionTok}[1]{#1}
\newcommand{\PreprocessorTok}[1]{\textcolor[rgb]{0.56,0.35,0.01}{\textit{#1}}}
\newcommand{\AttributeTok}[1]{\textcolor[rgb]{0.77,0.63,0.00}{#1}}
\newcommand{\RegionMarkerTok}[1]{#1}
\newcommand{\InformationTok}[1]{\textcolor[rgb]{0.56,0.35,0.01}{\textbf{\textit{#1}}}}
\newcommand{\WarningTok}[1]{\textcolor[rgb]{0.56,0.35,0.01}{\textbf{\textit{#1}}}}
\newcommand{\AlertTok}[1]{\textcolor[rgb]{0.94,0.16,0.16}{#1}}
\newcommand{\ErrorTok}[1]{\textcolor[rgb]{0.64,0.00,0.00}{\textbf{#1}}}
\newcommand{\NormalTok}[1]{#1}
\usepackage{graphicx,grffile}
\makeatletter
\def\maxwidth{\ifdim\Gin@nat@width>\linewidth\linewidth\else\Gin@nat@width\fi}
\def\maxheight{\ifdim\Gin@nat@height>\textheight\textheight\else\Gin@nat@height\fi}
\makeatother
% Scale images if necessary, so that they will not overflow the page
% margins by default, and it is still possible to overwrite the defaults
% using explicit options in \includegraphics[width, height, ...]{}
\setkeys{Gin}{width=\maxwidth,height=\maxheight,keepaspectratio}
\IfFileExists{parskip.sty}{%
\usepackage{parskip}
}{% else
\setlength{\parindent}{0pt}
\setlength{\parskip}{6pt plus 2pt minus 1pt}
}
\setlength{\emergencystretch}{3em}  % prevent overfull lines
\providecommand{\tightlist}{%
  \setlength{\itemsep}{0pt}\setlength{\parskip}{0pt}}
\setcounter{secnumdepth}{0}
% Redefines (sub)paragraphs to behave more like sections
\ifx\paragraph\undefined\else
\let\oldparagraph\paragraph
\renewcommand{\paragraph}[1]{\oldparagraph{#1}\mbox{}}
\fi
\ifx\subparagraph\undefined\else
\let\oldsubparagraph\subparagraph
\renewcommand{\subparagraph}[1]{\oldsubparagraph{#1}\mbox{}}
\fi

%%% Use protect on footnotes to avoid problems with footnotes in titles
\let\rmarkdownfootnote\footnote%
\def\footnote{\protect\rmarkdownfootnote}

%%% Change title format to be more compact
\usepackage{titling}

% Create subtitle command for use in maketitle
\newcommand{\subtitle}[1]{
  \posttitle{
    \begin{center}\large#1\end{center}
    }
}

\setlength{\droptitle}{-2em}
  \title{Noções Básicas de R - Aula 4}
  \pretitle{\vspace{\droptitle}\centering\huge}
  \posttitle{\par}
  \author{}
  \preauthor{}\postauthor{}
  \date{}
  \predate{}\postdate{}


\begin{document}
\maketitle

\section{Exemplo de uso de R + Markdown +
knitr}\label{exemplo-de-uso-de-r-markdown-knitr}

Prof.~Dr.~Cleuler Barbosa das Neves\\
\href{http://buscatextual.cnpq.br/buscatextual/visualizacv.do?id=K4786159E2}{currículo.lattes}

AULA N. 04 - OBJETOS: \emph{VETORES}, \textbf{MATRIZES},
\textbf{DATA.FRAME}, ARRAYS, \textbf{LIST}, \emph{DATE}, TS, LM etc.

\section{\texorpdfstring{\textbf{R} é uma Linguagem \textbf{funcional}
orientada para
\textbf{objetos}!}{R é uma Linguagem funcional orientada para objetos!}}\label{r-e-uma-linguagem-funcional-orientada-para-objetos}

\section{{[}================================================{]}}\label{section}

\section{{[}Faz uso de funções \& de suas composições
!!!}\label{faz-uso-de-funcoes-de-suas-composicoes}

\section{{[}Armazena\&Manipula objetos previamente
criados!!!}\label{armazenamanipula-objetos-previamente-criados}

\section{\texorpdfstring{{[}\emph{Aply} essas composições nesses
\emph{ob-jectos}
!!!}{{[}Aply essas composições nesses ob-jectos !!!}}\label{aply-essas-composicoes-nesses-ob-jectos}

\section{\texorpdfstring{{[}Há \emph{symbols} c/significados
operacionais
\emph{tipics}!!!}{{[}Há symbols c/significados operacionais tipics!!!}}\label{ha-symbols-csignificados-operacionais-tipics}

\section{\texorpdfstring{{[}CRAN c/centenas de milhares de
\emph{functions} em
\emph{packages}!!!}{{[}CRAN c/centenas de milhares de functions em packages!!!}}\label{cran-ccentenas-de-milhares-de-functions-em-packages}

\section{{[}================================================{]}}\label{section-1}

\begin{Shaded}
\begin{Highlighting}[]
\CommentTok{# As "duas" primeiras linhas de comando de um script em R (p. 13) deveriam ser:}

\CommentTok{# A 1ª Linha de comando:}
\CommentTok{# O símbolo ~ representa a abreviatura para o caminho da pasta pessoal (Linux e Windows)}
\CommentTok{#setwd("~") # Aponta o Diretório de Trabalho para a Pasta Pessoal e subpasta em que se encontra o arquivo deste script PGE-aula3.Rmd: "~/../Documents/R_CS/Aula3"}
\CommentTok{# Esse comando exibe a seguinte mensagem de alerta importante: "The working directory was changed to C:/Users/M/Documents inside a notebook chunk. The working directory will be reset when the chunk is finished running. Use the knitr root.dir option in the setup chunk to change the working directory for notebook chunks"}
\KeywordTok{getwd}\NormalTok{()    }\CommentTok{# Exibe  o Diretório de Trabalho, no caso o da Pasta Pessoal, executando uma linha de comando na janela Terminal da Console Area: "C:/Users/M/Documents/R_CS/Aula3"}
\end{Highlighting}
\end{Shaded}

\begin{verbatim}
## [1] "C:/Users/M/Documents/R_CS/Aula4"
\end{verbatim}

\begin{Shaded}
\begin{Highlighting}[]
\CommentTok{#setwd("~/../Documents/R_CS/Aula3") # Produz o mesmo efeito do código anterior}
\CommentTok{#getwd()}

\CommentTok{# A 2ª Linha de comando: é um exemplo do uso de **funções compostas** em Linguagem **R**}
\NormalTok{code<-}\DecValTok{0} \CommentTok{# somente irá resetar a Job Area se code == 1}
\ControlFlowTok{if}\NormalTok{(code}\OperatorTok{==}\DecValTok{1}\NormalTok{) }\KeywordTok{rm}\NormalTok{(}\DataTypeTok{list=}\KeywordTok{ls}\NormalTok{()) }\CommentTok{# Remove toda a list de variáveis da Job Area, i. e., dá um reset na Environment}

\CommentTok{#[=========================================================================]}
\CommentTok{#[                   Pacotes do System Library                             ]}
\CommentTok{#[=========================================================================]}

\CommentTok{#Pacotes de importação de BD}
\CommentTok{#para ativar um pacote do System Library (vem c/a instalação do R): 2.000 f's}
\KeywordTok{library}\NormalTok{(foreign) }\CommentTok{# argumento não precisa das aspas}
\CommentTok{# Para carregar Base de Dados dos aplicativos:}
\CommentTok{# Minitab, S, SAS, SPSS, Stata, Systat, Weka, dBase ...}

\CommentTok{#[=========================================================================]}
\CommentTok{#[                    Pacotes da User Library                              ]}
\CommentTok{#[=========================================================================]}

\CommentTok{#P/instalar um pacote da web (CRAN) basta executar install.packages() 1 vez}
\CommentTok{#install.packages("data.table") # Para carregar BD's de grandes dimensões}
\KeywordTok{library}\NormalTok{(data.table) }\CommentTok{# (p.53-53 do livro R_CS); argumento não precisa das aspas}
\end{Highlighting}
\end{Shaded}

\begin{verbatim}
## Warning: package 'data.table' was built under R version 3.4.4
\end{verbatim}

\begin{Shaded}
\begin{Highlighting}[]
\CommentTok{# 1- converter o arquivo para .csv usando a função fwf2csv () do pacote descr}
\CommentTok{# 2- carregar o BD com a função fread() do pacote data.table, que usa menos}
\CommentTok{#    memória que a função read.fwf() do pacote ...}
\CommentTok{#install.packages("sqldf") # p/carregar partes de BD's de grandes dimensões}
\KeywordTok{library}\NormalTok{(sqldf) }\CommentTok{# R_SC: (p. 53-54)}
\end{Highlighting}
\end{Shaded}

\begin{verbatim}
## Warning: package 'sqldf' was built under R version 3.4.4
\end{verbatim}

\begin{verbatim}
## Loading required package: gsubfn
\end{verbatim}

\begin{verbatim}
## Warning: package 'gsubfn' was built under R version 3.4.4
\end{verbatim}

\begin{verbatim}
## Loading required package: proto
\end{verbatim}

\begin{verbatim}
## Warning: package 'proto' was built under R version 3.4.4
\end{verbatim}

\begin{verbatim}
## Loading required package: RSQLite
\end{verbatim}

\begin{verbatim}
## Warning: package 'RSQLite' was built under R version 3.4.4
\end{verbatim}

\begin{Shaded}
\begin{Highlighting}[]
\CommentTok{#install.packages("descr")#Um pacote tem de ser instalado 1 vez no seu micro}
\KeywordTok{library}\NormalTok{(descr) }\CommentTok{# Ativado o pacote, suas funções são disponibilizadas p/uso}
\CommentTok{# "descr" é um pacote com funções voltadas para Estatística Descritiva}

\CommentTok{#install.packages("gdata")}
\KeywordTok{library}\NormalTok{(gdata) }\CommentTok{# pacote para manipulação de dados (BD's) (p. 45)}
\end{Highlighting}
\end{Shaded}

\begin{verbatim}
## Warning: execução do comando '"C:\PROGRA~2\LYX2~1.2\Perl\bin\perl.exe" "C:/
## Users/M/Documents/R/win-library/3.4/gdata/perl/supportedFormats.pl"' teve
## status 2
\end{verbatim}

\begin{verbatim}
## gdata: Unable to load perl libaries needed by read.xls()
## gdata: to support 'XLX' (Excel 97-2004) files.
\end{verbatim}

\begin{verbatim}
## 
\end{verbatim}

\begin{verbatim}
## gdata: Unable to load perl libaries needed by read.xls()
## gdata: to support 'XLSX' (Excel 2007+) files.
\end{verbatim}

\begin{verbatim}
## 
\end{verbatim}

\begin{verbatim}
## gdata: Run the function 'installXLSXsupport()'
## gdata: to automatically download and install the perl
## gdata: libaries needed to support Excel XLS and XLSX formats.
\end{verbatim}

\begin{verbatim}
## 
## Attaching package: 'gdata'
\end{verbatim}

\begin{verbatim}
## The following objects are masked from 'package:data.table':
## 
##     first, last
\end{verbatim}

\begin{verbatim}
## The following object is masked from 'package:stats':
## 
##     nobs
\end{verbatim}

\begin{verbatim}
## The following object is masked from 'package:utils':
## 
##     object.size
\end{verbatim}

\begin{verbatim}
## The following object is masked from 'package:base':
## 
##     startsWith
\end{verbatim}

\begin{Shaded}
\begin{Highlighting}[]
               \CommentTok{# No Windows poderá ser necessário instalar ActivePerl}
               \CommentTok{# ou outro interpretador da linguagem perl.}

\CommentTok{#install.packages("igraph") # Océu é o limite!!!!!!!!!!!!!!!!!!!!!!!!!!!!!!}
\KeywordTok{library}\NormalTok{(igraph) }\CommentTok{# pacote para Network Analysis and Visualization}
\end{Highlighting}
\end{Shaded}

\begin{verbatim}
## Warning: package 'igraph' was built under R version 3.4.4
\end{verbatim}

\begin{verbatim}
## 
## Attaching package: 'igraph'
\end{verbatim}

\begin{verbatim}
## The following objects are masked from 'package:stats':
## 
##     decompose, spectrum
\end{verbatim}

\begin{verbatim}
## The following object is masked from 'package:base':
## 
##     union
\end{verbatim}

\begin{Shaded}
\begin{Highlighting}[]
                \CommentTok{# R_CS: cap. 12- Análise de Redes Sociais (com grafos)}

\CommentTok{#install.packages("knitr")}
\KeywordTok{library}\NormalTok{(knitr) }\CommentTok{# pacote para geração de Relatório Dinâmico em R (p. 119)}
\end{Highlighting}
\end{Shaded}

\begin{verbatim}
## Warning: package 'knitr' was built under R version 3.4.4
\end{verbatim}

\begin{Shaded}
\begin{Highlighting}[]
\CommentTok{#install.packages("memisc") # para surveys}
\KeywordTok{library}\NormalTok{(memisc) }\CommentTok{# pacote para manipulação de pesquisa de dados (p. 66, 89)}
\end{Highlighting}
\end{Shaded}

\begin{verbatim}
## Loading required package: lattice
\end{verbatim}

\begin{verbatim}
## Loading required package: MASS
\end{verbatim}

\begin{verbatim}
## 
## Attaching package: 'memisc'
\end{verbatim}

\begin{verbatim}
## The following objects are masked from 'package:stats':
## 
##     contr.sum, contr.treatment, contrasts
\end{verbatim}

\begin{verbatim}
## The following object is masked from 'package:base':
## 
##     as.array
\end{verbatim}

\begin{Shaded}
\begin{Highlighting}[]
                \CommentTok{# e para apresentação de análises de seus resultados}

\CommentTok{#install.packages("rgdal") # para exibição de Mapas e dados espacializados}
\KeywordTok{library}\NormalTok{(rgdal) }\CommentTok{# R_SC: cap. 11- Mapas (p. 134-139)}
\end{Highlighting}
\end{Shaded}

\begin{verbatim}
## Warning: package 'rgdal' was built under R version 3.4.4
\end{verbatim}

\begin{verbatim}
## Loading required package: sp
\end{verbatim}

\begin{verbatim}
## Warning: package 'sp' was built under R version 3.4.4
\end{verbatim}

\begin{verbatim}
## rgdal: version: 1.2-18, (SVN revision 718)
##  Geospatial Data Abstraction Library extensions to R successfully loaded
##  Loaded GDAL runtime: GDAL 2.2.3, released 2017/11/20
##  Path to GDAL shared files: C:/Users/M/Documents/R/win-library/3.4/rgdal/gdal
##  GDAL binary built with GEOS: TRUE 
##  Loaded PROJ.4 runtime: Rel. 4.9.3, 15 August 2016, [PJ_VERSION: 493]
##  Path to PROJ.4 shared files: C:/Users/M/Documents/R/win-library/3.4/rgdal/proj
##  Linking to sp version: 1.2-7
\end{verbatim}

\begin{Shaded}
\begin{Highlighting}[]
\CommentTok{# Requer a instalação do pacote sp}
\CommentTok{#install.packages("sp")}
\KeywordTok{library}\NormalTok{(sp)}

\CommentTok{#install.packages("rmarkdown") # para instalação do RMarkdown}
\KeywordTok{library}\NormalTok{(rmarkdown) }\CommentTok{#R_SC: geração Relatórios Dinâmicos no RStudio(p. 115-19)}
\end{Highlighting}
\end{Shaded}

\begin{verbatim}
## Warning: package 'rmarkdown' was built under R version 3.4.4
\end{verbatim}

\begin{Shaded}
\begin{Highlighting}[]
\CommentTok{# Requer instalação de outros pacotes p/rodar o RMarkdown dentro do RStudio}
\CommentTok{#install.packages("htmltools") - esse não precisou, veio c/o RMarkdown}
\KeywordTok{library}\NormalTok{(htmltools) }\CommentTok{# Ferramentas para HTML}
\end{Highlighting}
\end{Shaded}

\begin{verbatim}
## Warning: package 'htmltools' was built under R version 3.4.4
\end{verbatim}

\begin{verbatim}
## 
## Attaching package: 'htmltools'
\end{verbatim}

\begin{verbatim}
## The following object is masked from 'package:memisc':
## 
##     css
\end{verbatim}

\begin{Shaded}
\begin{Highlighting}[]
\CommentTok{#install.packages("caTools")#   - esse precisou e instalou o bitops}
\KeywordTok{library}\NormalTok{(caTools)}\CommentTok{#Tools: moving windows statistics, GIF, Base64, ROC AUC etc.}
\end{Highlighting}
\end{Shaded}

\begin{verbatim}
## Warning: package 'caTools' was built under R version 3.4.4
\end{verbatim}

\begin{Shaded}
\begin{Highlighting}[]
\CommentTok{#install.packages(c("bindr", "bindrcpp", "Rcpp", "stringi")) é uma função composta}
\KeywordTok{library}\NormalTok{(bindr)}\CommentTok{# library deve ter package com comprimento 1}
\KeywordTok{library}\NormalTok{(bindrcpp)}\CommentTok{#}
\KeywordTok{library}\NormalTok{(Rcpp)}\CommentTok{#}
\KeywordTok{library}\NormalTok{(stringi)}\CommentTok{#}
\end{Highlighting}
\end{Shaded}

\begin{verbatim}
## Warning: package 'stringi' was built under R version 3.4.4
\end{verbatim}

\begin{Shaded}
\begin{Highlighting}[]
\CommentTok{#install.packages(c("cluster", "Matrix"), lib="C:/Users/cleuler-bn/Documents/R/R-3.4.4/library")}
\KeywordTok{library}\NormalTok{(cluster)}\CommentTok{#}
\KeywordTok{library}\NormalTok{(Matrix)}\CommentTok{#}

\CommentTok{#install.packages(c("financial", "FinancialInstrument", "FinancialMath"))}
\KeywordTok{library}\NormalTok{(financial)}\CommentTok{#}
\KeywordTok{library}\NormalTok{(FinancialInstrument)}\CommentTok{#}
\end{Highlighting}
\end{Shaded}

\begin{verbatim}
## Warning: package 'FinancialInstrument' was built under R version 3.4.4
\end{verbatim}

\begin{verbatim}
## Loading required package: quantmod
\end{verbatim}

\begin{verbatim}
## Warning: package 'quantmod' was built under R version 3.4.4
\end{verbatim}

\begin{verbatim}
## Loading required package: xts
\end{verbatim}

\begin{verbatim}
## Warning: package 'xts' was built under R version 3.4.4
\end{verbatim}

\begin{verbatim}
## Loading required package: zoo
\end{verbatim}

\begin{verbatim}
## Warning: package 'zoo' was built under R version 3.4.4
\end{verbatim}

\begin{verbatim}
## 
## Attaching package: 'zoo'
\end{verbatim}

\begin{verbatim}
## The following objects are masked from 'package:base':
## 
##     as.Date, as.Date.numeric
\end{verbatim}

\begin{verbatim}
## 
## Attaching package: 'xts'
\end{verbatim}

\begin{verbatim}
## The following objects are masked from 'package:gdata':
## 
##     first, last
\end{verbatim}

\begin{verbatim}
## The following objects are masked from 'package:data.table':
## 
##     first, last
\end{verbatim}

\begin{verbatim}
## Loading required package: TTR
\end{verbatim}

\begin{verbatim}
## Version 0.4-0 included new data defaults. See ?getSymbols.
\end{verbatim}

\begin{Shaded}
\begin{Highlighting}[]
\KeywordTok{library}\NormalTok{(FinancialMath)}\CommentTok{#}
\end{Highlighting}
\end{Shaded}

\begin{verbatim}
## 
## Attaching package: 'FinancialMath'
\end{verbatim}

\begin{verbatim}
## The following object is masked from 'package:FinancialInstrument':
## 
##     bond
\end{verbatim}

\begin{Shaded}
\begin{Highlighting}[]
\CommentTok{#install.packages("tinytex")#   - foi preciso instalar para gerar arquivo .pdf direto do RMarkdown}
\CommentTok{#library(tinytex)# para carregar o pacote tinytex, que gera arquivo .tex e certamente converte para .pdf}
\CommentTok{#                  Mas isso gerou uma v2.pdf no formato de uma janela do PDF, sem os marcadores do lado esquerdo!!!!!}
\CommentTok{#                  Do Jeito antigo estava melhor e gravava um .pdf na pasta R_CS/Aula1 que ao abrir no Adobe}
\CommentTok{#                  apresentou na parte esquerda da tela do Adobe todos os marcadores das secções do arquivo (melhor)!}

\CommentTok{# Um *look* na sua **Estação de Trabalho** desta sessão do **R** versão 3.4.3}
\KeywordTok{sessionInfo}\NormalTok{()}
\end{Highlighting}
\end{Shaded}

\begin{verbatim}
## R version 3.4.3 (2017-11-30)
## Platform: x86_64-w64-mingw32/x64 (64-bit)
## Running under: Windows 10 x64 (build 16299)
## 
## Matrix products: default
## 
## locale:
## [1] LC_COLLATE=Portuguese_Brazil.1252  LC_CTYPE=Portuguese_Brazil.1252   
## [3] LC_MONETARY=Portuguese_Brazil.1252 LC_NUMERIC=C                      
## [5] LC_TIME=Portuguese_Brazil.1252    
## 
## attached base packages:
## [1] stats     graphics  grDevices utils     datasets  methods   base     
## 
## other attached packages:
##  [1] FinancialMath_0.1.1       FinancialInstrument_1.3.1
##  [3] quantmod_0.4-13           TTR_0.23-3               
##  [5] xts_0.10-2                zoo_1.8-1                
##  [7] financial_0.2             Matrix_1.2-12            
##  [9] cluster_2.0.6             stringi_1.1.7            
## [11] Rcpp_0.12.15              bindrcpp_0.2             
## [13] bindr_0.1                 caTools_1.17.1           
## [15] htmltools_0.3.6           rmarkdown_1.9            
## [17] rgdal_1.2-18              sp_1.2-7                 
## [19] memisc_0.99.14.9          MASS_7.3-47              
## [21] lattice_0.20-35           knitr_1.20               
## [23] igraph_1.2.1              gdata_2.18.0             
## [25] descr_1.1.4               sqldf_0.4-11             
## [27] RSQLite_2.0               gsubfn_0.7               
## [29] proto_1.0.0               data.table_1.10.4-3      
## [31] foreign_0.8-69           
## 
## loaded via a namespace (and not attached):
##  [1] compiler_3.4.3  bitops_1.0-6    tools_3.4.3     digest_0.6.15  
##  [5] bit_1.1-12      evaluate_0.10.1 memoise_1.1.0   pkgconfig_2.0.1
##  [9] DBI_0.8         curl_3.2        yaml_2.1.18     repr_0.12.0    
## [13] stringr_1.3.0   gtools_3.5.0    rprojroot_1.3-2 bit64_0.9-7    
## [17] grid_3.4.3      tcltk_3.4.3     blob_1.1.1      magrittr_1.5   
## [21] backports_1.1.2 xtable_1.8-2    chron_2.3-52
\end{verbatim}

\begin{Shaded}
\begin{Highlighting}[]
\CommentTok{# Os interessados em assinar a *Lista Brasileira do R* -- [R-br] da **UFPR** devem [acessar](http://listas.inf.ufpr.br/cgi-bin/mailman/listinfo/r-br)}

\CommentTok{# Os interessados em compreender o pacote Knitr *Knitr: a general-purpose package for dynamic report generation in R* -- R package version 1.5 devem [acessar](http://yihui.name/knitr)}

\CommentTok{#[=========================================================================]}
\CommentTok{#[                                                                         ]}
\CommentTok{#[=========================================================================]}
\end{Highlighting}
\end{Shaded}

\section{GERANDO UMA BD - AS ALTURAS E PESOS DA
TURMA}\label{gerando-uma-bd---as-alturas-e-pesos-da-turma}

\subsection{\texorpdfstring{Criando matrizes: matrix é um tipo
}{Criando matrizes: matrix é um  tipo }}\label{criando-matrizes-matrix-e-um-tipo}

 é um conjunto de 's enfileirados por linha ou por coluna; ou seja, é um
conjunto de 's fundamentais do R; é uma estrutura de dados que permite
armazenar um conjunto de um conjunto de valores de \textbf{um mesmo tipo
e de mesmos tamanhos} sob um mesmo nome de .\\
Seus principais tipos são:\\
: \\
\\
\\
\\
\\
\\
O valor \textbf{NA} pode ser armazenado como valor NULL em qualquer um
desses tipos.\\
\#\#\#\#\#\#\#\#\#\#\#\#\#\#\#\#\#\#\#\#\#\#\#\#\#\#\#\#\#\#\#\#\#\#\#\#\#\#\#\#\#\#\#\#\#\#\#\#\#\#\#\#\#\#\#\#\#\#\#\#\#\#\#\#\#\#\#\#\#\#\#\#\#\#\#\#\#\#\#\#\#\#\#\#\#\#\#\#\#\#\#
\# CUIDADO PORQUE UM ÚNICO \textbf{NA} NUMA BD PROPAGA SUA CAPACIDADE DE
IMPEDIR QUE CÁLCULOS DE\\
\# ESTATÍSTICA SEJAM PROCESSADOS\\
\#\#\#\#\#\#\#\#\#\#\#\#\#\#\#\#\#\#\#\#\#\#\#\#\#\#\#\#\#\#\#\#\#\#\#\#\#\#\#\#\#\#\#\#\#\#\#\#\#\#\#\#\#\#\#\#\#\#\#\#\#\#\#\#\#\#\#\#\#\#\#\#\#\#\#\#\#\#\#\#\#\#\#\#\#\#\#\#\#\#\#
Os tamanhos uniformes desses 's, seus lengths(), poderão servir para
informar um dos parâmetros da dimensão da matriz resultante dessa união
de conjuntos, ou seja, poderá servir para informar ou o número de linhas
(byrows = TRUE) ou o número de colunas (byrows = FALSE) da matriz.\\
Esses parâmetros poderão ser repassados como argumento da função
dim()\textless{}-c(nrow, ncol), que transforma o que lhe é passado como
argumento em uma matriz .\\
Se o número de elementos de , length(), é igual a nrow\emph{ncol, então
o vetor argumento transformar-se-á numa matriz preenchida coluna por
coluna (by = col é seu default).\\
Se seu tamanho for menor ou maior que o número de elementos da matriz
então sera aplicada Regra da Reciclagem até o preenchimento completo da
nova n}m.\\
A função matrix(c(\ldots{}), nrow = , ncol = , byrow = TRUE) também cria
uma matriz diretamente a partir dos dados fornecidos sem que seja
preciso criar e transformar um em . Se o parâmetro byrow não for
repassado, por default, essa função também irá preencher a matriz com os
dados fornecidos coluna por coluna, valendo-se da Regra da Reciclagem
caso seja necessário.\\
Assim como no caso do também é possível dar nomes aos elementos de uma
matriz, valendo-se da função names()\textless{}- c(\ldots{}) para
aqueles e das funções rownames()\textless{}-c(\ldots{}) e
colnames()\textless{}-c(\ldots{}), sendo preciso passar valores como
argumento da função c() em todos os casos.\\
As funções cbind() e rbind() podem ser usadas para juntar dois ou mais
's ou 's por colunas ou por linhas, respectivamente.

A função \texttt{matrix()} cria uma matriz do R.\\
Seus \emph{argumentos} e \emph{defaults} são:(data = NA, nrow = 1, ncol
= 1, byrow = FALSE, dimnames = NULL).\\
data is an optional data vector or an expression vector.\\
dimnames: A dimnames attribute for the matrix: NULL or a list of length
2 giving the row and column names respectively. An empty list is treated
as NULL, and a list of length one as row names (nomes das variáveis, ou
seja, das colunas). The list can be named, and the \textbf{list names}
will be used as names for the dimensions.

Exemplos de uso dessas funções.

\section{Criando matrizes vazias de vários tipos básicos e de um tipo
especial}\label{criando-matrizes-vazias-de-varios-tipos-basicos-e-de-um-tipo-especial}

\begin{Shaded}
\begin{Highlighting}[]
\NormalTok{x <-}\StringTok{ }\KeywordTok{matrix}\NormalTok{(}\DataTypeTok{nrow =} \DecValTok{3}\NormalTok{, }\DataTypeTok{ncol =} \DecValTok{5}\NormalTok{)}
\NormalTok{y <-}\StringTok{ }\KeywordTok{matrix}\NormalTok{(}\DataTypeTok{nrow =} \DecValTok{4}\NormalTok{, }\DataTypeTok{ncol =} \DecValTok{6}\NormalTok{)}
\NormalTok{z <-}\StringTok{ }\KeywordTok{matrix}\NormalTok{(}\DataTypeTok{nrow =} \DecValTok{5}\NormalTok{, }\DataTypeTok{ncol =} \DecValTok{10}\NormalTok{)}
\NormalTok{w <-}\StringTok{ }\KeywordTok{matrix}\NormalTok{(}\DataTypeTok{nrow =} \DecValTok{10}\NormalTok{, }\DataTypeTok{ncol =} \DecValTok{5}\NormalTok{)}

\NormalTok{x}
\end{Highlighting}
\end{Shaded}

\begin{verbatim}
##      [,1] [,2] [,3] [,4] [,5]
## [1,]   NA   NA   NA   NA   NA
## [2,]   NA   NA   NA   NA   NA
## [3,]   NA   NA   NA   NA   NA
\end{verbatim}

\begin{Shaded}
\begin{Highlighting}[]
\NormalTok{y}
\end{Highlighting}
\end{Shaded}

\begin{verbatim}
##      [,1] [,2] [,3] [,4] [,5] [,6]
## [1,]   NA   NA   NA   NA   NA   NA
## [2,]   NA   NA   NA   NA   NA   NA
## [3,]   NA   NA   NA   NA   NA   NA
## [4,]   NA   NA   NA   NA   NA   NA
\end{verbatim}

\begin{Shaded}
\begin{Highlighting}[]
\NormalTok{z}
\end{Highlighting}
\end{Shaded}

\begin{verbatim}
##      [,1] [,2] [,3] [,4] [,5] [,6] [,7] [,8] [,9] [,10]
## [1,]   NA   NA   NA   NA   NA   NA   NA   NA   NA    NA
## [2,]   NA   NA   NA   NA   NA   NA   NA   NA   NA    NA
## [3,]   NA   NA   NA   NA   NA   NA   NA   NA   NA    NA
## [4,]   NA   NA   NA   NA   NA   NA   NA   NA   NA    NA
## [5,]   NA   NA   NA   NA   NA   NA   NA   NA   NA    NA
\end{verbatim}

\begin{Shaded}
\begin{Highlighting}[]
\NormalTok{w}
\end{Highlighting}
\end{Shaded}

\begin{verbatim}
##       [,1] [,2] [,3] [,4] [,5]
##  [1,]   NA   NA   NA   NA   NA
##  [2,]   NA   NA   NA   NA   NA
##  [3,]   NA   NA   NA   NA   NA
##  [4,]   NA   NA   NA   NA   NA
##  [5,]   NA   NA   NA   NA   NA
##  [6,]   NA   NA   NA   NA   NA
##  [7,]   NA   NA   NA   NA   NA
##  [8,]   NA   NA   NA   NA   NA
##  [9,]   NA   NA   NA   NA   NA
## [10,]   NA   NA   NA   NA   NA
\end{verbatim}

\begin{Shaded}
\begin{Highlighting}[]
\KeywordTok{str}\NormalTok{(x)}
\end{Highlighting}
\end{Shaded}

\begin{verbatim}
##  logi [1:3, 1:5] NA NA NA NA NA NA ...
\end{verbatim}

\begin{Shaded}
\begin{Highlighting}[]
\KeywordTok{str}\NormalTok{(y)}
\end{Highlighting}
\end{Shaded}

\begin{verbatim}
##  logi [1:4, 1:6] NA NA NA NA NA NA ...
\end{verbatim}

\begin{Shaded}
\begin{Highlighting}[]
\KeywordTok{str}\NormalTok{(z)}
\end{Highlighting}
\end{Shaded}

\begin{verbatim}
##  logi [1:5, 1:10] NA NA NA NA NA NA ...
\end{verbatim}

\begin{Shaded}
\begin{Highlighting}[]
\KeywordTok{str}\NormalTok{(}\KeywordTok{c}\NormalTok{(x,y,z))}
\end{Highlighting}
\end{Shaded}

\begin{verbatim}
##  logi [1:89] NA NA NA NA NA NA ...
\end{verbatim}

\begin{Shaded}
\begin{Highlighting}[]
\KeywordTok{str}\NormalTok{(w)}
\end{Highlighting}
\end{Shaded}

\begin{verbatim}
##  logi [1:10, 1:5] NA NA NA NA NA NA ...
\end{verbatim}

\begin{Shaded}
\begin{Highlighting}[]
\NormalTok{hoje<-}\KeywordTok{Sys.Date}\NormalTok{()}
\NormalTok{dez.semanas<-}\KeywordTok{seq}\NormalTok{(hoje, }\DataTypeTok{len =} \DecValTok{10}\NormalTok{, }\DataTypeTok{by =} \StringTok{"1 week"}\NormalTok{)}
\NormalTok{dez.semanas}
\end{Highlighting}
\end{Shaded}

\begin{verbatim}
##  [1] "2018-04-25" "2018-05-02" "2018-05-09" "2018-05-16" "2018-05-23"
##  [6] "2018-05-30" "2018-06-06" "2018-06-13" "2018-06-20" "2018-06-27"
\end{verbatim}

\begin{Shaded}
\begin{Highlighting}[]
\NormalTok{dez.semanas<-dez.semanas}\OperatorTok{-}\DecValTok{14}
\NormalTok{dez.semanas}
\end{Highlighting}
\end{Shaded}

\begin{verbatim}
##  [1] "2018-04-11" "2018-04-18" "2018-04-25" "2018-05-02" "2018-05-09"
##  [6] "2018-05-16" "2018-05-23" "2018-05-30" "2018-06-06" "2018-06-13"
\end{verbatim}

\begin{Shaded}
\begin{Highlighting}[]
\CommentTok{#w <- vector(mode = "Date", length = 6) # *Error*}
\CommentTok{# porque não é um parâmetro válido para o argumento *mode* da função vector()}

\CommentTok{# Criando um vetor de datas para servir de rótulos para nossa série temporal experimental}
\CommentTok{# dez_semanas<-seq(c("2018-04-11"), len = 10, by = "1 week") # Error}
\CommentTok{# Porque "2018-04-11" é um tipo básico <char> e não um tipo especial <Date>.}
\CommentTok{# É preciso converter <char> em <Date>. E, claro, há uma função para isso!!!}
\NormalTok{dez_semanas<-}\KeywordTok{seq.Date}\NormalTok{(}\DataTypeTok{from =} \KeywordTok{as.Date}\NormalTok{(}\StringTok{"2018-04-11"}\NormalTok{), }\DataTypeTok{len =} \DecValTok{10}\NormalTok{, }\DataTypeTok{by =} \StringTok{"1 week"}\NormalTok{)}
\NormalTok{dez_semanas<-}\KeywordTok{seq}\NormalTok{(}\DataTypeTok{from =} \KeywordTok{as.Date}\NormalTok{(}\StringTok{"2018-04-11"}\NormalTok{), }\DataTypeTok{len =} \DecValTok{10}\NormalTok{, }\DataTypeTok{by =} \StringTok{"1 week"}\NormalTok{)}
\NormalTok{dez_semanas}
\end{Highlighting}
\end{Shaded}

\begin{verbatim}
##  [1] "2018-04-11" "2018-04-18" "2018-04-25" "2018-05-02" "2018-05-09"
##  [6] "2018-05-16" "2018-05-23" "2018-05-30" "2018-06-06" "2018-06-13"
\end{verbatim}

\begin{Shaded}
\begin{Highlighting}[]
\CommentTok{# Atribuindo nomes às linhas e colunas de uma matriz}
\KeywordTok{colnames}\NormalTok{(z)<-}\KeywordTok{as.character.Date}\NormalTok{(dez_semanas)}
\NormalTok{z}
\end{Highlighting}
\end{Shaded}

\begin{verbatim}
##      2018-04-11 2018-04-18 2018-04-25 2018-05-02 2018-05-09 2018-05-16
## [1,]         NA         NA         NA         NA         NA         NA
## [2,]         NA         NA         NA         NA         NA         NA
## [3,]         NA         NA         NA         NA         NA         NA
## [4,]         NA         NA         NA         NA         NA         NA
## [5,]         NA         NA         NA         NA         NA         NA
##      2018-05-23 2018-05-30 2018-06-06 2018-06-13
## [1,]         NA         NA         NA         NA
## [2,]         NA         NA         NA         NA
## [3,]         NA         NA         NA         NA
## [4,]         NA         NA         NA         NA
## [5,]         NA         NA         NA         NA
\end{verbatim}

\begin{Shaded}
\begin{Highlighting}[]
\KeywordTok{rownames}\NormalTok{(w)<-}\KeywordTok{as.character.Date}\NormalTok{(dez_semanas)}
\NormalTok{w}
\end{Highlighting}
\end{Shaded}

\begin{verbatim}
##            [,1] [,2] [,3] [,4] [,5]
## 2018-04-11   NA   NA   NA   NA   NA
## 2018-04-18   NA   NA   NA   NA   NA
## 2018-04-25   NA   NA   NA   NA   NA
## 2018-05-02   NA   NA   NA   NA   NA
## 2018-05-09   NA   NA   NA   NA   NA
## 2018-05-16   NA   NA   NA   NA   NA
## 2018-05-23   NA   NA   NA   NA   NA
## 2018-05-30   NA   NA   NA   NA   NA
## 2018-06-06   NA   NA   NA   NA   NA
## 2018-06-13   NA   NA   NA   NA   NA
\end{verbatim}

\begin{Shaded}
\begin{Highlighting}[]
\KeywordTok{str}\NormalTok{(w) }\CommentTok{# não transforma w, que era um tipo <logi>, em uma <mtrx> <char>, mas só os names das suas colunas são um <vctr> [1:10] tipo <char>}
\end{Highlighting}
\end{Shaded}

\begin{verbatim}
##  logi [1:10, 1:5] NA NA NA NA NA NA ...
##  - attr(*, "dimnames")=List of 2
##   ..$ : chr [1:10] "2018-04-11" "2018-04-18" "2018-04-25" "2018-05-02" ...
##   ..$ : NULL
\end{verbatim}

\begin{Shaded}
\begin{Highlighting}[]
\KeywordTok{colnames}\NormalTok{(w)<-}\KeywordTok{c}\NormalTok{(}\StringTok{"altura"}\NormalTok{, }\StringTok{"peso"}\NormalTok{,}\StringTok{"IMC"}\NormalTok{,}\StringTok{"peso_max"}\NormalTok{,}\StringTok{"deltap"}\NormalTok{)}
\NormalTok{w}
\end{Highlighting}
\end{Shaded}

\begin{verbatim}
##            altura peso IMC peso_max deltap
## 2018-04-11     NA   NA  NA       NA     NA
## 2018-04-18     NA   NA  NA       NA     NA
## 2018-04-25     NA   NA  NA       NA     NA
## 2018-05-02     NA   NA  NA       NA     NA
## 2018-05-09     NA   NA  NA       NA     NA
## 2018-05-16     NA   NA  NA       NA     NA
## 2018-05-23     NA   NA  NA       NA     NA
## 2018-05-30     NA   NA  NA       NA     NA
## 2018-06-06     NA   NA  NA       NA     NA
## 2018-06-13     NA   NA  NA       NA     NA
\end{verbatim}

\begin{Shaded}
\begin{Highlighting}[]
\KeywordTok{str}\NormalTok{(w) }\CommentTok{# não transforma w, que era um tipo <logi>, em uma <mtrx> <char>, mas só os names das suas colunas permanecem um <vctr> [1:10] tipo <char> e de suas colunas outro <vctr>[1:5] tipo <char>}
\end{Highlighting}
\end{Shaded}

\begin{verbatim}
##  logi [1:10, 1:5] NA NA NA NA NA NA ...
##  - attr(*, "dimnames")=List of 2
##   ..$ : chr [1:10] "2018-04-11" "2018-04-18" "2018-04-25" "2018-05-02" ...
##   ..$ : chr [1:5] "altura" "peso" "IMC" "peso_max" ...
\end{verbatim}

\begin{Shaded}
\begin{Highlighting}[]
\CommentTok{# Mas quantas matrizes como w seriam necessárias para toda a nossa turma de R? 14 <mtrx>}
\end{Highlighting}
\end{Shaded}

\section{Criando vetores para receber variáveis de uma
BD}\label{criando-vetores-para-receber-variaveis-de-uma-bd}

\begin{Shaded}
\begin{Highlighting}[]
\NormalTok{nomes<-}\KeywordTok{c}\NormalTok{(}\StringTok{"Bernard"}\NormalTok{,}
\StringTok{"Carlos"}\NormalTok{,}
\StringTok{"Cleuler"}\NormalTok{,}
\StringTok{"Helber"}\NormalTok{,}
\StringTok{"Larissa"}\NormalTok{,}
\StringTok{"Mateus"}\NormalTok{,}
\StringTok{"Michell"}\NormalTok{,}
\StringTok{"Nayana"}\NormalTok{,}
\StringTok{"Paula"}\NormalTok{,}
\StringTok{"Rafael"}\NormalTok{,}
\StringTok{"Tatiane"}\NormalTok{,}
\StringTok{"Thiago"}\NormalTok{,}
\StringTok{"Wesley"}\NormalTok{)}
\NormalTok{h<-}\KeywordTok{c}\NormalTok{(}\FloatTok{1.74}\NormalTok{,}\FloatTok{1.63}\NormalTok{,}\FloatTok{1.77}\NormalTok{,}\FloatTok{1.75}\NormalTok{,}\OtherTok{NA}\NormalTok{,}\FloatTok{1.85}\NormalTok{,}\FloatTok{1.6}\NormalTok{,}\OtherTok{NA}\NormalTok{,}\FloatTok{1.55}\NormalTok{,}\FloatTok{1.7}\NormalTok{,}\FloatTok{1.63}\NormalTok{,}\FloatTok{1.7}\NormalTok{,}\FloatTok{1.75}\NormalTok{)}
\NormalTok{p<-}\KeywordTok{c}\NormalTok{(}\FloatTok{63.8}\NormalTok{,}
\FloatTok{79.5}\NormalTok{,}
\FloatTok{81.6}\NormalTok{,}
\FloatTok{81.3}\NormalTok{,}
\DecValTok{49}\NormalTok{,}
\FloatTok{82.7}\NormalTok{,}
\FloatTok{57.6}\NormalTok{,}
\FloatTok{56.3}\NormalTok{,}
\FloatTok{72.4}\NormalTok{,}
\FloatTok{62.1}\NormalTok{,}
\FloatTok{52.6}\NormalTok{,}
\FloatTok{82.1}\NormalTok{,}
\FloatTok{81.9}\NormalTok{)}

\NormalTok{dez_semanas[}\DecValTok{1}\NormalTok{]}
\end{Highlighting}
\end{Shaded}

\begin{verbatim}
## [1] "2018-04-11"
\end{verbatim}

\begin{Shaded}
\begin{Highlighting}[]
\NormalTok{nomes}
\end{Highlighting}
\end{Shaded}

\begin{verbatim}
##  [1] "Bernard" "Carlos"  "Cleuler" "Helber"  "Larissa" "Mateus"  "Michell"
##  [8] "Nayana"  "Paula"   "Rafael"  "Tatiane" "Thiago"  "Wesley"
\end{verbatim}

\begin{Shaded}
\begin{Highlighting}[]
\KeywordTok{length}\NormalTok{(nomes)}
\end{Highlighting}
\end{Shaded}

\begin{verbatim}
## [1] 13
\end{verbatim}

\begin{Shaded}
\begin{Highlighting}[]
\NormalTok{h}
\end{Highlighting}
\end{Shaded}

\begin{verbatim}
##  [1] 1.74 1.63 1.77 1.75   NA 1.85 1.60   NA 1.55 1.70 1.63 1.70 1.75
\end{verbatim}

\begin{Shaded}
\begin{Highlighting}[]
\KeywordTok{length}\NormalTok{(h)}
\end{Highlighting}
\end{Shaded}

\begin{verbatim}
## [1] 13
\end{verbatim}

\begin{Shaded}
\begin{Highlighting}[]
\NormalTok{p}
\end{Highlighting}
\end{Shaded}

\begin{verbatim}
##  [1] 63.8 79.5 81.6 81.3 49.0 82.7 57.6 56.3 72.4 62.1 52.6 82.1 81.9
\end{verbatim}

\begin{Shaded}
\begin{Highlighting}[]
\KeywordTok{length}\NormalTok{(p)}
\end{Highlighting}
\end{Shaded}

\begin{verbatim}
## [1] 13
\end{verbatim}

\begin{Shaded}
\begin{Highlighting}[]
\CommentTok{# Como há vetores do mesmo tipo e do mesmo comprimento eles podem ser reunidos numa matriz}
\CommentTok{# Mas nomes <char> não poderá ser reunido numa matriz com h ou p <num>.}
\NormalTok{r<-}\KeywordTok{cbind}\NormalTok{(h,p)}
\NormalTok{r}
\end{Highlighting}
\end{Shaded}

\begin{verbatim}
##          h    p
##  [1,] 1.74 63.8
##  [2,] 1.63 79.5
##  [3,] 1.77 81.6
##  [4,] 1.75 81.3
##  [5,]   NA 49.0
##  [6,] 1.85 82.7
##  [7,] 1.60 57.6
##  [8,]   NA 56.3
##  [9,] 1.55 72.4
## [10,] 1.70 62.1
## [11,] 1.63 52.6
## [12,] 1.70 82.1
## [13,] 1.75 81.9
\end{verbatim}

\begin{Shaded}
\begin{Highlighting}[]
\KeywordTok{rownames}\NormalTok{(r)<-nomes}
\NormalTok{r}
\end{Highlighting}
\end{Shaded}

\begin{verbatim}
##            h    p
## Bernard 1.74 63.8
## Carlos  1.63 79.5
## Cleuler 1.77 81.6
## Helber  1.75 81.3
## Larissa   NA 49.0
## Mateus  1.85 82.7
## Michell 1.60 57.6
## Nayana    NA 56.3
## Paula   1.55 72.4
## Rafael  1.70 62.1
## Tatiane 1.63 52.6
## Thiago  1.70 82.1
## Wesley  1.75 81.9
\end{verbatim}

\begin{Shaded}
\begin{Highlighting}[]
\NormalTok{IMC<-r[,}\DecValTok{2}\NormalTok{]}\OperatorTok{/}\NormalTok{r[,}\DecValTok{1}\NormalTok{]}\OperatorTok{^}\DecValTok{2}
\NormalTok{r<-}\KeywordTok{cbind}\NormalTok{(r,IMC)}
\NormalTok{r}
\end{Highlighting}
\end{Shaded}

\begin{verbatim}
##            h    p      IMC
## Bernard 1.74 63.8 21.07280
## Carlos  1.63 79.5 29.92209
## Cleuler 1.77 81.6 26.04616
## Helber  1.75 81.3 26.54694
## Larissa   NA 49.0       NA
## Mateus  1.85 82.7 24.16362
## Michell 1.60 57.6 22.50000
## Nayana    NA 56.3       NA
## Paula   1.55 72.4 30.13528
## Rafael  1.70 62.1 21.48789
## Tatiane 1.63 52.6 19.79751
## Thiago  1.70 82.1 28.40830
## Wesley  1.75 81.9 26.74286
\end{verbatim}

\begin{Shaded}
\begin{Highlighting}[]
\NormalTok{pmax<-}\DecValTok{25}\OperatorTok{*}\NormalTok{r[,}\StringTok{"h"}\NormalTok{]}\OperatorTok{^}\DecValTok{2}
\NormalTok{pmax}
\end{Highlighting}
\end{Shaded}

\begin{verbatim}
## Bernard  Carlos Cleuler  Helber Larissa  Mateus Michell  Nayana   Paula 
## 75.6900 66.4225 78.3225 76.5625      NA 85.5625 64.0000      NA 60.0625 
##  Rafael Tatiane  Thiago  Wesley 
## 72.2500 66.4225 72.2500 76.5625
\end{verbatim}

\begin{Shaded}
\begin{Highlighting}[]
\NormalTok{deltap<-r[,}\StringTok{"p"}\NormalTok{]}\OperatorTok{-}\NormalTok{pmax}
\CommentTok{#deltap}
\NormalTok{r<-}\KeywordTok{cbind}\NormalTok{(r,pmax,deltap)}
\NormalTok{dez_semanas[}\DecValTok{1}\NormalTok{]}
\end{Highlighting}
\end{Shaded}

\begin{verbatim}
## [1] "2018-04-11"
\end{verbatim}

\begin{Shaded}
\begin{Highlighting}[]
\NormalTok{r[,}\OperatorTok{-}\DecValTok{5}\NormalTok{]}
\end{Highlighting}
\end{Shaded}

\begin{verbatim}
##            h    p      IMC    pmax
## Bernard 1.74 63.8 21.07280 75.6900
## Carlos  1.63 79.5 29.92209 66.4225
## Cleuler 1.77 81.6 26.04616 78.3225
## Helber  1.75 81.3 26.54694 76.5625
## Larissa   NA 49.0       NA      NA
## Mateus  1.85 82.7 24.16362 85.5625
## Michell 1.60 57.6 22.50000 64.0000
## Nayana    NA 56.3       NA      NA
## Paula   1.55 72.4 30.13528 60.0625
## Rafael  1.70 62.1 21.48789 72.2500
## Tatiane 1.63 52.6 19.79751 66.4225
## Thiago  1.70 82.1 28.40830 72.2500
## Wesley  1.75 81.9 26.74286 76.5625
\end{verbatim}

\section{Exercícios - Para Resolução em
Sala}\label{exercicios---para-resolucao-em-sala}

Refletir e responder às seguintes questões \emph{pragmáticas} com o uso
de matrizes:\\
1) Qual a altura média da sua turma de R na aula do dia 11 abr. 2018?\\
2) Qual o peso médio da sua turma de R na aula do dia 11 abr. 2018?\\
3) Qual o número médio de caracteres dos prenomes dos alunos da turma de
R que mediram seus pesos no dia 11 abr. 2018? 4) Qual o número médio de
caracteres dos prenomes dos alunos matriculados nesta turma de R? 5)
Qual o desvio padrão das médias encontradas?\\
6) Quem está abaixo e acima da média mais ou menos 1 Desvio Padrão? 7)
Calculado o IMC de cada observação do dia 11 abr. 2018 encontre sua
média e dp? 8) Como criar uma estrutura de dados em R para armazenar 10
matrizes como a matriz r?

\begin{Shaded}
\begin{Highlighting}[]
\CommentTok{# Invocando as funções mean() e sd() para uma <var> <vector> <num>}
\CommentTok{#1) Média e Desvio Padrão (#5) das alturas dos alunos:}
\NormalTok{hm<-}\StringTok{ }\KeywordTok{mean}\NormalTok{(h, }\DataTypeTok{na.rm=}\OtherTok{TRUE}\NormalTok{)}
\NormalTok{hDP<-}\KeywordTok{sd}\NormalTok{(h, }\DataTypeTok{na.rm=}\OtherTok{TRUE}\NormalTok{) }\CommentTok{# Desvio padrão da altura é uma medida de dispersão dessa variável}
\CommentTok{# É uma turma com 8.7 cm de dispersão em torno da altura média de 1.70 m}
\CommentTok{# São brasileiros de estatura mediana, gostam muito de..., mas...}
\CommentTok{#6) Quem está abaixo e acima da média mais ou menos 1 Desvio Padrão?}
\NormalTok{hm}
\end{Highlighting}
\end{Shaded}

\begin{verbatim}
## [1] 1.697273
\end{verbatim}

\begin{Shaded}
\begin{Highlighting}[]
\NormalTok{hDP}
\end{Highlighting}
\end{Shaded}

\begin{verbatim}
## [1] 0.08730303
\end{verbatim}

\begin{Shaded}
\begin{Highlighting}[]
\NormalTok{h[h}\OperatorTok{<}\NormalTok{hm}\OperatorTok{-}\NormalTok{hDP }\OperatorTok{|}\StringTok{ }\NormalTok{h}\OperatorTok{>}\NormalTok{hm}\OperatorTok{+}\NormalTok{hDP]}
\end{Highlighting}
\end{Shaded}

\begin{verbatim}
## [1]   NA 1.85 1.60   NA 1.55
\end{verbatim}

\begin{Shaded}
\begin{Highlighting}[]
\NormalTok{nomes[h}\OperatorTok{<}\NormalTok{hm}\OperatorTok{-}\NormalTok{hDP }\OperatorTok{|}\StringTok{ }\NormalTok{h}\OperatorTok{>}\NormalTok{hm}\OperatorTok{+}\NormalTok{hDP] }\CommentTok{# Eis os outliers da estatura de nossa turma.}
\end{Highlighting}
\end{Shaded}

\begin{verbatim}
## [1] NA        "Mateus"  "Michell" NA        "Paula"
\end{verbatim}

\begin{Shaded}
\begin{Highlighting}[]
\CommentTok{#2) Média e Desvio Padrão (#5) dos pesos dos alunos:}
\NormalTok{pm<-}\StringTok{ }\KeywordTok{mean}\NormalTok{(p, }\DataTypeTok{na.rm=}\OtherTok{TRUE}\NormalTok{) }\CommentTok{# É uma uma turma de magros!!! Conclusão precipitada?}
\NormalTok{pDP<-}\KeywordTok{sd}\NormalTok{(p, }\DataTypeTok{na.rm=}\OtherTok{TRUE}\NormalTok{)}
\CommentTok{# A média do peso da turma no dia 11 abr. 2018 é de 69.5 Kg}
\CommentTok{# O Desvio Padrão dessas 13 obsevrações de peso   = 12.9 Kg}
\CommentTok{#6) Quem está abaixo e acima da média mais ou menos 1 Desvio Padrão?}
\NormalTok{p[p}\OperatorTok{<}\NormalTok{pm}\OperatorTok{-}\NormalTok{pDP }\OperatorTok{|}\StringTok{ }\NormalTok{p}\OperatorTok{>}\NormalTok{pm}\OperatorTok{+}\NormalTok{pDP]}
\end{Highlighting}
\end{Shaded}

\begin{verbatim}
## [1] 49.0 82.7 56.3 52.6
\end{verbatim}

\begin{Shaded}
\begin{Highlighting}[]
\NormalTok{nomes[p}\OperatorTok{<}\NormalTok{pm}\OperatorTok{-}\NormalTok{pDP }\OperatorTok{|}\StringTok{ }\NormalTok{p}\OperatorTok{>}\NormalTok{pm}\OperatorTok{+}\NormalTok{pDP] }\CommentTok{# Eis os outliers da nossa turma.}
\end{Highlighting}
\end{Shaded}

\begin{verbatim}
## [1] "Larissa" "Mateus"  "Nayana"  "Tatiane"
\end{verbatim}

\begin{Shaded}
\begin{Highlighting}[]
\CommentTok{#3) Número médio de caracteres dos prenomes dos alunos da turma de R que mediram seus pesos no dia 11 abr. 2018 e seu Desvio Padrão (#5)}
\KeywordTok{mean}\NormalTok{(}\KeywordTok{length}\NormalTok{(nomes))}
\end{Highlighting}
\end{Shaded}

\begin{verbatim}
## [1] 13
\end{verbatim}

\begin{Shaded}
\begin{Highlighting}[]
\KeywordTok{sd}\NormalTok{(}\KeywordTok{mean}\NormalTok{(}\KeywordTok{length}\NormalTok{(nomes))) }\CommentTok{# retorna um NA. Por que?}
\end{Highlighting}
\end{Shaded}

\begin{verbatim}
## [1] NA
\end{verbatim}

\begin{Shaded}
\begin{Highlighting}[]
\NormalTok{nomes}
\end{Highlighting}
\end{Shaded}

\begin{verbatim}
##  [1] "Bernard" "Carlos"  "Cleuler" "Helber"  "Larissa" "Mateus"  "Michell"
##  [8] "Nayana"  "Paula"   "Rafael"  "Tatiane" "Thiago"  "Wesley"
\end{verbatim}

\begin{Shaded}
\begin{Highlighting}[]
\NormalTok{tam_nomes<-}\KeywordTok{length}\NormalTok{(nomes) }\CommentTok{# cuidado porque retorna o comprimento do vetor names = 13!!!}
\NormalTok{tam_nomes<-}\KeywordTok{nchar}\NormalTok{(nomes, }\DataTypeTok{keepNA =} \OtherTok{NA}\NormalTok{)}
\NormalTok{tam_nomes}
\end{Highlighting}
\end{Shaded}

\begin{verbatim}
##  [1] 7 6 7 6 7 6 7 6 5 6 7 6 6
\end{verbatim}

\begin{Shaded}
\begin{Highlighting}[]
\NormalTok{tam_nomes_media<-}\KeywordTok{mean}\NormalTok{(tam_nomes)}
\NormalTok{tam_nomes_media}
\end{Highlighting}
\end{Shaded}

\begin{verbatim}
## [1] 6.307692
\end{verbatim}

\begin{Shaded}
\begin{Highlighting}[]
\NormalTok{tam_nomes_DP   <-}\KeywordTok{sd}\NormalTok{(tam_nomes)}
\NormalTok{tam_nomes_DP}
\end{Highlighting}
\end{Shaded}

\begin{verbatim}
## [1] 0.6304252
\end{verbatim}

\begin{Shaded}
\begin{Highlighting}[]
\CommentTok{#6) Quem está abaixo e acima da média mais ou menos 1 Desvio Padrão?}
\NormalTok{tam_nomes[tam_nomes}\OperatorTok{<}\NormalTok{tam_nomes_media}\OperatorTok{-}\NormalTok{tam_nomes_DP }\OperatorTok{|}\StringTok{ }\NormalTok{tam_nomes}\OperatorTok{>}\NormalTok{tam_nomes_media}\OperatorTok{+}\NormalTok{tam_nomes_DP]}
\end{Highlighting}
\end{Shaded}

\begin{verbatim}
## [1] 7 7 7 7 5 7
\end{verbatim}

\begin{Shaded}
\begin{Highlighting}[]
\NormalTok{nomes[tam_nomes}\OperatorTok{<}\NormalTok{tam_nomes_media}\OperatorTok{-}\NormalTok{tam_nomes_DP }\OperatorTok{|}\StringTok{ }\NormalTok{tam_nomes}\OperatorTok{>}\NormalTok{tam_nomes_media}\OperatorTok{+}\NormalTok{tam_nomes_DP] }\CommentTok{# Eis os nomes daqueles com tamanhos de nomes outliers da nossa turma.}
\end{Highlighting}
\end{Shaded}

\begin{verbatim}
## [1] "Bernard" "Cleuler" "Larissa" "Michell" "Paula"   "Tatiane"
\end{verbatim}

\begin{Shaded}
\begin{Highlighting}[]
\CommentTok{#7) Cálculo do IMC de cada observação do dia 11 abr. 2018.}
\CommentTok{#O cálculo do IMC é feito dividindo o peso (em quilogramas) pela altura (em metros) elevada ao quadrado.}
\NormalTok{IMC<-p}\OperatorTok{/}\NormalTok{h}\OperatorTok{^}\DecValTok{2}
\NormalTok{IMC}
\end{Highlighting}
\end{Shaded}

\begin{verbatim}
##  [1] 21.07280 29.92209 26.04616 26.54694       NA 24.16362 22.50000
##  [8]       NA 30.13528 21.48789 19.79751 28.40830 26.74286
\end{verbatim}

\begin{Shaded}
\begin{Highlighting}[]
\NormalTok{IMC_m<-}\StringTok{ }\KeywordTok{mean}\NormalTok{(IMC, }\DataTypeTok{na.rm=}\OtherTok{TRUE}\NormalTok{) }\CommentTok{# É uma uma turma de magros!!! Conclusão precipitada?}
\NormalTok{IMC_m }\CommentTok{# = 25.17 Kg/m2     # O IMC médio da turma indica ligeiramente acima do peso normal}
\end{Highlighting}
\end{Shaded}

\begin{verbatim}
## [1] 25.16577
\end{verbatim}

\begin{Shaded}
\begin{Highlighting}[]
\NormalTok{IMC_DP<-}\KeywordTok{sd}\NormalTok{(IMC, }\DataTypeTok{na.rm=}\OtherTok{TRUE}\NormalTok{)}
\NormalTok{IMC_DP}\CommentTok{# =  3.61 Kg/m2}
\end{Highlighting}
\end{Shaded}

\begin{verbatim}
## [1] 3.608471
\end{verbatim}

\begin{Shaded}
\begin{Highlighting}[]
\NormalTok{IMC[IMC}\OperatorTok{<}\FloatTok{18.5} \OperatorTok{|}\StringTok{ }\NormalTok{IMC}\OperatorTok{>=}\DecValTok{25}\NormalTok{]}
\end{Highlighting}
\end{Shaded}

\begin{verbatim}
## [1] 29.92209 26.04616 26.54694       NA       NA 30.13528 28.40830 26.74286
\end{verbatim}

\begin{Shaded}
\begin{Highlighting}[]
\NormalTok{nomes[IMC}\OperatorTok{<}\FloatTok{18.5} \OperatorTok{|}\StringTok{ }\NormalTok{IMC}\OperatorTok{>=}\DecValTok{25}\NormalTok{] }\CommentTok{# Revelar ou não revelar. Eis a questão!!!!!!!!!!!!!!!!!!!!!!!!!!}
\end{Highlighting}
\end{Shaded}

\begin{verbatim}
## [1] "Carlos"  "Cleuler" "Helber"  NA        NA        "Paula"   "Thiago" 
## [8] "Wesley"
\end{verbatim}

\begin{Shaded}
\begin{Highlighting}[]
\CommentTok{# cut() Convert Numeric <num> to Factor <fctr>}
\NormalTok{###########################################################################################}
\CommentTok{# CUIDADO PORQUE UM ÚNICO ERRO DE SINTAXE FAZ COM QUE O COMPILADOR INTERROMPA A EXECUÇÃO DO SRCIPT (CÓDIGO FONTE)  #}
\NormalTok{###########################################################################################}
\end{Highlighting}
\end{Shaded}

Exercíco da Aula n. 2: 8) Redija e salve um script para a função linear
em \textbf{R}. Gere um gráfico para essa função no intervalo {[}0,5{]} e
salve-o no formato .pdf.

\begin{Shaded}
\begin{Highlighting}[]
\NormalTok{###########################################################################################}
\CommentTok{# TENTATIVAS E  ERROS PARA GERAR UM GRÁFICO y=f(x)=a.x + b }
\CommentTok{#}
\CommentTok{# IMPORTANDO UM ARQUIVO FEITO PELO BERNARD E CONVERTENDO-O DE UTF-8 PARA WINDOWS-1252}
\CommentTok{#}
\NormalTok{###########################################################################################}

\KeywordTok{library}\NormalTok{(descr)}
\KeywordTok{library}\NormalTok{(stats)}
\KeywordTok{getwd}\NormalTok{()}
\end{Highlighting}
\end{Shaded}

\begin{verbatim}
## [1] "C:/Users/M/Documents/R_CS/Aula4"
\end{verbatim}

\begin{Shaded}
\begin{Highlighting}[]
\CommentTok{#linhas<-readLines("Plottar_grafico.R")}
\CommentTok{#linhas<-fromUTF8(linhas)}
\CommentTok{#writeLines(linhas, "Plotar_grafico-win.R")}

\CommentTok{#Script desenvolvido para criar graficos}
\CommentTok{#Criar função da equação da reta > y = ax+b}
\CommentTok{#------------------Parametros------------------}
\CommentTok{#     a = Coeficiente linear}
\CommentTok{#     b = Coeficiente Angular}
\CommentTok{#     x = Vetor de valores no Eixo X}
\CommentTok{#     y = Vetor de valores no Eixo Y}

\CommentTok{#Cria funcao que representa a equacao da reta}
\NormalTok{linear <-}\StringTok{ }\ControlFlowTok{function}\NormalTok{(a,b,x)\{}
\NormalTok{  y <-}\StringTok{ }\NormalTok{a}\OperatorTok{*}\NormalTok{x }\OperatorTok{+}\StringTok{ }\NormalTok{b}
  \KeywordTok{return}\NormalTok{(y)}
\NormalTok{\}}

\NormalTok{a <-}\StringTok{ }\FloatTok{1.5} \CommentTok{#Coeficiente linear}
\NormalTok{b <-}\StringTok{ }\FloatTok{0.5} \CommentTok{#Coeficiente angular}

\CommentTok{#x < 1:10 #Vetor de valores do Eixo X # HAVIA UM ERROR. NÃO DE SINTAXE, MAS DE PROGRAMAÇÃO}
\NormalTok{x <-}\StringTok{ }\DecValTok{1}\OperatorTok{:}\DecValTok{10} \CommentTok{#Vetor de valores do Eixo X}

\CommentTok{#y <- linear(x) # OCORREU OUTRO ERROR. AO CHAMAR A FUNÇÃO linear()}
\NormalTok{y <-}\StringTok{ }\KeywordTok{linear}\NormalTok{(a,b,x) }\CommentTok{# é preciso repassar os parâmetros dos argumentos a e b da função}

\KeywordTok{print}\NormalTok{(y) }\CommentTok{#Mostrar os valores do Eixo Y}
\end{Highlighting}
\end{Shaded}

\begin{verbatim}
##  [1]  2.0  3.5  5.0  6.5  8.0  9.5 11.0 12.5 14.0 15.5
\end{verbatim}

\begin{Shaded}
\begin{Highlighting}[]
\CommentTok{#Parametros do plot}
\CommentTok{#     main = Titulo do grafico}
\CommentTok{#     ylab = Nome do Eixo Y}
\CommentTok{#     xlab = Nome do Eixo X}
\CommentTok{#     type = Tipo de plotagem > l = linha,p = pontos, h = histograma}
\KeywordTok{plot}\NormalTok{(x,y,}\DataTypeTok{main=}\StringTok{'Gráfico Curso R'}\NormalTok{,}\DataTypeTok{ylab=}\StringTok{'Eixo y'}\NormalTok{,}\DataTypeTok{xlab=}\StringTok{'Eixo x'}\NormalTok{,}\DataTypeTok{type=}\StringTok{'o'}\NormalTok{)}
\end{Highlighting}
\end{Shaded}

\includegraphics{PGE-aula4_files/figure-latex/reta-1.pdf}

\begin{Shaded}
\begin{Highlighting}[]
\CommentTok{# Redesenhando o mesmo Gráfico}
\NormalTok{x <-}\StringTok{ }\DecValTok{0}\OperatorTok{:}\DecValTok{10} \CommentTok{#Vetor de valores do Eixo X}
\NormalTok{y <-}\StringTok{ }\KeywordTok{linear}\NormalTok{(a,b,x) }\CommentTok{# é preciso repassar os parâmetros dos argumentos a e b da função}
\KeywordTok{print}\NormalTok{(y) }\CommentTok{#Mostrar os valores do Eixo Y}
\end{Highlighting}
\end{Shaded}

\begin{verbatim}
##  [1]  0.5  2.0  3.5  5.0  6.5  8.0  9.5 11.0 12.5 14.0 15.5
\end{verbatim}

\begin{Shaded}
\begin{Highlighting}[]
\NormalTok{y1 <-}\StringTok{ }\KeywordTok{linear}\NormalTok{(}\DecValTok{2}\NormalTok{,}\DecValTok{0}\NormalTok{,x)}
\NormalTok{y2 <-}\StringTok{ }\KeywordTok{linear}\NormalTok{(}\FloatTok{0.5}\NormalTok{,}\DecValTok{0}\NormalTok{,x)}

\KeywordTok{plot.new}\NormalTok{()}
\KeywordTok{plot}\NormalTok{(x,y,}\DataTypeTok{xlim=}\KeywordTok{c}\NormalTok{(}\DecValTok{0}\NormalTok{,}\DecValTok{10}\NormalTok{),}\DataTypeTok{ylim=}\KeywordTok{c}\NormalTok{(}\DecValTok{0}\NormalTok{,}\DecValTok{16}\NormalTok{),}\DataTypeTok{main=}\StringTok{'Gráfico Curso R'}\NormalTok{,}\DataTypeTok{ylab=}\StringTok{'y'}\NormalTok{,}\DataTypeTok{xlab=}\StringTok{'x'}\NormalTok{,}\DataTypeTok{type=}\StringTok{'o'}\NormalTok{)}
\end{Highlighting}
\end{Shaded}

\includegraphics{PGE-aula4_files/figure-latex/reta-2.pdf}

\begin{Shaded}
\begin{Highlighting}[]
\CommentTok{#lines(x,y1, col="blue") # Error: plot.new has not been called yet}
\CommentTok{#lines(x,y2, col="red")}
\end{Highlighting}
\end{Shaded}

Exercícios remanescentes da Aula n. 01:\\
9) Apresente duas funções lineraes que sejam inversas. Plote-as
juntamente com a função identidade. 10) Descrever os tipos de variáveis
geradas na Job Area e suas características.

\textbf{Trabalho Final do curso: } Na primeira aula registrar a altura
declarada e medir e registrar o peso de cada aluno, que poderá
identificar-se com um apelido.\\
Em cada aula medir e registrar numa BD o peso de cada aluno numa
sequencia aleatória.\\
Calcular o IMC de cada observação e apontar para os IMC's abaixo ou
acima da faixa recomendada pela literatura médica.\\
Calcular a média e o desvio padrão do IMC da população observada;
detectar os pontos \emph{outliers}.\\
Gerar um série temporal, com período de 7 dias, ao longo dos nossos 10
encontros.\\
Tratar eventuais NA's.\\
\emph{Descrever} a variação do IMC médio da turma ao longo do curso,
dado que será feito um apelo geral para aqueles acima da média para
tentarem reduzi-lo nas próximas 10 semanas.\\
Fazer uma regressão linear do IMC médio em função do tempo analisando se
ele sofreu alguma variação estatisticamente significativa.\\
\emph{Inferir} qual resultado seria alcançado se o curso durasse 20
semanas.

\section{Objetos}\label{objetos}

\subsection{Vetores tipo factor}\label{vetores-tipo-factor}

\begin{Shaded}
\begin{Highlighting}[]
\CommentTok{# O que é um vetor do tipo factor}
\CommentTok{# usado para variáveis categóricas}
\CommentTok{# Que apresenta vávios Levels (níveis)}
\CommentTok{# Comumente cada nível recebe um nome gerando um conjunto denominado Labels}

\CommentTok{# Exemplo: No nosso estudo de caso seria interessante separar os dados amostrado segundo o sexo biológico de cada obsevração, porque biologicamente seres humanos XY tam estatura média superioe às dos XX. Logo também o IMC deve ser tratado para cada subgrupo separadamente.}

\NormalTok{s<-}\KeywordTok{c}\NormalTok{(}\StringTok{"m"}\NormalTok{,}\StringTok{"m"}\NormalTok{,}\StringTok{"m"}\NormalTok{,}\StringTok{"m"}\NormalTok{,}\StringTok{"f"}\NormalTok{,}\StringTok{"m"}\NormalTok{,}\StringTok{"m"}\NormalTok{,}\StringTok{"f"}\NormalTok{,}\StringTok{"f"}\NormalTok{,}\StringTok{"m"}\NormalTok{,}\StringTok{"f"}\NormalTok{,}\StringTok{"m"}\NormalTok{,}\StringTok{"m"}\NormalTok{)}
\NormalTok{s }\CommentTok{# um <vctr> do tipo <chr>}
\end{Highlighting}
\end{Shaded}

\begin{verbatim}
##  [1] "m" "m" "m" "m" "f" "m" "m" "f" "f" "m" "f" "m" "m"
\end{verbatim}

\begin{Shaded}
\begin{Highlighting}[]
\KeywordTok{mode}\NormalTok{(s)}
\end{Highlighting}
\end{Shaded}

\begin{verbatim}
## [1] "character"
\end{verbatim}

\begin{Shaded}
\begin{Highlighting}[]
\KeywordTok{class}\NormalTok{(s)}
\end{Highlighting}
\end{Shaded}

\begin{verbatim}
## [1] "character"
\end{verbatim}

\begin{Shaded}
\begin{Highlighting}[]
\KeywordTok{length}\NormalTok{(s)}
\end{Highlighting}
\end{Shaded}

\begin{verbatim}
## [1] 13
\end{verbatim}

\begin{Shaded}
\begin{Highlighting}[]
\KeywordTok{summary}\NormalTok{(s)}
\end{Highlighting}
\end{Shaded}

\begin{verbatim}
##    Length     Class      Mode 
##        13 character character
\end{verbatim}

\begin{Shaded}
\begin{Highlighting}[]
\KeywordTok{str}\NormalTok{(s)}
\end{Highlighting}
\end{Shaded}

\begin{verbatim}
##  chr [1:13] "m" "m" "m" "m" "f" "m" "m" "f" "f" "m" "f" "m" "m"
\end{verbatim}

\begin{Shaded}
\begin{Highlighting}[]
\KeywordTok{dput}\NormalTok{(s)}
\end{Highlighting}
\end{Shaded}

\begin{verbatim}
## c("m", "m", "m", "m", "f", "m", "m", "f", "f", "m", "f", "m", 
## "m")
\end{verbatim}

\begin{Shaded}
\begin{Highlighting}[]
\CommentTok{# Transformando numa variável factor <fctr>}
\NormalTok{s<-}\KeywordTok{as.factor}\NormalTok{(s) }\CommentTok{# Destroi <chr> e recria o vetor s como um <fctr>}
\NormalTok{s}
\end{Highlighting}
\end{Shaded}

\begin{verbatim}
##  [1] m m m m f m m f f m f m m
## Levels: f m
\end{verbatim}

\begin{Shaded}
\begin{Highlighting}[]
\KeywordTok{mode}\NormalTok{(s) }\CommentTok{# é um vetor do tipo <numeric>}
\end{Highlighting}
\end{Shaded}

\begin{verbatim}
## [1] "numeric"
\end{verbatim}

\begin{Shaded}
\begin{Highlighting}[]
\KeywordTok{class}\NormalTok{(s) }\CommentTok{# é um factor <fctr>, que é um caso especial de <numeric> indexado a Labels}
\end{Highlighting}
\end{Shaded}

\begin{verbatim}
## [1] "factor"
\end{verbatim}

\begin{Shaded}
\begin{Highlighting}[]
\KeywordTok{length}\NormalTok{(s)}
\end{Highlighting}
\end{Shaded}

\begin{verbatim}
## [1] 13
\end{verbatim}

\begin{Shaded}
\begin{Highlighting}[]
\KeywordTok{summary}\NormalTok{(s)}
\end{Highlighting}
\end{Shaded}

\begin{verbatim}
## f m 
## 4 9
\end{verbatim}

\begin{Shaded}
\begin{Highlighting}[]
\KeywordTok{str}\NormalTok{(s) }\CommentTok{# investigando a structure da variável s do tipo <fctr>}
\end{Highlighting}
\end{Shaded}

\begin{verbatim}
##  Factor w/ 2 levels "f","m": 2 2 2 2 1 2 2 1 1 2 ...
\end{verbatim}

\begin{Shaded}
\begin{Highlighting}[]
\KeywordTok{dput}\NormalTok{(s)}
\end{Highlighting}
\end{Shaded}

\begin{verbatim}
## structure(c(2L, 2L, 2L, 2L, 1L, 2L, 2L, 1L, 1L, 2L, 1L, 2L, 2L
## ), .Label = c("f", "m"), class = "factor")
\end{verbatim}

\begin{Shaded}
\begin{Highlighting}[]
\KeywordTok{table}\NormalTok{(s) }\CommentTok{# retorna um vetor tipo <fctr> com a frequência de cada um dos níveis (Levels) ou categorias do vetor que é repassado como parâmetro do argumento da função table()}
\end{Highlighting}
\end{Shaded}

\begin{verbatim}
## s
## f m 
## 4 9
\end{verbatim}

\begin{Shaded}
\begin{Highlighting}[]
\CommentTok{# Essa mesma função é usada para retornar tabulações cruzadas (cross table) de duas variáveis categóricas}
\KeywordTok{max}\NormalTok{(h, }\DataTypeTok{na.rm =} \OtherTok{TRUE}\NormalTok{)}
\end{Highlighting}
\end{Shaded}

\begin{verbatim}
## [1] 1.85
\end{verbatim}

\begin{Shaded}
\begin{Highlighting}[]
\NormalTok{hcat <-}\StringTok{ }\KeywordTok{cut}\NormalTok{(h,}\KeywordTok{c}\NormalTok{(}\DecValTok{0}\NormalTok{,}\FloatTok{1.6}\NormalTok{,}\FloatTok{1.7}\NormalTok{,}\FloatTok{2.0}\NormalTok{),}\DataTypeTok{labels =} \KeywordTok{c}\NormalTok{(}\StringTok{"Baixo"}\NormalTok{,}\StringTok{"Médio"}\NormalTok{,}\StringTok{"Alto"}\NormalTok{))}
\CommentTok{# função cat() Convert Numeric to Factor}
\KeywordTok{str}\NormalTok{(hcat)}
\end{Highlighting}
\end{Shaded}

\begin{verbatim}
##  Factor w/ 3 levels "Baixo","Médio",..: 3 2 3 3 NA 3 1 NA 1 2 ...
\end{verbatim}

\begin{Shaded}
\begin{Highlighting}[]
\KeywordTok{dput}\NormalTok{(hcat)}
\end{Highlighting}
\end{Shaded}

\begin{verbatim}
## structure(c(3L, 2L, 3L, 3L, NA, 3L, 1L, NA, 1L, 2L, 2L, 2L, 3L
## ), .Label = c("Baixo", "Médio", "Alto"), class = "factor")
\end{verbatim}

\begin{Shaded}
\begin{Highlighting}[]
\KeywordTok{table}\NormalTok{(hcat,s)}
\end{Highlighting}
\end{Shaded}

\begin{verbatim}
##        s
## hcat    f m
##   Baixo 1 1
##   Médio 1 3
##   Alto  0 5
\end{verbatim}

\begin{Shaded}
\begin{Highlighting}[]
\NormalTok{hm}
\end{Highlighting}
\end{Shaded}

\begin{verbatim}
## [1] 1.697273
\end{verbatim}

\begin{Shaded}
\begin{Highlighting}[]
\NormalTok{ct<-}\KeywordTok{table}\NormalTok{(hcat,s)}
\KeywordTok{prop.table}\NormalTok{(ct,}\DecValTok{1}\NormalTok{)}
\end{Highlighting}
\end{Shaded}

\begin{verbatim}
##        s
## hcat       f    m
##   Baixo 0.50 0.50
##   Médio 0.25 0.75
##   Alto  0.00 1.00
\end{verbatim}

\begin{Shaded}
\begin{Highlighting}[]
\KeywordTok{prop.table}\NormalTok{(ct,}\DecValTok{2}\NormalTok{)}
\end{Highlighting}
\end{Shaded}

\begin{verbatim}
##        s
## hcat            f         m
##   Baixo 0.5000000 0.1111111
##   Médio 0.5000000 0.3333333
##   Alto  0.0000000 0.5555556
\end{verbatim}

\begin{Shaded}
\begin{Highlighting}[]
\KeywordTok{prop.table}\NormalTok{(ct)}
\end{Highlighting}
\end{Shaded}

\begin{verbatim}
##        s
## hcat             f          m
##   Baixo 0.09090909 0.09090909
##   Médio 0.09090909 0.27272727
##   Alto  0.00000000 0.45454545
\end{verbatim}

\begin{Shaded}
\begin{Highlighting}[]
\DecValTok{100}\OperatorTok{*}\KeywordTok{prop.table}\NormalTok{(ct)}
\end{Highlighting}
\end{Shaded}

\begin{verbatim}
##        s
## hcat            f         m
##   Baixo  9.090909  9.090909
##   Médio  9.090909 27.272727
##   Alto   0.000000 45.454545
\end{verbatim}

\begin{Shaded}
\begin{Highlighting}[]
\CommentTok{# Analisando o resultados dessas cross tables p.u. vê-se que o IMC deve ser categorizado em feminino (XX) e masculino (XY)}

\CommentTok{# Calculando a altura média das observações s == f}
\NormalTok{s}\OperatorTok{==}\StringTok{"f"}
\end{Highlighting}
\end{Shaded}

\begin{verbatim}
##  [1] FALSE FALSE FALSE FALSE  TRUE FALSE FALSE  TRUE  TRUE FALSE  TRUE
## [12] FALSE FALSE
\end{verbatim}

\begin{Shaded}
\begin{Highlighting}[]
\NormalTok{h[s}\OperatorTok{==}\StringTok{"f"}\NormalTok{]}
\end{Highlighting}
\end{Shaded}

\begin{verbatim}
## [1]   NA   NA 1.55 1.63
\end{verbatim}

\begin{Shaded}
\begin{Highlighting}[]
\KeywordTok{mean}\NormalTok{(h[s}\OperatorTok{==}\StringTok{"f"}\NormalTok{], }\DataTypeTok{na.rm=}\OtherTok{TRUE}\NormalTok{) }\CommentTok{# é média da estatura do sexo feminino  = 1.59 m}
\end{Highlighting}
\end{Shaded}

\begin{verbatim}
## [1] 1.59
\end{verbatim}

\begin{Shaded}
\begin{Highlighting}[]
\KeywordTok{mean}\NormalTok{(h[s}\OperatorTok{==}\StringTok{"m"}\NormalTok{], }\DataTypeTok{na.rm=}\OtherTok{TRUE}\NormalTok{) }\CommentTok{# é média da estatura do sexo masculino = 1.72 m}
\end{Highlighting}
\end{Shaded}

\begin{verbatim}
## [1] 1.721111
\end{verbatim}

\begin{Shaded}
\begin{Highlighting}[]
\CommentTok{# Exibindo essa diferença graficamente}
\KeywordTok{boxplot}\NormalTok{(h}\OperatorTok{~}\NormalTok{s, }\DataTypeTok{ylab =} \StringTok{"altura"}\NormalTok{) }\CommentTok{# homens são, em média, mais alto que as mulheres}
\end{Highlighting}
\end{Shaded}

\includegraphics{PGE-aula4_files/figure-latex/unnamed-chunk-1-1.pdf}

\begin{Shaded}
\begin{Highlighting}[]
\KeywordTok{boxplot}\NormalTok{(p}\OperatorTok{~}\NormalTok{s, }\DataTypeTok{ylab =} \StringTok{"peso"}\NormalTok{) }\CommentTok{# homens são, em média, mais pesados que as mulheres}
\end{Highlighting}
\end{Shaded}

\includegraphics{PGE-aula4_files/figure-latex/unnamed-chunk-1-2.pdf}

\begin{Shaded}
\begin{Highlighting}[]
\CommentTok{# Esses gráficos corroboram uma Hipótese de estratificação f & m para analisar o IMC?????}
\CommentTok{# Duvidar é preciso.}
\CommentTok{# Transformar sua dúvida nums hipótese testável.}
\CommentTok{# E testar adequadamente a Hipótese **contra** as observações colhidas no campo.}

\KeywordTok{boxplot}\NormalTok{(IMC}\OperatorTok{~}\NormalTok{s, }\DataTypeTok{ylab =} \StringTok{"IMC"}\NormalTok{)}
\end{Highlighting}
\end{Shaded}

\includegraphics{PGE-aula4_files/figure-latex/unnamed-chunk-1-3.pdf}

\subsection{Matriz}\label{matriz}

Conjunto de elementos dispostos em linhas e colunas, em que todos os
elementos são do mesmo tipo.\\
Conjuntos de conjunto de elementos do mesmo tipo (logical, numeric,
integer, double, character, ts, lm etc.) que tenham o mesmo comprimento.

\begin{enumerate}
\def\labelenumi{\arabic{enumi}.}
\tightlist
\item
  A forma mais simples de se criar uma matriz é usar a função
  \texttt{matrix()}, sendo qua definição do seu tipo depende da extensão
  do nome da como um argumento informado como abaixo (default é logic).
  Observe-se que não se valeu nesses exemplos do comando atribuição.
\end{enumerate}

\begin{Shaded}
\begin{Highlighting}[]
\NormalTok{mat.num  =}\StringTok{ }\KeywordTok{matrix}\NormalTok{(}\KeywordTok{c}\NormalTok{(}\DecValTok{1}\OperatorTok{:}\DecValTok{16}\NormalTok{),}\DecValTok{4}\NormalTok{,}\DecValTok{4}\NormalTok{)}

\NormalTok{mat.num}
\end{Highlighting}
\end{Shaded}

\begin{verbatim}
##      [,1] [,2] [,3] [,4]
## [1,]    1    5    9   13
## [2,]    2    6   10   14
## [3,]    3    7   11   15
## [4,]    4    8   12   16
\end{verbatim}

\begin{Shaded}
\begin{Highlighting}[]
\NormalTok{mat.char =}\StringTok{ }\KeywordTok{matrix}\NormalTok{(LETTERS[}\DecValTok{1}\OperatorTok{:}\DecValTok{4}\NormalTok{],}\DecValTok{2}\NormalTok{,}\DecValTok{2}\NormalTok{)}

\NormalTok{mat.char}
\end{Highlighting}
\end{Shaded}

\begin{verbatim}
##      [,1] [,2]
## [1,] "A"  "C" 
## [2,] "B"  "D"
\end{verbatim}

\begin{Shaded}
\begin{Highlighting}[]
\NormalTok{w <-}\StringTok{ }\KeywordTok{matrix}\NormalTok{(}\DataTypeTok{nrow =} \DecValTok{10}\NormalTok{, }\DataTypeTok{ncol =} \DecValTok{5}\NormalTok{)}
\NormalTok{w}
\end{Highlighting}
\end{Shaded}

\begin{verbatim}
##       [,1] [,2] [,3] [,4] [,5]
##  [1,]   NA   NA   NA   NA   NA
##  [2,]   NA   NA   NA   NA   NA
##  [3,]   NA   NA   NA   NA   NA
##  [4,]   NA   NA   NA   NA   NA
##  [5,]   NA   NA   NA   NA   NA
##  [6,]   NA   NA   NA   NA   NA
##  [7,]   NA   NA   NA   NA   NA
##  [8,]   NA   NA   NA   NA   NA
##  [9,]   NA   NA   NA   NA   NA
## [10,]   NA   NA   NA   NA   NA
\end{verbatim}

\begin{Shaded}
\begin{Highlighting}[]
\KeywordTok{str}\NormalTok{(w)}
\end{Highlighting}
\end{Shaded}

\begin{verbatim}
##  logi [1:10, 1:5] NA NA NA NA NA NA ...
\end{verbatim}

\begin{Shaded}
\begin{Highlighting}[]
\CommentTok{# Atribuindo nomes às linhas e colunas de uma matriz}
\KeywordTok{rownames}\NormalTok{(w)<-}\KeywordTok{as.character.Date}\NormalTok{(dez_semanas)}
\NormalTok{w}
\end{Highlighting}
\end{Shaded}

\begin{verbatim}
##            [,1] [,2] [,3] [,4] [,5]
## 2018-04-11   NA   NA   NA   NA   NA
## 2018-04-18   NA   NA   NA   NA   NA
## 2018-04-25   NA   NA   NA   NA   NA
## 2018-05-02   NA   NA   NA   NA   NA
## 2018-05-09   NA   NA   NA   NA   NA
## 2018-05-16   NA   NA   NA   NA   NA
## 2018-05-23   NA   NA   NA   NA   NA
## 2018-05-30   NA   NA   NA   NA   NA
## 2018-06-06   NA   NA   NA   NA   NA
## 2018-06-13   NA   NA   NA   NA   NA
\end{verbatim}

\begin{Shaded}
\begin{Highlighting}[]
\KeywordTok{colnames}\NormalTok{(w)<-}\KeywordTok{c}\NormalTok{(}\StringTok{"altura"}\NormalTok{, }\StringTok{"peso"}\NormalTok{,}\StringTok{"IMC"}\NormalTok{,}\StringTok{"peso_max"}\NormalTok{,}\StringTok{"deltap"}\NormalTok{)}
\NormalTok{w}
\end{Highlighting}
\end{Shaded}

\begin{verbatim}
##            altura peso IMC peso_max deltap
## 2018-04-11     NA   NA  NA       NA     NA
## 2018-04-18     NA   NA  NA       NA     NA
## 2018-04-25     NA   NA  NA       NA     NA
## 2018-05-02     NA   NA  NA       NA     NA
## 2018-05-09     NA   NA  NA       NA     NA
## 2018-05-16     NA   NA  NA       NA     NA
## 2018-05-23     NA   NA  NA       NA     NA
## 2018-05-30     NA   NA  NA       NA     NA
## 2018-06-06     NA   NA  NA       NA     NA
## 2018-06-13     NA   NA  NA       NA     NA
\end{verbatim}

\begin{Shaded}
\begin{Highlighting}[]
\KeywordTok{str}\NormalTok{(w) }\CommentTok{# não transforma w, que era um tipo <logi>, em uma <mtrx> <char>, mas só os names das suas colunas permanecem um <vctr> [1:10] tipo <char> e de suas colunas outro <vctr>[1:5] tipo <char>}
\end{Highlighting}
\end{Shaded}

\begin{verbatim}
##  logi [1:10, 1:5] NA NA NA NA NA NA ...
##  - attr(*, "dimnames")=List of 2
##   ..$ : chr [1:10] "2018-04-11" "2018-04-18" "2018-04-25" "2018-05-02" ...
##   ..$ : chr [1:5] "altura" "peso" "IMC" "peso_max" ...
\end{verbatim}

\begin{Shaded}
\begin{Highlighting}[]
\CommentTok{# Mas quantas matrizes como w seriam necessárias para toda a nossa turma de R? 14 <mtrx>}
\end{Highlighting}
\end{Shaded}

\section{Manipulando Matrizes}\label{manipulando-matrizes}

\begin{Shaded}
\begin{Highlighting}[]
\CommentTok{#Criando nomes para as linhas de uma matriz}

\KeywordTok{rownames}\NormalTok{(mat.num) =}\StringTok{ }\KeywordTok{c}\NormalTok{(}\StringTok{"Sao Paulo"}\NormalTok{, }\StringTok{"Americana"}\NormalTok{, }\StringTok{"Piracicaba"}\NormalTok{, }\StringTok{"Madson"}\NormalTok{ )}

\KeywordTok{colnames}\NormalTok{(mat.num) =}\StringTok{ }\DecValTok{1}\OperatorTok{:}\DecValTok{4}

\NormalTok{mat.num}
\end{Highlighting}
\end{Shaded}

\begin{verbatim}
##            1 2  3  4
## Sao Paulo  1 5  9 13
## Americana  2 6 10 14
## Piracicaba 3 7 11 15
## Madson     4 8 12 16
\end{verbatim}

\begin{Shaded}
\begin{Highlighting}[]
\CommentTok{#Multiplicação elemento a elemento}

\NormalTok{mat.num2 =}\StringTok{ }\KeywordTok{diag}\NormalTok{(}\KeywordTok{seq}\NormalTok{(}\DecValTok{10}\NormalTok{,}\DecValTok{40}\NormalTok{,}\DataTypeTok{by=}\DecValTok{10}\NormalTok{))}

\NormalTok{mat.num2}
\end{Highlighting}
\end{Shaded}

\begin{verbatim}
##      [,1] [,2] [,3] [,4]
## [1,]   10    0    0    0
## [2,]    0   20    0    0
## [3,]    0    0   30    0
## [4,]    0    0    0   40
\end{verbatim}

\begin{Shaded}
\begin{Highlighting}[]
\NormalTok{mat.num3 =}\StringTok{ }\NormalTok{mat.num }\OperatorTok{*}\StringTok{ }\NormalTok{mat.num2}

\NormalTok{mat.num3}
\end{Highlighting}
\end{Shaded}

\begin{verbatim}
##             1   2   3   4
## Sao Paulo  10   0   0   0
## Americana   0 120   0   0
## Piracicaba  0   0 330   0
## Madson      0   0   0 640
\end{verbatim}

\begin{Shaded}
\begin{Highlighting}[]
\CommentTok{#Multiplicação de Matrizes}

\NormalTok{iden =}\StringTok{ }\KeywordTok{diag}\NormalTok{(}\DecValTok{4}\NormalTok{)}

\NormalTok{iden}
\end{Highlighting}
\end{Shaded}

\begin{verbatim}
##      [,1] [,2] [,3] [,4]
## [1,]    1    0    0    0
## [2,]    0    1    0    0
## [3,]    0    0    1    0
## [4,]    0    0    0    1
\end{verbatim}

\begin{Shaded}
\begin{Highlighting}[]
\NormalTok{mat.num}\OperatorTok\NormalTok{iden}
\end{Highlighting}
\end{Shaded}

\begin{verbatim}
##            [,1] [,2] [,3] [,4]
## Sao Paulo     1    5    9   13
## Americana     2    6   10   14
## Piracicaba    3    7   11   15
## Madson        4    8   12   16
\end{verbatim}

\begin{Shaded}
\begin{Highlighting}[]
\CommentTok{#Acessando elementos das matrizes }

\CommentTok{#Um elemento}
\NormalTok{mat.num[}\DecValTok{1}\NormalTok{,}\DecValTok{1}\NormalTok{]}
\end{Highlighting}
\end{Shaded}

\begin{verbatim}
## [1] 1
\end{verbatim}

\begin{Shaded}
\begin{Highlighting}[]
\CommentTok{#Linhas }
\NormalTok{mat.num[}\DecValTok{1}\NormalTok{,]}
\end{Highlighting}
\end{Shaded}

\begin{verbatim}
##  1  2  3  4 
##  1  5  9 13
\end{verbatim}

\begin{Shaded}
\begin{Highlighting}[]
\CommentTok{#Colunas}
\NormalTok{mat.num[,}\DecValTok{3}\NormalTok{]}
\end{Highlighting}
\end{Shaded}

\begin{verbatim}
##  Sao Paulo  Americana Piracicaba     Madson 
##          9         10         11         12
\end{verbatim}

\begin{Shaded}
\begin{Highlighting}[]
\CommentTok{#Sub Matrizes}

\NormalTok{mat.num[}\KeywordTok{c}\NormalTok{(}\DecValTok{1}\NormalTok{,}\DecValTok{3}\NormalTok{,}\DecValTok{4}\NormalTok{), }\KeywordTok{c}\NormalTok{(}\DecValTok{2}\NormalTok{,}\DecValTok{1}\NormalTok{,}\DecValTok{4}\NormalTok{)]}
\end{Highlighting}
\end{Shaded}

\begin{verbatim}
##            2 1  4
## Sao Paulo  5 1 13
## Piracicaba 7 3 15
## Madson     8 4 16
\end{verbatim}

\begin{Shaded}
\begin{Highlighting}[]
\NormalTok{mat.num[}\KeywordTok{c}\NormalTok{(T,F,T,T), }\KeywordTok{c}\NormalTok{(T,T,F,T)]}
\end{Highlighting}
\end{Shaded}

\begin{verbatim}
##            1 2  4
## Sao Paulo  1 5 13
## Piracicaba 3 7 15
## Madson     4 8 16
\end{verbatim}

\begin{Shaded}
\begin{Highlighting}[]
\NormalTok{mat.num[}\OperatorTok{-}\KeywordTok{c}\NormalTok{(}\DecValTok{1}\NormalTok{,}\DecValTok{3}\NormalTok{,}\DecValTok{4}\NormalTok{), }\OperatorTok{-}\KeywordTok{c}\NormalTok{(}\DecValTok{2}\NormalTok{,}\DecValTok{1}\NormalTok{,}\DecValTok{4}\NormalTok{)]}
\end{Highlighting}
\end{Shaded}

\begin{verbatim}
## [1] 10
\end{verbatim}

\section{Arrays}\label{arrays}

São extensões das matrizes para mais do que duas dimensões que permitem,
desde que sejam todas de um mesmo tipo ou ou ou ou ou ou etc., a reunião
de vários conjuntos dessas matrizes de mesmo tipo e dimensão em uma
outra dimensão (a rigos em várias outras dimensões)!!!

\begin{Shaded}
\begin{Highlighting}[]
\CommentTok{# Construindo um exemplo}
\NormalTok{a<-}\StringTok{ }\KeywordTok{array}\NormalTok{(}\DecValTok{1}\OperatorTok{:}\DecValTok{50}\NormalTok{, }\DataTypeTok{dim =} \KeywordTok{c}\NormalTok{(}\DecValTok{2}\NormalTok{,}\DecValTok{5}\NormalTok{,}\DecValTok{5}\NormalTok{))}
\NormalTok{a}
\end{Highlighting}
\end{Shaded}

\begin{verbatim}
## , , 1
## 
##      [,1] [,2] [,3] [,4] [,5]
## [1,]    1    3    5    7    9
## [2,]    2    4    6    8   10
## 
## , , 2
## 
##      [,1] [,2] [,3] [,4] [,5]
## [1,]   11   13   15   17   19
## [2,]   12   14   16   18   20
## 
## , , 3
## 
##      [,1] [,2] [,3] [,4] [,5]
## [1,]   21   23   25   27   29
## [2,]   22   24   26   28   30
## 
## , , 4
## 
##      [,1] [,2] [,3] [,4] [,5]
## [1,]   31   33   35   37   39
## [2,]   32   34   36   38   40
## 
## , , 5
## 
##      [,1] [,2] [,3] [,4] [,5]
## [1,]   41   43   45   47   49
## [2,]   42   44   46   48   50
\end{verbatim}

\begin{Shaded}
\begin{Highlighting}[]
\NormalTok{r[,}\OperatorTok{-}\DecValTok{5}\NormalTok{]}
\end{Highlighting}
\end{Shaded}

\begin{verbatim}
##            h    p      IMC    pmax
## Bernard 1.74 63.8 21.07280 75.6900
## Carlos  1.63 79.5 29.92209 66.4225
## Cleuler 1.77 81.6 26.04616 78.3225
## Helber  1.75 81.3 26.54694 76.5625
## Larissa   NA 49.0       NA      NA
## Mateus  1.85 82.7 24.16362 85.5625
## Michell 1.60 57.6 22.50000 64.0000
## Nayana    NA 56.3       NA      NA
## Paula   1.55 72.4 30.13528 60.0625
## Rafael  1.70 62.1 21.48789 72.2500
## Tatiane 1.63 52.6 19.79751 66.4225
## Thiago  1.70 82.1 28.40830 72.2500
## Wesley  1.75 81.9 26.74286 76.5625
\end{verbatim}

\begin{Shaded}
\begin{Highlighting}[]
\NormalTok{rs<-}\KeywordTok{array}\NormalTok{(}\DataTypeTok{dim =} \KeywordTok{c}\NormalTok{(}\DecValTok{13}\NormalTok{,}\DecValTok{5}\NormalTok{,}\DecValTok{10}\NormalTok{))}
\NormalTok{rs}
\end{Highlighting}
\end{Shaded}

\begin{verbatim}
## , , 1
## 
##       [,1] [,2] [,3] [,4] [,5]
##  [1,]   NA   NA   NA   NA   NA
##  [2,]   NA   NA   NA   NA   NA
##  [3,]   NA   NA   NA   NA   NA
##  [4,]   NA   NA   NA   NA   NA
##  [5,]   NA   NA   NA   NA   NA
##  [6,]   NA   NA   NA   NA   NA
##  [7,]   NA   NA   NA   NA   NA
##  [8,]   NA   NA   NA   NA   NA
##  [9,]   NA   NA   NA   NA   NA
## [10,]   NA   NA   NA   NA   NA
## [11,]   NA   NA   NA   NA   NA
## [12,]   NA   NA   NA   NA   NA
## [13,]   NA   NA   NA   NA   NA
## 
## , , 2
## 
##       [,1] [,2] [,3] [,4] [,5]
##  [1,]   NA   NA   NA   NA   NA
##  [2,]   NA   NA   NA   NA   NA
##  [3,]   NA   NA   NA   NA   NA
##  [4,]   NA   NA   NA   NA   NA
##  [5,]   NA   NA   NA   NA   NA
##  [6,]   NA   NA   NA   NA   NA
##  [7,]   NA   NA   NA   NA   NA
##  [8,]   NA   NA   NA   NA   NA
##  [9,]   NA   NA   NA   NA   NA
## [10,]   NA   NA   NA   NA   NA
## [11,]   NA   NA   NA   NA   NA
## [12,]   NA   NA   NA   NA   NA
## [13,]   NA   NA   NA   NA   NA
## 
## , , 3
## 
##       [,1] [,2] [,3] [,4] [,5]
##  [1,]   NA   NA   NA   NA   NA
##  [2,]   NA   NA   NA   NA   NA
##  [3,]   NA   NA   NA   NA   NA
##  [4,]   NA   NA   NA   NA   NA
##  [5,]   NA   NA   NA   NA   NA
##  [6,]   NA   NA   NA   NA   NA
##  [7,]   NA   NA   NA   NA   NA
##  [8,]   NA   NA   NA   NA   NA
##  [9,]   NA   NA   NA   NA   NA
## [10,]   NA   NA   NA   NA   NA
## [11,]   NA   NA   NA   NA   NA
## [12,]   NA   NA   NA   NA   NA
## [13,]   NA   NA   NA   NA   NA
## 
## , , 4
## 
##       [,1] [,2] [,3] [,4] [,5]
##  [1,]   NA   NA   NA   NA   NA
##  [2,]   NA   NA   NA   NA   NA
##  [3,]   NA   NA   NA   NA   NA
##  [4,]   NA   NA   NA   NA   NA
##  [5,]   NA   NA   NA   NA   NA
##  [6,]   NA   NA   NA   NA   NA
##  [7,]   NA   NA   NA   NA   NA
##  [8,]   NA   NA   NA   NA   NA
##  [9,]   NA   NA   NA   NA   NA
## [10,]   NA   NA   NA   NA   NA
## [11,]   NA   NA   NA   NA   NA
## [12,]   NA   NA   NA   NA   NA
## [13,]   NA   NA   NA   NA   NA
## 
## , , 5
## 
##       [,1] [,2] [,3] [,4] [,5]
##  [1,]   NA   NA   NA   NA   NA
##  [2,]   NA   NA   NA   NA   NA
##  [3,]   NA   NA   NA   NA   NA
##  [4,]   NA   NA   NA   NA   NA
##  [5,]   NA   NA   NA   NA   NA
##  [6,]   NA   NA   NA   NA   NA
##  [7,]   NA   NA   NA   NA   NA
##  [8,]   NA   NA   NA   NA   NA
##  [9,]   NA   NA   NA   NA   NA
## [10,]   NA   NA   NA   NA   NA
## [11,]   NA   NA   NA   NA   NA
## [12,]   NA   NA   NA   NA   NA
## [13,]   NA   NA   NA   NA   NA
## 
## , , 6
## 
##       [,1] [,2] [,3] [,4] [,5]
##  [1,]   NA   NA   NA   NA   NA
##  [2,]   NA   NA   NA   NA   NA
##  [3,]   NA   NA   NA   NA   NA
##  [4,]   NA   NA   NA   NA   NA
##  [5,]   NA   NA   NA   NA   NA
##  [6,]   NA   NA   NA   NA   NA
##  [7,]   NA   NA   NA   NA   NA
##  [8,]   NA   NA   NA   NA   NA
##  [9,]   NA   NA   NA   NA   NA
## [10,]   NA   NA   NA   NA   NA
## [11,]   NA   NA   NA   NA   NA
## [12,]   NA   NA   NA   NA   NA
## [13,]   NA   NA   NA   NA   NA
## 
## , , 7
## 
##       [,1] [,2] [,3] [,4] [,5]
##  [1,]   NA   NA   NA   NA   NA
##  [2,]   NA   NA   NA   NA   NA
##  [3,]   NA   NA   NA   NA   NA
##  [4,]   NA   NA   NA   NA   NA
##  [5,]   NA   NA   NA   NA   NA
##  [6,]   NA   NA   NA   NA   NA
##  [7,]   NA   NA   NA   NA   NA
##  [8,]   NA   NA   NA   NA   NA
##  [9,]   NA   NA   NA   NA   NA
## [10,]   NA   NA   NA   NA   NA
## [11,]   NA   NA   NA   NA   NA
## [12,]   NA   NA   NA   NA   NA
## [13,]   NA   NA   NA   NA   NA
## 
## , , 8
## 
##       [,1] [,2] [,3] [,4] [,5]
##  [1,]   NA   NA   NA   NA   NA
##  [2,]   NA   NA   NA   NA   NA
##  [3,]   NA   NA   NA   NA   NA
##  [4,]   NA   NA   NA   NA   NA
##  [5,]   NA   NA   NA   NA   NA
##  [6,]   NA   NA   NA   NA   NA
##  [7,]   NA   NA   NA   NA   NA
##  [8,]   NA   NA   NA   NA   NA
##  [9,]   NA   NA   NA   NA   NA
## [10,]   NA   NA   NA   NA   NA
## [11,]   NA   NA   NA   NA   NA
## [12,]   NA   NA   NA   NA   NA
## [13,]   NA   NA   NA   NA   NA
## 
## , , 9
## 
##       [,1] [,2] [,3] [,4] [,5]
##  [1,]   NA   NA   NA   NA   NA
##  [2,]   NA   NA   NA   NA   NA
##  [3,]   NA   NA   NA   NA   NA
##  [4,]   NA   NA   NA   NA   NA
##  [5,]   NA   NA   NA   NA   NA
##  [6,]   NA   NA   NA   NA   NA
##  [7,]   NA   NA   NA   NA   NA
##  [8,]   NA   NA   NA   NA   NA
##  [9,]   NA   NA   NA   NA   NA
## [10,]   NA   NA   NA   NA   NA
## [11,]   NA   NA   NA   NA   NA
## [12,]   NA   NA   NA   NA   NA
## [13,]   NA   NA   NA   NA   NA
## 
## , , 10
## 
##       [,1] [,2] [,3] [,4] [,5]
##  [1,]   NA   NA   NA   NA   NA
##  [2,]   NA   NA   NA   NA   NA
##  [3,]   NA   NA   NA   NA   NA
##  [4,]   NA   NA   NA   NA   NA
##  [5,]   NA   NA   NA   NA   NA
##  [6,]   NA   NA   NA   NA   NA
##  [7,]   NA   NA   NA   NA   NA
##  [8,]   NA   NA   NA   NA   NA
##  [9,]   NA   NA   NA   NA   NA
## [10,]   NA   NA   NA   NA   NA
## [11,]   NA   NA   NA   NA   NA
## [12,]   NA   NA   NA   NA   NA
## [13,]   NA   NA   NA   NA   NA
\end{verbatim}

\begin{Shaded}
\begin{Highlighting}[]
\NormalTok{rs[,,}\DecValTok{1}\NormalTok{]<-r }\CommentTok{# Transaforma um array tipo logic num numeric.}
\NormalTok{rs[,}\OperatorTok{-}\DecValTok{5}\NormalTok{,]}
\end{Highlighting}
\end{Shaded}

\begin{verbatim}
## , , 1
## 
##       [,1] [,2]     [,3]    [,4]
##  [1,] 1.74 63.8 21.07280 75.6900
##  [2,] 1.63 79.5 29.92209 66.4225
##  [3,] 1.77 81.6 26.04616 78.3225
##  [4,] 1.75 81.3 26.54694 76.5625
##  [5,]   NA 49.0       NA      NA
##  [6,] 1.85 82.7 24.16362 85.5625
##  [7,] 1.60 57.6 22.50000 64.0000
##  [8,]   NA 56.3       NA      NA
##  [9,] 1.55 72.4 30.13528 60.0625
## [10,] 1.70 62.1 21.48789 72.2500
## [11,] 1.63 52.6 19.79751 66.4225
## [12,] 1.70 82.1 28.40830 72.2500
## [13,] 1.75 81.9 26.74286 76.5625
## 
## , , 2
## 
##       [,1] [,2] [,3] [,4]
##  [1,]   NA   NA   NA   NA
##  [2,]   NA   NA   NA   NA
##  [3,]   NA   NA   NA   NA
##  [4,]   NA   NA   NA   NA
##  [5,]   NA   NA   NA   NA
##  [6,]   NA   NA   NA   NA
##  [7,]   NA   NA   NA   NA
##  [8,]   NA   NA   NA   NA
##  [9,]   NA   NA   NA   NA
## [10,]   NA   NA   NA   NA
## [11,]   NA   NA   NA   NA
## [12,]   NA   NA   NA   NA
## [13,]   NA   NA   NA   NA
## 
## , , 3
## 
##       [,1] [,2] [,3] [,4]
##  [1,]   NA   NA   NA   NA
##  [2,]   NA   NA   NA   NA
##  [3,]   NA   NA   NA   NA
##  [4,]   NA   NA   NA   NA
##  [5,]   NA   NA   NA   NA
##  [6,]   NA   NA   NA   NA
##  [7,]   NA   NA   NA   NA
##  [8,]   NA   NA   NA   NA
##  [9,]   NA   NA   NA   NA
## [10,]   NA   NA   NA   NA
## [11,]   NA   NA   NA   NA
## [12,]   NA   NA   NA   NA
## [13,]   NA   NA   NA   NA
## 
## , , 4
## 
##       [,1] [,2] [,3] [,4]
##  [1,]   NA   NA   NA   NA
##  [2,]   NA   NA   NA   NA
##  [3,]   NA   NA   NA   NA
##  [4,]   NA   NA   NA   NA
##  [5,]   NA   NA   NA   NA
##  [6,]   NA   NA   NA   NA
##  [7,]   NA   NA   NA   NA
##  [8,]   NA   NA   NA   NA
##  [9,]   NA   NA   NA   NA
## [10,]   NA   NA   NA   NA
## [11,]   NA   NA   NA   NA
## [12,]   NA   NA   NA   NA
## [13,]   NA   NA   NA   NA
## 
## , , 5
## 
##       [,1] [,2] [,3] [,4]
##  [1,]   NA   NA   NA   NA
##  [2,]   NA   NA   NA   NA
##  [3,]   NA   NA   NA   NA
##  [4,]   NA   NA   NA   NA
##  [5,]   NA   NA   NA   NA
##  [6,]   NA   NA   NA   NA
##  [7,]   NA   NA   NA   NA
##  [8,]   NA   NA   NA   NA
##  [9,]   NA   NA   NA   NA
## [10,]   NA   NA   NA   NA
## [11,]   NA   NA   NA   NA
## [12,]   NA   NA   NA   NA
## [13,]   NA   NA   NA   NA
## 
## , , 6
## 
##       [,1] [,2] [,3] [,4]
##  [1,]   NA   NA   NA   NA
##  [2,]   NA   NA   NA   NA
##  [3,]   NA   NA   NA   NA
##  [4,]   NA   NA   NA   NA
##  [5,]   NA   NA   NA   NA
##  [6,]   NA   NA   NA   NA
##  [7,]   NA   NA   NA   NA
##  [8,]   NA   NA   NA   NA
##  [9,]   NA   NA   NA   NA
## [10,]   NA   NA   NA   NA
## [11,]   NA   NA   NA   NA
## [12,]   NA   NA   NA   NA
## [13,]   NA   NA   NA   NA
## 
## , , 7
## 
##       [,1] [,2] [,3] [,4]
##  [1,]   NA   NA   NA   NA
##  [2,]   NA   NA   NA   NA
##  [3,]   NA   NA   NA   NA
##  [4,]   NA   NA   NA   NA
##  [5,]   NA   NA   NA   NA
##  [6,]   NA   NA   NA   NA
##  [7,]   NA   NA   NA   NA
##  [8,]   NA   NA   NA   NA
##  [9,]   NA   NA   NA   NA
## [10,]   NA   NA   NA   NA
## [11,]   NA   NA   NA   NA
## [12,]   NA   NA   NA   NA
## [13,]   NA   NA   NA   NA
## 
## , , 8
## 
##       [,1] [,2] [,3] [,4]
##  [1,]   NA   NA   NA   NA
##  [2,]   NA   NA   NA   NA
##  [3,]   NA   NA   NA   NA
##  [4,]   NA   NA   NA   NA
##  [5,]   NA   NA   NA   NA
##  [6,]   NA   NA   NA   NA
##  [7,]   NA   NA   NA   NA
##  [8,]   NA   NA   NA   NA
##  [9,]   NA   NA   NA   NA
## [10,]   NA   NA   NA   NA
## [11,]   NA   NA   NA   NA
## [12,]   NA   NA   NA   NA
## [13,]   NA   NA   NA   NA
## 
## , , 9
## 
##       [,1] [,2] [,3] [,4]
##  [1,]   NA   NA   NA   NA
##  [2,]   NA   NA   NA   NA
##  [3,]   NA   NA   NA   NA
##  [4,]   NA   NA   NA   NA
##  [5,]   NA   NA   NA   NA
##  [6,]   NA   NA   NA   NA
##  [7,]   NA   NA   NA   NA
##  [8,]   NA   NA   NA   NA
##  [9,]   NA   NA   NA   NA
## [10,]   NA   NA   NA   NA
## [11,]   NA   NA   NA   NA
## [12,]   NA   NA   NA   NA
## [13,]   NA   NA   NA   NA
## 
## , , 10
## 
##       [,1] [,2] [,3] [,4]
##  [1,]   NA   NA   NA   NA
##  [2,]   NA   NA   NA   NA
##  [3,]   NA   NA   NA   NA
##  [4,]   NA   NA   NA   NA
##  [5,]   NA   NA   NA   NA
##  [6,]   NA   NA   NA   NA
##  [7,]   NA   NA   NA   NA
##  [8,]   NA   NA   NA   NA
##  [9,]   NA   NA   NA   NA
## [10,]   NA   NA   NA   NA
## [11,]   NA   NA   NA   NA
## [12,]   NA   NA   NA   NA
## [13,]   NA   NA   NA   NA
\end{verbatim}

\begin{Shaded}
\begin{Highlighting}[]
\NormalTok{nomes}
\end{Highlighting}
\end{Shaded}

\begin{verbatim}
##  [1] "Bernard" "Carlos"  "Cleuler" "Helber"  "Larissa" "Mateus"  "Michell"
##  [8] "Nayana"  "Paula"   "Rafael"  "Tatiane" "Thiago"  "Wesley"
\end{verbatim}

\begin{Shaded}
\begin{Highlighting}[]
\KeywordTok{rownames}\NormalTok{(rs[,,}\DecValTok{1}\OperatorTok{:}\DecValTok{10}\NormalTok{])<-nomes}
\KeywordTok{colnames}\NormalTok{(rs[,,}\DecValTok{1}\OperatorTok{:}\DecValTok{10}\NormalTok{])<-}\KeywordTok{c}\NormalTok{(}\StringTok{"altura"}\NormalTok{, }\StringTok{"peso"}\NormalTok{,}\StringTok{"IMC"}\NormalTok{,}\StringTok{"peso_max"}\NormalTok{,}\StringTok{"deltap"}\NormalTok{)}
\NormalTok{rs[,,}\DecValTok{1}\NormalTok{] }\CommentTok{# As funções rownames e colnames não funcionam com um array.}
\end{Highlighting}
\end{Shaded}

\begin{verbatim}
##       [,1] [,2]     [,3]    [,4]     [,5]
##  [1,] 1.74 63.8 21.07280 75.6900 -11.8900
##  [2,] 1.63 79.5 29.92209 66.4225  13.0775
##  [3,] 1.77 81.6 26.04616 78.3225   3.2775
##  [4,] 1.75 81.3 26.54694 76.5625   4.7375
##  [5,]   NA 49.0       NA      NA       NA
##  [6,] 1.85 82.7 24.16362 85.5625  -2.8625
##  [7,] 1.60 57.6 22.50000 64.0000  -6.4000
##  [8,]   NA 56.3       NA      NA       NA
##  [9,] 1.55 72.4 30.13528 60.0625  12.3375
## [10,] 1.70 62.1 21.48789 72.2500 -10.1500
## [11,] 1.63 52.6 19.79751 66.4225 -13.8225
## [12,] 1.70 82.1 28.40830 72.2500   9.8500
## [13,] 1.75 81.9 26.74286 76.5625   5.3375
\end{verbatim}

\begin{Shaded}
\begin{Highlighting}[]
\NormalTok{rs<-}\KeywordTok{array}\NormalTok{(r,}\DataTypeTok{dim =} \KeywordTok{c}\NormalTok{(}\DecValTok{13}\NormalTok{,}\DecValTok{5}\NormalTok{,}\DecValTok{10}\NormalTok{))}
\NormalTok{rs }\CommentTok{# o o bjeto array não herda os nomes de colunas e linhas da matriz r.}
\end{Highlighting}
\end{Shaded}

\begin{verbatim}
## , , 1
## 
##       [,1] [,2]     [,3]    [,4]     [,5]
##  [1,] 1.74 63.8 21.07280 75.6900 -11.8900
##  [2,] 1.63 79.5 29.92209 66.4225  13.0775
##  [3,] 1.77 81.6 26.04616 78.3225   3.2775
##  [4,] 1.75 81.3 26.54694 76.5625   4.7375
##  [5,]   NA 49.0       NA      NA       NA
##  [6,] 1.85 82.7 24.16362 85.5625  -2.8625
##  [7,] 1.60 57.6 22.50000 64.0000  -6.4000
##  [8,]   NA 56.3       NA      NA       NA
##  [9,] 1.55 72.4 30.13528 60.0625  12.3375
## [10,] 1.70 62.1 21.48789 72.2500 -10.1500
## [11,] 1.63 52.6 19.79751 66.4225 -13.8225
## [12,] 1.70 82.1 28.40830 72.2500   9.8500
## [13,] 1.75 81.9 26.74286 76.5625   5.3375
## 
## , , 2
## 
##       [,1] [,2]     [,3]    [,4]     [,5]
##  [1,] 1.74 63.8 21.07280 75.6900 -11.8900
##  [2,] 1.63 79.5 29.92209 66.4225  13.0775
##  [3,] 1.77 81.6 26.04616 78.3225   3.2775
##  [4,] 1.75 81.3 26.54694 76.5625   4.7375
##  [5,]   NA 49.0       NA      NA       NA
##  [6,] 1.85 82.7 24.16362 85.5625  -2.8625
##  [7,] 1.60 57.6 22.50000 64.0000  -6.4000
##  [8,]   NA 56.3       NA      NA       NA
##  [9,] 1.55 72.4 30.13528 60.0625  12.3375
## [10,] 1.70 62.1 21.48789 72.2500 -10.1500
## [11,] 1.63 52.6 19.79751 66.4225 -13.8225
## [12,] 1.70 82.1 28.40830 72.2500   9.8500
## [13,] 1.75 81.9 26.74286 76.5625   5.3375
## 
## , , 3
## 
##       [,1] [,2]     [,3]    [,4]     [,5]
##  [1,] 1.74 63.8 21.07280 75.6900 -11.8900
##  [2,] 1.63 79.5 29.92209 66.4225  13.0775
##  [3,] 1.77 81.6 26.04616 78.3225   3.2775
##  [4,] 1.75 81.3 26.54694 76.5625   4.7375
##  [5,]   NA 49.0       NA      NA       NA
##  [6,] 1.85 82.7 24.16362 85.5625  -2.8625
##  [7,] 1.60 57.6 22.50000 64.0000  -6.4000
##  [8,]   NA 56.3       NA      NA       NA
##  [9,] 1.55 72.4 30.13528 60.0625  12.3375
## [10,] 1.70 62.1 21.48789 72.2500 -10.1500
## [11,] 1.63 52.6 19.79751 66.4225 -13.8225
## [12,] 1.70 82.1 28.40830 72.2500   9.8500
## [13,] 1.75 81.9 26.74286 76.5625   5.3375
## 
## , , 4
## 
##       [,1] [,2]     [,3]    [,4]     [,5]
##  [1,] 1.74 63.8 21.07280 75.6900 -11.8900
##  [2,] 1.63 79.5 29.92209 66.4225  13.0775
##  [3,] 1.77 81.6 26.04616 78.3225   3.2775
##  [4,] 1.75 81.3 26.54694 76.5625   4.7375
##  [5,]   NA 49.0       NA      NA       NA
##  [6,] 1.85 82.7 24.16362 85.5625  -2.8625
##  [7,] 1.60 57.6 22.50000 64.0000  -6.4000
##  [8,]   NA 56.3       NA      NA       NA
##  [9,] 1.55 72.4 30.13528 60.0625  12.3375
## [10,] 1.70 62.1 21.48789 72.2500 -10.1500
## [11,] 1.63 52.6 19.79751 66.4225 -13.8225
## [12,] 1.70 82.1 28.40830 72.2500   9.8500
## [13,] 1.75 81.9 26.74286 76.5625   5.3375
## 
## , , 5
## 
##       [,1] [,2]     [,3]    [,4]     [,5]
##  [1,] 1.74 63.8 21.07280 75.6900 -11.8900
##  [2,] 1.63 79.5 29.92209 66.4225  13.0775
##  [3,] 1.77 81.6 26.04616 78.3225   3.2775
##  [4,] 1.75 81.3 26.54694 76.5625   4.7375
##  [5,]   NA 49.0       NA      NA       NA
##  [6,] 1.85 82.7 24.16362 85.5625  -2.8625
##  [7,] 1.60 57.6 22.50000 64.0000  -6.4000
##  [8,]   NA 56.3       NA      NA       NA
##  [9,] 1.55 72.4 30.13528 60.0625  12.3375
## [10,] 1.70 62.1 21.48789 72.2500 -10.1500
## [11,] 1.63 52.6 19.79751 66.4225 -13.8225
## [12,] 1.70 82.1 28.40830 72.2500   9.8500
## [13,] 1.75 81.9 26.74286 76.5625   5.3375
## 
## , , 6
## 
##       [,1] [,2]     [,3]    [,4]     [,5]
##  [1,] 1.74 63.8 21.07280 75.6900 -11.8900
##  [2,] 1.63 79.5 29.92209 66.4225  13.0775
##  [3,] 1.77 81.6 26.04616 78.3225   3.2775
##  [4,] 1.75 81.3 26.54694 76.5625   4.7375
##  [5,]   NA 49.0       NA      NA       NA
##  [6,] 1.85 82.7 24.16362 85.5625  -2.8625
##  [7,] 1.60 57.6 22.50000 64.0000  -6.4000
##  [8,]   NA 56.3       NA      NA       NA
##  [9,] 1.55 72.4 30.13528 60.0625  12.3375
## [10,] 1.70 62.1 21.48789 72.2500 -10.1500
## [11,] 1.63 52.6 19.79751 66.4225 -13.8225
## [12,] 1.70 82.1 28.40830 72.2500   9.8500
## [13,] 1.75 81.9 26.74286 76.5625   5.3375
## 
## , , 7
## 
##       [,1] [,2]     [,3]    [,4]     [,5]
##  [1,] 1.74 63.8 21.07280 75.6900 -11.8900
##  [2,] 1.63 79.5 29.92209 66.4225  13.0775
##  [3,] 1.77 81.6 26.04616 78.3225   3.2775
##  [4,] 1.75 81.3 26.54694 76.5625   4.7375
##  [5,]   NA 49.0       NA      NA       NA
##  [6,] 1.85 82.7 24.16362 85.5625  -2.8625
##  [7,] 1.60 57.6 22.50000 64.0000  -6.4000
##  [8,]   NA 56.3       NA      NA       NA
##  [9,] 1.55 72.4 30.13528 60.0625  12.3375
## [10,] 1.70 62.1 21.48789 72.2500 -10.1500
## [11,] 1.63 52.6 19.79751 66.4225 -13.8225
## [12,] 1.70 82.1 28.40830 72.2500   9.8500
## [13,] 1.75 81.9 26.74286 76.5625   5.3375
## 
## , , 8
## 
##       [,1] [,2]     [,3]    [,4]     [,5]
##  [1,] 1.74 63.8 21.07280 75.6900 -11.8900
##  [2,] 1.63 79.5 29.92209 66.4225  13.0775
##  [3,] 1.77 81.6 26.04616 78.3225   3.2775
##  [4,] 1.75 81.3 26.54694 76.5625   4.7375
##  [5,]   NA 49.0       NA      NA       NA
##  [6,] 1.85 82.7 24.16362 85.5625  -2.8625
##  [7,] 1.60 57.6 22.50000 64.0000  -6.4000
##  [8,]   NA 56.3       NA      NA       NA
##  [9,] 1.55 72.4 30.13528 60.0625  12.3375
## [10,] 1.70 62.1 21.48789 72.2500 -10.1500
## [11,] 1.63 52.6 19.79751 66.4225 -13.8225
## [12,] 1.70 82.1 28.40830 72.2500   9.8500
## [13,] 1.75 81.9 26.74286 76.5625   5.3375
## 
## , , 9
## 
##       [,1] [,2]     [,3]    [,4]     [,5]
##  [1,] 1.74 63.8 21.07280 75.6900 -11.8900
##  [2,] 1.63 79.5 29.92209 66.4225  13.0775
##  [3,] 1.77 81.6 26.04616 78.3225   3.2775
##  [4,] 1.75 81.3 26.54694 76.5625   4.7375
##  [5,]   NA 49.0       NA      NA       NA
##  [6,] 1.85 82.7 24.16362 85.5625  -2.8625
##  [7,] 1.60 57.6 22.50000 64.0000  -6.4000
##  [8,]   NA 56.3       NA      NA       NA
##  [9,] 1.55 72.4 30.13528 60.0625  12.3375
## [10,] 1.70 62.1 21.48789 72.2500 -10.1500
## [11,] 1.63 52.6 19.79751 66.4225 -13.8225
## [12,] 1.70 82.1 28.40830 72.2500   9.8500
## [13,] 1.75 81.9 26.74286 76.5625   5.3375
## 
## , , 10
## 
##       [,1] [,2]     [,3]    [,4]     [,5]
##  [1,] 1.74 63.8 21.07280 75.6900 -11.8900
##  [2,] 1.63 79.5 29.92209 66.4225  13.0775
##  [3,] 1.77 81.6 26.04616 78.3225   3.2775
##  [4,] 1.75 81.3 26.54694 76.5625   4.7375
##  [5,]   NA 49.0       NA      NA       NA
##  [6,] 1.85 82.7 24.16362 85.5625  -2.8625
##  [7,] 1.60 57.6 22.50000 64.0000  -6.4000
##  [8,]   NA 56.3       NA      NA       NA
##  [9,] 1.55 72.4 30.13528 60.0625  12.3375
## [10,] 1.70 62.1 21.48789 72.2500 -10.1500
## [11,] 1.63 52.6 19.79751 66.4225 -13.8225
## [12,] 1.70 82.1 28.40830 72.2500   9.8500
## [13,] 1.75 81.9 26.74286 76.5625   5.3375
\end{verbatim}

\begin{Shaded}
\begin{Highlighting}[]
\CommentTok{#names(rs)[,,1:10]<-c(nomes,c("altura", "peso","IMC","peso_max","deltap"))}
\CommentTok{#names(rs)[-(1:13),,1:10]<-c("altura", "peso","IMC","peso_max","deltap")}
\CommentTok{#names(rs)<-c(nomes,c("altura", "peso","IMC","peso_max","deltap"), as.character.Date(dez_semanas)) # Não funcionou a contento}
\NormalTok{rs}
\end{Highlighting}
\end{Shaded}

\begin{verbatim}
## , , 1
## 
##       [,1] [,2]     [,3]    [,4]     [,5]
##  [1,] 1.74 63.8 21.07280 75.6900 -11.8900
##  [2,] 1.63 79.5 29.92209 66.4225  13.0775
##  [3,] 1.77 81.6 26.04616 78.3225   3.2775
##  [4,] 1.75 81.3 26.54694 76.5625   4.7375
##  [5,]   NA 49.0       NA      NA       NA
##  [6,] 1.85 82.7 24.16362 85.5625  -2.8625
##  [7,] 1.60 57.6 22.50000 64.0000  -6.4000
##  [8,]   NA 56.3       NA      NA       NA
##  [9,] 1.55 72.4 30.13528 60.0625  12.3375
## [10,] 1.70 62.1 21.48789 72.2500 -10.1500
## [11,] 1.63 52.6 19.79751 66.4225 -13.8225
## [12,] 1.70 82.1 28.40830 72.2500   9.8500
## [13,] 1.75 81.9 26.74286 76.5625   5.3375
## 
## , , 2
## 
##       [,1] [,2]     [,3]    [,4]     [,5]
##  [1,] 1.74 63.8 21.07280 75.6900 -11.8900
##  [2,] 1.63 79.5 29.92209 66.4225  13.0775
##  [3,] 1.77 81.6 26.04616 78.3225   3.2775
##  [4,] 1.75 81.3 26.54694 76.5625   4.7375
##  [5,]   NA 49.0       NA      NA       NA
##  [6,] 1.85 82.7 24.16362 85.5625  -2.8625
##  [7,] 1.60 57.6 22.50000 64.0000  -6.4000
##  [8,]   NA 56.3       NA      NA       NA
##  [9,] 1.55 72.4 30.13528 60.0625  12.3375
## [10,] 1.70 62.1 21.48789 72.2500 -10.1500
## [11,] 1.63 52.6 19.79751 66.4225 -13.8225
## [12,] 1.70 82.1 28.40830 72.2500   9.8500
## [13,] 1.75 81.9 26.74286 76.5625   5.3375
## 
## , , 3
## 
##       [,1] [,2]     [,3]    [,4]     [,5]
##  [1,] 1.74 63.8 21.07280 75.6900 -11.8900
##  [2,] 1.63 79.5 29.92209 66.4225  13.0775
##  [3,] 1.77 81.6 26.04616 78.3225   3.2775
##  [4,] 1.75 81.3 26.54694 76.5625   4.7375
##  [5,]   NA 49.0       NA      NA       NA
##  [6,] 1.85 82.7 24.16362 85.5625  -2.8625
##  [7,] 1.60 57.6 22.50000 64.0000  -6.4000
##  [8,]   NA 56.3       NA      NA       NA
##  [9,] 1.55 72.4 30.13528 60.0625  12.3375
## [10,] 1.70 62.1 21.48789 72.2500 -10.1500
## [11,] 1.63 52.6 19.79751 66.4225 -13.8225
## [12,] 1.70 82.1 28.40830 72.2500   9.8500
## [13,] 1.75 81.9 26.74286 76.5625   5.3375
## 
## , , 4
## 
##       [,1] [,2]     [,3]    [,4]     [,5]
##  [1,] 1.74 63.8 21.07280 75.6900 -11.8900
##  [2,] 1.63 79.5 29.92209 66.4225  13.0775
##  [3,] 1.77 81.6 26.04616 78.3225   3.2775
##  [4,] 1.75 81.3 26.54694 76.5625   4.7375
##  [5,]   NA 49.0       NA      NA       NA
##  [6,] 1.85 82.7 24.16362 85.5625  -2.8625
##  [7,] 1.60 57.6 22.50000 64.0000  -6.4000
##  [8,]   NA 56.3       NA      NA       NA
##  [9,] 1.55 72.4 30.13528 60.0625  12.3375
## [10,] 1.70 62.1 21.48789 72.2500 -10.1500
## [11,] 1.63 52.6 19.79751 66.4225 -13.8225
## [12,] 1.70 82.1 28.40830 72.2500   9.8500
## [13,] 1.75 81.9 26.74286 76.5625   5.3375
## 
## , , 5
## 
##       [,1] [,2]     [,3]    [,4]     [,5]
##  [1,] 1.74 63.8 21.07280 75.6900 -11.8900
##  [2,] 1.63 79.5 29.92209 66.4225  13.0775
##  [3,] 1.77 81.6 26.04616 78.3225   3.2775
##  [4,] 1.75 81.3 26.54694 76.5625   4.7375
##  [5,]   NA 49.0       NA      NA       NA
##  [6,] 1.85 82.7 24.16362 85.5625  -2.8625
##  [7,] 1.60 57.6 22.50000 64.0000  -6.4000
##  [8,]   NA 56.3       NA      NA       NA
##  [9,] 1.55 72.4 30.13528 60.0625  12.3375
## [10,] 1.70 62.1 21.48789 72.2500 -10.1500
## [11,] 1.63 52.6 19.79751 66.4225 -13.8225
## [12,] 1.70 82.1 28.40830 72.2500   9.8500
## [13,] 1.75 81.9 26.74286 76.5625   5.3375
## 
## , , 6
## 
##       [,1] [,2]     [,3]    [,4]     [,5]
##  [1,] 1.74 63.8 21.07280 75.6900 -11.8900
##  [2,] 1.63 79.5 29.92209 66.4225  13.0775
##  [3,] 1.77 81.6 26.04616 78.3225   3.2775
##  [4,] 1.75 81.3 26.54694 76.5625   4.7375
##  [5,]   NA 49.0       NA      NA       NA
##  [6,] 1.85 82.7 24.16362 85.5625  -2.8625
##  [7,] 1.60 57.6 22.50000 64.0000  -6.4000
##  [8,]   NA 56.3       NA      NA       NA
##  [9,] 1.55 72.4 30.13528 60.0625  12.3375
## [10,] 1.70 62.1 21.48789 72.2500 -10.1500
## [11,] 1.63 52.6 19.79751 66.4225 -13.8225
## [12,] 1.70 82.1 28.40830 72.2500   9.8500
## [13,] 1.75 81.9 26.74286 76.5625   5.3375
## 
## , , 7
## 
##       [,1] [,2]     [,3]    [,4]     [,5]
##  [1,] 1.74 63.8 21.07280 75.6900 -11.8900
##  [2,] 1.63 79.5 29.92209 66.4225  13.0775
##  [3,] 1.77 81.6 26.04616 78.3225   3.2775
##  [4,] 1.75 81.3 26.54694 76.5625   4.7375
##  [5,]   NA 49.0       NA      NA       NA
##  [6,] 1.85 82.7 24.16362 85.5625  -2.8625
##  [7,] 1.60 57.6 22.50000 64.0000  -6.4000
##  [8,]   NA 56.3       NA      NA       NA
##  [9,] 1.55 72.4 30.13528 60.0625  12.3375
## [10,] 1.70 62.1 21.48789 72.2500 -10.1500
## [11,] 1.63 52.6 19.79751 66.4225 -13.8225
## [12,] 1.70 82.1 28.40830 72.2500   9.8500
## [13,] 1.75 81.9 26.74286 76.5625   5.3375
## 
## , , 8
## 
##       [,1] [,2]     [,3]    [,4]     [,5]
##  [1,] 1.74 63.8 21.07280 75.6900 -11.8900
##  [2,] 1.63 79.5 29.92209 66.4225  13.0775
##  [3,] 1.77 81.6 26.04616 78.3225   3.2775
##  [4,] 1.75 81.3 26.54694 76.5625   4.7375
##  [5,]   NA 49.0       NA      NA       NA
##  [6,] 1.85 82.7 24.16362 85.5625  -2.8625
##  [7,] 1.60 57.6 22.50000 64.0000  -6.4000
##  [8,]   NA 56.3       NA      NA       NA
##  [9,] 1.55 72.4 30.13528 60.0625  12.3375
## [10,] 1.70 62.1 21.48789 72.2500 -10.1500
## [11,] 1.63 52.6 19.79751 66.4225 -13.8225
## [12,] 1.70 82.1 28.40830 72.2500   9.8500
## [13,] 1.75 81.9 26.74286 76.5625   5.3375
## 
## , , 9
## 
##       [,1] [,2]     [,3]    [,4]     [,5]
##  [1,] 1.74 63.8 21.07280 75.6900 -11.8900
##  [2,] 1.63 79.5 29.92209 66.4225  13.0775
##  [3,] 1.77 81.6 26.04616 78.3225   3.2775
##  [4,] 1.75 81.3 26.54694 76.5625   4.7375
##  [5,]   NA 49.0       NA      NA       NA
##  [6,] 1.85 82.7 24.16362 85.5625  -2.8625
##  [7,] 1.60 57.6 22.50000 64.0000  -6.4000
##  [8,]   NA 56.3       NA      NA       NA
##  [9,] 1.55 72.4 30.13528 60.0625  12.3375
## [10,] 1.70 62.1 21.48789 72.2500 -10.1500
## [11,] 1.63 52.6 19.79751 66.4225 -13.8225
## [12,] 1.70 82.1 28.40830 72.2500   9.8500
## [13,] 1.75 81.9 26.74286 76.5625   5.3375
## 
## , , 10
## 
##       [,1] [,2]     [,3]    [,4]     [,5]
##  [1,] 1.74 63.8 21.07280 75.6900 -11.8900
##  [2,] 1.63 79.5 29.92209 66.4225  13.0775
##  [3,] 1.77 81.6 26.04616 78.3225   3.2775
##  [4,] 1.75 81.3 26.54694 76.5625   4.7375
##  [5,]   NA 49.0       NA      NA       NA
##  [6,] 1.85 82.7 24.16362 85.5625  -2.8625
##  [7,] 1.60 57.6 22.50000 64.0000  -6.4000
##  [8,]   NA 56.3       NA      NA       NA
##  [9,] 1.55 72.4 30.13528 60.0625  12.3375
## [10,] 1.70 62.1 21.48789 72.2500 -10.1500
## [11,] 1.63 52.6 19.79751 66.4225 -13.8225
## [12,] 1.70 82.1 28.40830 72.2500   9.8500
## [13,] 1.75 81.9 26.74286 76.5625   5.3375
\end{verbatim}

\begin{Shaded}
\begin{Highlighting}[]
\CommentTok{# Talvez dimnames(x) <- value possa funcionar; é preciso testar}
\KeywordTok{dim}\NormalTok{(rs)}
\end{Highlighting}
\end{Shaded}

\begin{verbatim}
## [1] 13  5 10
\end{verbatim}

\begin{Shaded}
\begin{Highlighting}[]
\KeywordTok{as.character.Date}\NormalTok{(dez_semanas)}
\end{Highlighting}
\end{Shaded}

\begin{verbatim}
##  [1] "2018-04-11" "2018-04-18" "2018-04-25" "2018-05-02" "2018-05-09"
##  [6] "2018-05-16" "2018-05-23" "2018-05-30" "2018-06-06" "2018-06-13"
\end{verbatim}

\begin{Shaded}
\begin{Highlighting}[]
\CommentTok{#dimnames(rs) <- c(nomes,c("altura", "peso","IMC","peso_max","deltap"), as.character.Date(dez_semanas))}
\CommentTok{# gerou o seguinte erro: "Error in dimnames(rs) <- c(nomes, c("altura", "peso", "IMC", "peso_max",  : 'dimnames' must be a list"}
\CommentTok{# Ou seja, melhot partir para o uso de data.frame e de list}
\end{Highlighting}
\end{Shaded}

\section{List}\label{list}

Generalização dos vetores no sentido de que uma lista é uma coleção de
objetos de tipos os mais variados.\\
São vetores formados por dataframes, ``matrizes que permitem que suas
colunas tenham diferentes tipos de variáveis etc.

\begin{Shaded}
\begin{Highlighting}[]
\NormalTok{dados<-}\KeywordTok{c}\NormalTok{(}\KeywordTok{rep}\NormalTok{(}\DecValTok{1}\OperatorTok{:}\DecValTok{4}\NormalTok{, }\DecValTok{2}\NormalTok{, }\DataTypeTok{each =} \DecValTok{2}\NormalTok{))}
\NormalTok{A =}\StringTok{ }\KeywordTok{list}\NormalTok{(}\DataTypeTok{x =} \DecValTok{1}\OperatorTok{:}\DecValTok{4}\NormalTok{, }\DataTypeTok{y =} \KeywordTok{matrix}\NormalTok{(}\DecValTok{1}\OperatorTok{:}\DecValTok{4}\NormalTok{,}\DecValTok{2}\NormalTok{,}\DecValTok{2}\NormalTok{), }\DataTypeTok{w =}\NormalTok{ dados, }\DataTypeTok{v =} \KeywordTok{list}\NormalTok{(}\DataTypeTok{B=}\DecValTok{4}\NormalTok{,}\DataTypeTok{C=}\DecValTok{5}\NormalTok{))}

\NormalTok{A}
\end{Highlighting}
\end{Shaded}

\begin{verbatim}
## $x
## [1] 1 2 3 4
## 
## $y
##      [,1] [,2]
## [1,]    1    3
## [2,]    2    4
## 
## $w
##  [1] 1 1 2 2 3 3 4 4 1 1 2 2 3 3 4 4
## 
## $v
## $v$B
## [1] 4
## 
## $v$C
## [1] 5
\end{verbatim}

\begin{Shaded}
\begin{Highlighting}[]
\NormalTok{A[[}\DecValTok{1}\NormalTok{]]}
\end{Highlighting}
\end{Shaded}

\begin{verbatim}
## [1] 1 2 3 4
\end{verbatim}

\begin{Shaded}
\begin{Highlighting}[]
\NormalTok{A[[}\DecValTok{4}\NormalTok{]]}
\end{Highlighting}
\end{Shaded}

\begin{verbatim}
## $B
## [1] 4
## 
## $C
## [1] 5
\end{verbatim}

\begin{Shaded}
\begin{Highlighting}[]
\NormalTok{A}\OperatorTok{$}\NormalTok{x}
\end{Highlighting}
\end{Shaded}

\begin{verbatim}
## [1] 1 2 3 4
\end{verbatim}

\begin{Shaded}
\begin{Highlighting}[]
\NormalTok{A}\OperatorTok{$}\NormalTok{y}
\end{Highlighting}
\end{Shaded}

\begin{verbatim}
##      [,1] [,2]
## [1,]    1    3
## [2,]    2    4
\end{verbatim}

\begin{Shaded}
\begin{Highlighting}[]
\NormalTok{A}\OperatorTok{$}\NormalTok{w}
\end{Highlighting}
\end{Shaded}

\begin{verbatim}
##  [1] 1 1 2 2 3 3 4 4 1 1 2 2 3 3 4 4
\end{verbatim}

\begin{Shaded}
\begin{Highlighting}[]
\NormalTok{A}\OperatorTok{$}\NormalTok{v}
\end{Highlighting}
\end{Shaded}

\begin{verbatim}
## $B
## [1] 4
## 
## $C
## [1] 5
\end{verbatim}

\begin{Shaded}
\begin{Highlighting}[]
\NormalTok{B =}\StringTok{ }\KeywordTok{list}\NormalTok{(}\DataTypeTok{s =} \DecValTok{1}\OperatorTok{:}\DecValTok{5}\NormalTok{, }\DataTypeTok{r =} \DecValTok{2}\NormalTok{)}

\NormalTok{Q =}\StringTok{ }\KeywordTok{c}\NormalTok{(A,B)}

\NormalTok{Q}
\end{Highlighting}
\end{Shaded}

\begin{verbatim}
## $x
## [1] 1 2 3 4
## 
## $y
##      [,1] [,2]
## [1,]    1    3
## [2,]    2    4
## 
## $w
##  [1] 1 1 2 2 3 3 4 4 1 1 2 2 3 3 4 4
## 
## $v
## $v$B
## [1] 4
## 
## $v$C
## [1] 5
## 
## 
## $s
## [1] 1 2 3 4 5
## 
## $r
## [1] 2
\end{verbatim}

\section{data.frame}\label{data.frame}

Generalização dos vetores no sentido de que uma data.frame é uma coleção
de objetos de tipos os mais variados, mas todos do mesmo tamanho.\\
São vetores formados por dataframes, matrizes que permitem que suas
colunas tenham diferentes tipos de variáveis etc.\\
Usados para guardar tabelas de dados de um problema qualquer.\\
Suas colunas tem nomes e podem conter dados de tipos diferentes,
diferindo de uma matriz.\\
Cada registro da BD corresponde a uma linha da data.frame e cada coluna
a uma variável variável, campo ou propriedade das observações
coletadas.\\
Podem ser criadas pela função data.table()

\begin{Shaded}
\begin{Highlighting}[]
\KeywordTok{data}\NormalTok{(iris)}

\NormalTok{iris}
\end{Highlighting}
\end{Shaded}

\begin{verbatim}
##     Sepal.Length Sepal.Width Petal.Length Petal.Width    Species
## 1            5.1         3.5          1.4         0.2     setosa
## 2            4.9         3.0          1.4         0.2     setosa
## 3            4.7         3.2          1.3         0.2     setosa
## 4            4.6         3.1          1.5         0.2     setosa
## 5            5.0         3.6          1.4         0.2     setosa
## 6            5.4         3.9          1.7         0.4     setosa
## 7            4.6         3.4          1.4         0.3     setosa
## 8            5.0         3.4          1.5         0.2     setosa
## 9            4.4         2.9          1.4         0.2     setosa
## 10           4.9         3.1          1.5         0.1     setosa
## 11           5.4         3.7          1.5         0.2     setosa
## 12           4.8         3.4          1.6         0.2     setosa
## 13           4.8         3.0          1.4         0.1     setosa
## 14           4.3         3.0          1.1         0.1     setosa
## 15           5.8         4.0          1.2         0.2     setosa
## 16           5.7         4.4          1.5         0.4     setosa
## 17           5.4         3.9          1.3         0.4     setosa
## 18           5.1         3.5          1.4         0.3     setosa
## 19           5.7         3.8          1.7         0.3     setosa
## 20           5.1         3.8          1.5         0.3     setosa
## 21           5.4         3.4          1.7         0.2     setosa
## 22           5.1         3.7          1.5         0.4     setosa
## 23           4.6         3.6          1.0         0.2     setosa
## 24           5.1         3.3          1.7         0.5     setosa
## 25           4.8         3.4          1.9         0.2     setosa
## 26           5.0         3.0          1.6         0.2     setosa
## 27           5.0         3.4          1.6         0.4     setosa
## 28           5.2         3.5          1.5         0.2     setosa
## 29           5.2         3.4          1.4         0.2     setosa
## 30           4.7         3.2          1.6         0.2     setosa
## 31           4.8         3.1          1.6         0.2     setosa
## 32           5.4         3.4          1.5         0.4     setosa
## 33           5.2         4.1          1.5         0.1     setosa
## 34           5.5         4.2          1.4         0.2     setosa
## 35           4.9         3.1          1.5         0.2     setosa
## 36           5.0         3.2          1.2         0.2     setosa
## 37           5.5         3.5          1.3         0.2     setosa
## 38           4.9         3.6          1.4         0.1     setosa
## 39           4.4         3.0          1.3         0.2     setosa
## 40           5.1         3.4          1.5         0.2     setosa
## 41           5.0         3.5          1.3         0.3     setosa
## 42           4.5         2.3          1.3         0.3     setosa
## 43           4.4         3.2          1.3         0.2     setosa
## 44           5.0         3.5          1.6         0.6     setosa
## 45           5.1         3.8          1.9         0.4     setosa
## 46           4.8         3.0          1.4         0.3     setosa
## 47           5.1         3.8          1.6         0.2     setosa
## 48           4.6         3.2          1.4         0.2     setosa
## 49           5.3         3.7          1.5         0.2     setosa
## 50           5.0         3.3          1.4         0.2     setosa
## 51           7.0         3.2          4.7         1.4 versicolor
## 52           6.4         3.2          4.5         1.5 versicolor
## 53           6.9         3.1          4.9         1.5 versicolor
## 54           5.5         2.3          4.0         1.3 versicolor
## 55           6.5         2.8          4.6         1.5 versicolor
## 56           5.7         2.8          4.5         1.3 versicolor
## 57           6.3         3.3          4.7         1.6 versicolor
## 58           4.9         2.4          3.3         1.0 versicolor
## 59           6.6         2.9          4.6         1.3 versicolor
## 60           5.2         2.7          3.9         1.4 versicolor
## 61           5.0         2.0          3.5         1.0 versicolor
## 62           5.9         3.0          4.2         1.5 versicolor
## 63           6.0         2.2          4.0         1.0 versicolor
## 64           6.1         2.9          4.7         1.4 versicolor
## 65           5.6         2.9          3.6         1.3 versicolor
## 66           6.7         3.1          4.4         1.4 versicolor
## 67           5.6         3.0          4.5         1.5 versicolor
## 68           5.8         2.7          4.1         1.0 versicolor
## 69           6.2         2.2          4.5         1.5 versicolor
## 70           5.6         2.5          3.9         1.1 versicolor
## 71           5.9         3.2          4.8         1.8 versicolor
## 72           6.1         2.8          4.0         1.3 versicolor
## 73           6.3         2.5          4.9         1.5 versicolor
## 74           6.1         2.8          4.7         1.2 versicolor
## 75           6.4         2.9          4.3         1.3 versicolor
## 76           6.6         3.0          4.4         1.4 versicolor
## 77           6.8         2.8          4.8         1.4 versicolor
## 78           6.7         3.0          5.0         1.7 versicolor
## 79           6.0         2.9          4.5         1.5 versicolor
## 80           5.7         2.6          3.5         1.0 versicolor
## 81           5.5         2.4          3.8         1.1 versicolor
## 82           5.5         2.4          3.7         1.0 versicolor
## 83           5.8         2.7          3.9         1.2 versicolor
## 84           6.0         2.7          5.1         1.6 versicolor
## 85           5.4         3.0          4.5         1.5 versicolor
## 86           6.0         3.4          4.5         1.6 versicolor
## 87           6.7         3.1          4.7         1.5 versicolor
## 88           6.3         2.3          4.4         1.3 versicolor
## 89           5.6         3.0          4.1         1.3 versicolor
## 90           5.5         2.5          4.0         1.3 versicolor
## 91           5.5         2.6          4.4         1.2 versicolor
## 92           6.1         3.0          4.6         1.4 versicolor
## 93           5.8         2.6          4.0         1.2 versicolor
## 94           5.0         2.3          3.3         1.0 versicolor
## 95           5.6         2.7          4.2         1.3 versicolor
## 96           5.7         3.0          4.2         1.2 versicolor
## 97           5.7         2.9          4.2         1.3 versicolor
## 98           6.2         2.9          4.3         1.3 versicolor
## 99           5.1         2.5          3.0         1.1 versicolor
## 100          5.7         2.8          4.1         1.3 versicolor
## 101          6.3         3.3          6.0         2.5  virginica
## 102          5.8         2.7          5.1         1.9  virginica
## 103          7.1         3.0          5.9         2.1  virginica
## 104          6.3         2.9          5.6         1.8  virginica
## 105          6.5         3.0          5.8         2.2  virginica
## 106          7.6         3.0          6.6         2.1  virginica
## 107          4.9         2.5          4.5         1.7  virginica
## 108          7.3         2.9          6.3         1.8  virginica
## 109          6.7         2.5          5.8         1.8  virginica
## 110          7.2         3.6          6.1         2.5  virginica
## 111          6.5         3.2          5.1         2.0  virginica
## 112          6.4         2.7          5.3         1.9  virginica
## 113          6.8         3.0          5.5         2.1  virginica
## 114          5.7         2.5          5.0         2.0  virginica
## 115          5.8         2.8          5.1         2.4  virginica
## 116          6.4         3.2          5.3         2.3  virginica
## 117          6.5         3.0          5.5         1.8  virginica
## 118          7.7         3.8          6.7         2.2  virginica
## 119          7.7         2.6          6.9         2.3  virginica
## 120          6.0         2.2          5.0         1.5  virginica
## 121          6.9         3.2          5.7         2.3  virginica
## 122          5.6         2.8          4.9         2.0  virginica
## 123          7.7         2.8          6.7         2.0  virginica
## 124          6.3         2.7          4.9         1.8  virginica
## 125          6.7         3.3          5.7         2.1  virginica
## 126          7.2         3.2          6.0         1.8  virginica
## 127          6.2         2.8          4.8         1.8  virginica
## 128          6.1         3.0          4.9         1.8  virginica
## 129          6.4         2.8          5.6         2.1  virginica
## 130          7.2         3.0          5.8         1.6  virginica
## 131          7.4         2.8          6.1         1.9  virginica
## 132          7.9         3.8          6.4         2.0  virginica
## 133          6.4         2.8          5.6         2.2  virginica
## 134          6.3         2.8          5.1         1.5  virginica
## 135          6.1         2.6          5.6         1.4  virginica
## 136          7.7         3.0          6.1         2.3  virginica
## 137          6.3         3.4          5.6         2.4  virginica
## 138          6.4         3.1          5.5         1.8  virginica
## 139          6.0         3.0          4.8         1.8  virginica
## 140          6.9         3.1          5.4         2.1  virginica
## 141          6.7         3.1          5.6         2.4  virginica
## 142          6.9         3.1          5.1         2.3  virginica
## 143          5.8         2.7          5.1         1.9  virginica
## 144          6.8         3.2          5.9         2.3  virginica
## 145          6.7         3.3          5.7         2.5  virginica
## 146          6.7         3.0          5.2         2.3  virginica
## 147          6.3         2.5          5.0         1.9  virginica
## 148          6.5         3.0          5.2         2.0  virginica
## 149          6.2         3.4          5.4         2.3  virginica
## 150          5.9         3.0          5.1         1.8  virginica
\end{verbatim}

\begin{Shaded}
\begin{Highlighting}[]
\NormalTok{iris}\OperatorTok{$}\NormalTok{Sepal.Length}
\end{Highlighting}
\end{Shaded}

\begin{verbatim}
##   [1] 5.1 4.9 4.7 4.6 5.0 5.4 4.6 5.0 4.4 4.9 5.4 4.8 4.8 4.3 5.8 5.7 5.4
##  [18] 5.1 5.7 5.1 5.4 5.1 4.6 5.1 4.8 5.0 5.0 5.2 5.2 4.7 4.8 5.4 5.2 5.5
##  [35] 4.9 5.0 5.5 4.9 4.4 5.1 5.0 4.5 4.4 5.0 5.1 4.8 5.1 4.6 5.3 5.0 7.0
##  [52] 6.4 6.9 5.5 6.5 5.7 6.3 4.9 6.6 5.2 5.0 5.9 6.0 6.1 5.6 6.7 5.6 5.8
##  [69] 6.2 5.6 5.9 6.1 6.3 6.1 6.4 6.6 6.8 6.7 6.0 5.7 5.5 5.5 5.8 6.0 5.4
##  [86] 6.0 6.7 6.3 5.6 5.5 5.5 6.1 5.8 5.0 5.6 5.7 5.7 6.2 5.1 5.7 6.3 5.8
## [103] 7.1 6.3 6.5 7.6 4.9 7.3 6.7 7.2 6.5 6.4 6.8 5.7 5.8 6.4 6.5 7.7 7.7
## [120] 6.0 6.9 5.6 7.7 6.3 6.7 7.2 6.2 6.1 6.4 7.2 7.4 7.9 6.4 6.3 6.1 7.7
## [137] 6.3 6.4 6.0 6.9 6.7 6.9 5.8 6.8 6.7 6.7 6.3 6.5 6.2 5.9
\end{verbatim}

\begin{Shaded}
\begin{Highlighting}[]
\NormalTok{iris}\OperatorTok{$}\NormalTok{Renato =}\StringTok{ }\OtherTok{TRUE}

\NormalTok{iris}
\end{Highlighting}
\end{Shaded}

\begin{verbatim}
##     Sepal.Length Sepal.Width Petal.Length Petal.Width    Species Renato
## 1            5.1         3.5          1.4         0.2     setosa   TRUE
## 2            4.9         3.0          1.4         0.2     setosa   TRUE
## 3            4.7         3.2          1.3         0.2     setosa   TRUE
## 4            4.6         3.1          1.5         0.2     setosa   TRUE
## 5            5.0         3.6          1.4         0.2     setosa   TRUE
## 6            5.4         3.9          1.7         0.4     setosa   TRUE
## 7            4.6         3.4          1.4         0.3     setosa   TRUE
## 8            5.0         3.4          1.5         0.2     setosa   TRUE
## 9            4.4         2.9          1.4         0.2     setosa   TRUE
## 10           4.9         3.1          1.5         0.1     setosa   TRUE
## 11           5.4         3.7          1.5         0.2     setosa   TRUE
## 12           4.8         3.4          1.6         0.2     setosa   TRUE
## 13           4.8         3.0          1.4         0.1     setosa   TRUE
## 14           4.3         3.0          1.1         0.1     setosa   TRUE
## 15           5.8         4.0          1.2         0.2     setosa   TRUE
## 16           5.7         4.4          1.5         0.4     setosa   TRUE
## 17           5.4         3.9          1.3         0.4     setosa   TRUE
## 18           5.1         3.5          1.4         0.3     setosa   TRUE
## 19           5.7         3.8          1.7         0.3     setosa   TRUE
## 20           5.1         3.8          1.5         0.3     setosa   TRUE
## 21           5.4         3.4          1.7         0.2     setosa   TRUE
## 22           5.1         3.7          1.5         0.4     setosa   TRUE
## 23           4.6         3.6          1.0         0.2     setosa   TRUE
## 24           5.1         3.3          1.7         0.5     setosa   TRUE
## 25           4.8         3.4          1.9         0.2     setosa   TRUE
## 26           5.0         3.0          1.6         0.2     setosa   TRUE
## 27           5.0         3.4          1.6         0.4     setosa   TRUE
## 28           5.2         3.5          1.5         0.2     setosa   TRUE
## 29           5.2         3.4          1.4         0.2     setosa   TRUE
## 30           4.7         3.2          1.6         0.2     setosa   TRUE
## 31           4.8         3.1          1.6         0.2     setosa   TRUE
## 32           5.4         3.4          1.5         0.4     setosa   TRUE
## 33           5.2         4.1          1.5         0.1     setosa   TRUE
## 34           5.5         4.2          1.4         0.2     setosa   TRUE
## 35           4.9         3.1          1.5         0.2     setosa   TRUE
## 36           5.0         3.2          1.2         0.2     setosa   TRUE
## 37           5.5         3.5          1.3         0.2     setosa   TRUE
## 38           4.9         3.6          1.4         0.1     setosa   TRUE
## 39           4.4         3.0          1.3         0.2     setosa   TRUE
## 40           5.1         3.4          1.5         0.2     setosa   TRUE
## 41           5.0         3.5          1.3         0.3     setosa   TRUE
## 42           4.5         2.3          1.3         0.3     setosa   TRUE
## 43           4.4         3.2          1.3         0.2     setosa   TRUE
## 44           5.0         3.5          1.6         0.6     setosa   TRUE
## 45           5.1         3.8          1.9         0.4     setosa   TRUE
## 46           4.8         3.0          1.4         0.3     setosa   TRUE
## 47           5.1         3.8          1.6         0.2     setosa   TRUE
## 48           4.6         3.2          1.4         0.2     setosa   TRUE
## 49           5.3         3.7          1.5         0.2     setosa   TRUE
## 50           5.0         3.3          1.4         0.2     setosa   TRUE
## 51           7.0         3.2          4.7         1.4 versicolor   TRUE
## 52           6.4         3.2          4.5         1.5 versicolor   TRUE
## 53           6.9         3.1          4.9         1.5 versicolor   TRUE
## 54           5.5         2.3          4.0         1.3 versicolor   TRUE
## 55           6.5         2.8          4.6         1.5 versicolor   TRUE
## 56           5.7         2.8          4.5         1.3 versicolor   TRUE
## 57           6.3         3.3          4.7         1.6 versicolor   TRUE
## 58           4.9         2.4          3.3         1.0 versicolor   TRUE
## 59           6.6         2.9          4.6         1.3 versicolor   TRUE
## 60           5.2         2.7          3.9         1.4 versicolor   TRUE
## 61           5.0         2.0          3.5         1.0 versicolor   TRUE
## 62           5.9         3.0          4.2         1.5 versicolor   TRUE
## 63           6.0         2.2          4.0         1.0 versicolor   TRUE
## 64           6.1         2.9          4.7         1.4 versicolor   TRUE
## 65           5.6         2.9          3.6         1.3 versicolor   TRUE
## 66           6.7         3.1          4.4         1.4 versicolor   TRUE
## 67           5.6         3.0          4.5         1.5 versicolor   TRUE
## 68           5.8         2.7          4.1         1.0 versicolor   TRUE
## 69           6.2         2.2          4.5         1.5 versicolor   TRUE
## 70           5.6         2.5          3.9         1.1 versicolor   TRUE
## 71           5.9         3.2          4.8         1.8 versicolor   TRUE
## 72           6.1         2.8          4.0         1.3 versicolor   TRUE
## 73           6.3         2.5          4.9         1.5 versicolor   TRUE
## 74           6.1         2.8          4.7         1.2 versicolor   TRUE
## 75           6.4         2.9          4.3         1.3 versicolor   TRUE
## 76           6.6         3.0          4.4         1.4 versicolor   TRUE
## 77           6.8         2.8          4.8         1.4 versicolor   TRUE
## 78           6.7         3.0          5.0         1.7 versicolor   TRUE
## 79           6.0         2.9          4.5         1.5 versicolor   TRUE
## 80           5.7         2.6          3.5         1.0 versicolor   TRUE
## 81           5.5         2.4          3.8         1.1 versicolor   TRUE
## 82           5.5         2.4          3.7         1.0 versicolor   TRUE
## 83           5.8         2.7          3.9         1.2 versicolor   TRUE
## 84           6.0         2.7          5.1         1.6 versicolor   TRUE
## 85           5.4         3.0          4.5         1.5 versicolor   TRUE
## 86           6.0         3.4          4.5         1.6 versicolor   TRUE
## 87           6.7         3.1          4.7         1.5 versicolor   TRUE
## 88           6.3         2.3          4.4         1.3 versicolor   TRUE
## 89           5.6         3.0          4.1         1.3 versicolor   TRUE
## 90           5.5         2.5          4.0         1.3 versicolor   TRUE
## 91           5.5         2.6          4.4         1.2 versicolor   TRUE
## 92           6.1         3.0          4.6         1.4 versicolor   TRUE
## 93           5.8         2.6          4.0         1.2 versicolor   TRUE
## 94           5.0         2.3          3.3         1.0 versicolor   TRUE
## 95           5.6         2.7          4.2         1.3 versicolor   TRUE
## 96           5.7         3.0          4.2         1.2 versicolor   TRUE
## 97           5.7         2.9          4.2         1.3 versicolor   TRUE
## 98           6.2         2.9          4.3         1.3 versicolor   TRUE
## 99           5.1         2.5          3.0         1.1 versicolor   TRUE
## 100          5.7         2.8          4.1         1.3 versicolor   TRUE
## 101          6.3         3.3          6.0         2.5  virginica   TRUE
## 102          5.8         2.7          5.1         1.9  virginica   TRUE
## 103          7.1         3.0          5.9         2.1  virginica   TRUE
## 104          6.3         2.9          5.6         1.8  virginica   TRUE
## 105          6.5         3.0          5.8         2.2  virginica   TRUE
## 106          7.6         3.0          6.6         2.1  virginica   TRUE
## 107          4.9         2.5          4.5         1.7  virginica   TRUE
## 108          7.3         2.9          6.3         1.8  virginica   TRUE
## 109          6.7         2.5          5.8         1.8  virginica   TRUE
## 110          7.2         3.6          6.1         2.5  virginica   TRUE
## 111          6.5         3.2          5.1         2.0  virginica   TRUE
## 112          6.4         2.7          5.3         1.9  virginica   TRUE
## 113          6.8         3.0          5.5         2.1  virginica   TRUE
## 114          5.7         2.5          5.0         2.0  virginica   TRUE
## 115          5.8         2.8          5.1         2.4  virginica   TRUE
## 116          6.4         3.2          5.3         2.3  virginica   TRUE
## 117          6.5         3.0          5.5         1.8  virginica   TRUE
## 118          7.7         3.8          6.7         2.2  virginica   TRUE
## 119          7.7         2.6          6.9         2.3  virginica   TRUE
## 120          6.0         2.2          5.0         1.5  virginica   TRUE
## 121          6.9         3.2          5.7         2.3  virginica   TRUE
## 122          5.6         2.8          4.9         2.0  virginica   TRUE
## 123          7.7         2.8          6.7         2.0  virginica   TRUE
## 124          6.3         2.7          4.9         1.8  virginica   TRUE
## 125          6.7         3.3          5.7         2.1  virginica   TRUE
## 126          7.2         3.2          6.0         1.8  virginica   TRUE
## 127          6.2         2.8          4.8         1.8  virginica   TRUE
## 128          6.1         3.0          4.9         1.8  virginica   TRUE
## 129          6.4         2.8          5.6         2.1  virginica   TRUE
## 130          7.2         3.0          5.8         1.6  virginica   TRUE
## 131          7.4         2.8          6.1         1.9  virginica   TRUE
## 132          7.9         3.8          6.4         2.0  virginica   TRUE
## 133          6.4         2.8          5.6         2.2  virginica   TRUE
## 134          6.3         2.8          5.1         1.5  virginica   TRUE
## 135          6.1         2.6          5.6         1.4  virginica   TRUE
## 136          7.7         3.0          6.1         2.3  virginica   TRUE
## 137          6.3         3.4          5.6         2.4  virginica   TRUE
## 138          6.4         3.1          5.5         1.8  virginica   TRUE
## 139          6.0         3.0          4.8         1.8  virginica   TRUE
## 140          6.9         3.1          5.4         2.1  virginica   TRUE
## 141          6.7         3.1          5.6         2.4  virginica   TRUE
## 142          6.9         3.1          5.1         2.3  virginica   TRUE
## 143          5.8         2.7          5.1         1.9  virginica   TRUE
## 144          6.8         3.2          5.9         2.3  virginica   TRUE
## 145          6.7         3.3          5.7         2.5  virginica   TRUE
## 146          6.7         3.0          5.2         2.3  virginica   TRUE
## 147          6.3         2.5          5.0         1.9  virginica   TRUE
## 148          6.5         3.0          5.2         2.0  virginica   TRUE
## 149          6.2         3.4          5.4         2.3  virginica   TRUE
## 150          5.9         3.0          5.1         1.8  virginica   TRUE
\end{verbatim}

\begin{Shaded}
\begin{Highlighting}[]
\NormalTok{rs<-}\KeywordTok{data.table}\NormalTok{(nomes,h,p,IMC,pmax,deltap)}
\NormalTok{rs[,}\OperatorTok{-}\DecValTok{6}\NormalTok{]}
\end{Highlighting}
\end{Shaded}

\begin{verbatim}
##       nomes    h    p      IMC    pmax
##  1: Bernard 1.74 63.8 21.07280 75.6900
##  2:  Carlos 1.63 79.5 29.92209 66.4225
##  3: Cleuler 1.77 81.6 26.04616 78.3225
##  4:  Helber 1.75 81.3 26.54694 76.5625
##  5: Larissa   NA 49.0       NA      NA
##  6:  Mateus 1.85 82.7 24.16362 85.5625
##  7: Michell 1.60 57.6 22.50000 64.0000
##  8:  Nayana   NA 56.3       NA      NA
##  9:   Paula 1.55 72.4 30.13528 60.0625
## 10:  Rafael 1.70 62.1 21.48789 72.2500
## 11: Tatiane 1.63 52.6 19.79751 66.4225
## 12:  Thiago 1.70 82.1 28.40830 72.2500
## 13:  Wesley 1.75 81.9 26.74286 76.5625
\end{verbatim}

\begin{Shaded}
\begin{Highlighting}[]
\NormalTok{rs[,}\OperatorTok{-}\DecValTok{1}\NormalTok{]}
\end{Highlighting}
\end{Shaded}

\begin{verbatim}
##        h    p      IMC    pmax   deltap
##  1: 1.74 63.8 21.07280 75.6900 -11.8900
##  2: 1.63 79.5 29.92209 66.4225  13.0775
##  3: 1.77 81.6 26.04616 78.3225   3.2775
##  4: 1.75 81.3 26.54694 76.5625   4.7375
##  5:   NA 49.0       NA      NA       NA
##  6: 1.85 82.7 24.16362 85.5625  -2.8625
##  7: 1.60 57.6 22.50000 64.0000  -6.4000
##  8:   NA 56.3       NA      NA       NA
##  9: 1.55 72.4 30.13528 60.0625  12.3375
## 10: 1.70 62.1 21.48789 72.2500 -10.1500
## 11: 1.63 52.6 19.79751 66.4225 -13.8225
## 12: 1.70 82.1 28.40830 72.2500   9.8500
## 13: 1.75 81.9 26.74286 76.5625   5.3375
\end{verbatim}


\end{document}
