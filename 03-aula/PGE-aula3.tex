\documentclass[]{article}
\usepackage{lmodern}
\usepackage{amssymb,amsmath}
\usepackage{ifxetex,ifluatex}
\usepackage{fixltx2e} % provides \textsubscript
\ifnum 0\ifxetex 1\fi\ifluatex 1\fi=0 % if pdftex
  \usepackage[T1]{fontenc}
  \usepackage[utf8]{inputenc}
\else % if luatex or xelatex
  \ifxetex
    \usepackage{mathspec}
  \else
    \usepackage{fontspec}
  \fi
  \defaultfontfeatures{Ligatures=TeX,Scale=MatchLowercase}
\fi
% use upquote if available, for straight quotes in verbatim environments
\IfFileExists{upquote.sty}{\usepackage{upquote}}{}
% use microtype if available
\IfFileExists{microtype.sty}{%
\usepackage{microtype}
\UseMicrotypeSet[protrusion]{basicmath} % disable protrusion for tt fonts
}{}
\usepackage[margin=1in]{geometry}
\usepackage{hyperref}
\hypersetup{unicode=true,
            pdftitle={Noções Básicas de R - Aula 3},
            pdfborder={0 0 0},
            breaklinks=true}
\urlstyle{same}  % don't use monospace font for urls
\usepackage{color}
\usepackage{fancyvrb}
\newcommand{\VerbBar}{|}
\newcommand{\VERB}{\Verb[commandchars=\\\{\}]}
\DefineVerbatimEnvironment{Highlighting}{Verbatim}{commandchars=\\\{\}}
% Add ',fontsize=\small' for more characters per line
\usepackage{framed}
\definecolor{shadecolor}{RGB}{248,248,248}
\newenvironment{Shaded}{\begin{snugshade}}{\end{snugshade}}
\newcommand{\KeywordTok}[1]{\textcolor[rgb]{0.13,0.29,0.53}{\textbf{#1}}}
\newcommand{\DataTypeTok}[1]{\textcolor[rgb]{0.13,0.29,0.53}{#1}}
\newcommand{\DecValTok}[1]{\textcolor[rgb]{0.00,0.00,0.81}{#1}}
\newcommand{\BaseNTok}[1]{\textcolor[rgb]{0.00,0.00,0.81}{#1}}
\newcommand{\FloatTok}[1]{\textcolor[rgb]{0.00,0.00,0.81}{#1}}
\newcommand{\ConstantTok}[1]{\textcolor[rgb]{0.00,0.00,0.00}{#1}}
\newcommand{\CharTok}[1]{\textcolor[rgb]{0.31,0.60,0.02}{#1}}
\newcommand{\SpecialCharTok}[1]{\textcolor[rgb]{0.00,0.00,0.00}{#1}}
\newcommand{\StringTok}[1]{\textcolor[rgb]{0.31,0.60,0.02}{#1}}
\newcommand{\VerbatimStringTok}[1]{\textcolor[rgb]{0.31,0.60,0.02}{#1}}
\newcommand{\SpecialStringTok}[1]{\textcolor[rgb]{0.31,0.60,0.02}{#1}}
\newcommand{\ImportTok}[1]{#1}
\newcommand{\CommentTok}[1]{\textcolor[rgb]{0.56,0.35,0.01}{\textit{#1}}}
\newcommand{\DocumentationTok}[1]{\textcolor[rgb]{0.56,0.35,0.01}{\textbf{\textit{#1}}}}
\newcommand{\AnnotationTok}[1]{\textcolor[rgb]{0.56,0.35,0.01}{\textbf{\textit{#1}}}}
\newcommand{\CommentVarTok}[1]{\textcolor[rgb]{0.56,0.35,0.01}{\textbf{\textit{#1}}}}
\newcommand{\OtherTok}[1]{\textcolor[rgb]{0.56,0.35,0.01}{#1}}
\newcommand{\FunctionTok}[1]{\textcolor[rgb]{0.00,0.00,0.00}{#1}}
\newcommand{\VariableTok}[1]{\textcolor[rgb]{0.00,0.00,0.00}{#1}}
\newcommand{\ControlFlowTok}[1]{\textcolor[rgb]{0.13,0.29,0.53}{\textbf{#1}}}
\newcommand{\OperatorTok}[1]{\textcolor[rgb]{0.81,0.36,0.00}{\textbf{#1}}}
\newcommand{\BuiltInTok}[1]{#1}
\newcommand{\ExtensionTok}[1]{#1}
\newcommand{\PreprocessorTok}[1]{\textcolor[rgb]{0.56,0.35,0.01}{\textit{#1}}}
\newcommand{\AttributeTok}[1]{\textcolor[rgb]{0.77,0.63,0.00}{#1}}
\newcommand{\RegionMarkerTok}[1]{#1}
\newcommand{\InformationTok}[1]{\textcolor[rgb]{0.56,0.35,0.01}{\textbf{\textit{#1}}}}
\newcommand{\WarningTok}[1]{\textcolor[rgb]{0.56,0.35,0.01}{\textbf{\textit{#1}}}}
\newcommand{\AlertTok}[1]{\textcolor[rgb]{0.94,0.16,0.16}{#1}}
\newcommand{\ErrorTok}[1]{\textcolor[rgb]{0.64,0.00,0.00}{\textbf{#1}}}
\newcommand{\NormalTok}[1]{#1}
\usepackage{graphicx,grffile}
\makeatletter
\def\maxwidth{\ifdim\Gin@nat@width>\linewidth\linewidth\else\Gin@nat@width\fi}
\def\maxheight{\ifdim\Gin@nat@height>\textheight\textheight\else\Gin@nat@height\fi}
\makeatother
% Scale images if necessary, so that they will not overflow the page
% margins by default, and it is still possible to overwrite the defaults
% using explicit options in \includegraphics[width, height, ...]{}
\setkeys{Gin}{width=\maxwidth,height=\maxheight,keepaspectratio}
\IfFileExists{parskip.sty}{%
\usepackage{parskip}
}{% else
\setlength{\parindent}{0pt}
\setlength{\parskip}{6pt plus 2pt minus 1pt}
}
\setlength{\emergencystretch}{3em}  % prevent overfull lines
\providecommand{\tightlist}{%
  \setlength{\itemsep}{0pt}\setlength{\parskip}{0pt}}
\setcounter{secnumdepth}{0}
% Redefines (sub)paragraphs to behave more like sections
\ifx\paragraph\undefined\else
\let\oldparagraph\paragraph
\renewcommand{\paragraph}[1]{\oldparagraph{#1}\mbox{}}
\fi
\ifx\subparagraph\undefined\else
\let\oldsubparagraph\subparagraph
\renewcommand{\subparagraph}[1]{\oldsubparagraph{#1}\mbox{}}
\fi

%%% Use protect on footnotes to avoid problems with footnotes in titles
\let\rmarkdownfootnote\footnote%
\def\footnote{\protect\rmarkdownfootnote}

%%% Change title format to be more compact
\usepackage{titling}

% Create subtitle command for use in maketitle
\newcommand{\subtitle}[1]{
  \posttitle{
    \begin{center}\large#1\end{center}
    }
}

\setlength{\droptitle}{-2em}
  \title{Noções Básicas de R - Aula 3}
  \pretitle{\vspace{\droptitle}\centering\huge}
  \posttitle{\par}
  \author{}
  \preauthor{}\postauthor{}
  \date{}
  \predate{}\postdate{}


\begin{document}
\maketitle

\section{Exemplo de uso de R + Markdown +
knitr}\label{exemplo-de-uso-de-r-markdown-knitr}

Prof.~Dr.~Cleuler Barbosa das Neves\\
\href{http://buscatextual.cnpq.br/buscatextual/visualizacv.do?id=K4786159E2}{currículo.lattes}

AULA N. 03 - OBJETOS: VETORES, MATRIZES, DATA.FRAME, ARRAYS, LIST, DATE,
TS etc.

\section{\texorpdfstring{\textbf{R} é uma Linguagem \textbf{funcional}
orientada para
\textbf{objetos}!}{R é uma Linguagem funcional orientada para objetos!}}\label{r-e-uma-linguagem-funcional-orientada-para-objetos}

\section{{[}================================================{]}}\label{section}

\section{{[}Faz uso de funções \& de suas composições
!!!}\label{faz-uso-de-funcoes-de-suas-composicoes}

\section{{[}Armazena\&Manipula objetos previamente
criados!!!}\label{armazenamanipula-objetos-previamente-criados}

\section{\texorpdfstring{{[}\emph{Aply} essas composições nesses
\emph{ob-jectos}
!!!}{{[}Aply essas composições nesses ob-jectos !!!}}\label{aply-essas-composicoes-nesses-ob-jectos}

\section{\texorpdfstring{{[}Há \emph{symbols} c/significados
operacionais
\emph{tipics}!!!}{{[}Há symbols c/significados operacionais tipics!!!}}\label{ha-symbols-csignificados-operacionais-tipics}

\section{\texorpdfstring{{[}CRAN c/centenas de milhares de
\emph{functions} em
\emph{packages}!!!}{{[}CRAN c/centenas de milhares de functions em packages!!!}}\label{cran-ccentenas-de-milhares-de-functions-em-packages}

\section{{[}================================================{]}}\label{section-1}

\begin{Shaded}
\begin{Highlighting}[]
\CommentTok{# As "duas" primeiras linhas de comando de um script em R (p. 13) deve ser:}

\CommentTok{# A 1ª Linha de comando:}
\CommentTok{# O símbolo ~ representa a abreviatura para o caminho da pasta pessoal (Linux e Windows)}
\KeywordTok{setwd}\NormalTok{(}\StringTok{"~"}\NormalTok{) }\CommentTok{# Aponta o Diretório de Trabalho para a Pasta Pessoal e subpasta em que se encontra o arquivo deste script PGE-aula3.Rmd: "~/../Documents/R_CS/Aula3"}
\CommentTok{# Esse comando exibe a seguinte mensagem de alerta importante: "The working directory was changed to C:/Users/M/Documents inside a notebook chunk. The working directory will be reset when the chunk is finished running. Use the knitr root.dir option in the setup chunk to change the working directory for notebook chunks"}
\KeywordTok{getwd}\NormalTok{()    }\CommentTok{# Exibe  o Diretório de Trabalho, no caso o da Pasta Pessoal, executando uma linha de comando na janela Terminal da Console Area: "C:/Users/M/Documents/R_CS/Aula3"}
\end{Highlighting}
\end{Shaded}

\begin{verbatim}
## [1] "C:/Users/M/Documents"
\end{verbatim}

\begin{Shaded}
\begin{Highlighting}[]
\KeywordTok{setwd}\NormalTok{(}\StringTok{"~/../Documents/R_CS/Aula3"}\NormalTok{) }\CommentTok{# Produz o mesmo efeito do código anterior}
\KeywordTok{getwd}\NormalTok{()}
\end{Highlighting}
\end{Shaded}

\begin{verbatim}
## [1] "C:/Users/M/Documents/R_CS/Aula3"
\end{verbatim}

\begin{Shaded}
\begin{Highlighting}[]
\CommentTok{# A 2ª Linha de comando: é um exemplo do uso de **funções compostas** em Linguagem **R**}
\KeywordTok{rm}\NormalTok{(}\DataTypeTok{list=}\KeywordTok{ls}\NormalTok{()) }\CommentTok{# Remove toda a list de variáveis da Job Area, i. e., da Environment}

\CommentTok{#[=========================================================================]}
\CommentTok{#[                   Pacotes do System Library                             ]}
\CommentTok{#[=========================================================================]}

\CommentTok{#Pacotes de importação de BD}
\CommentTok{#para ativar um pacote do System Library (vem c/a instalação do R): 2.000 f's}
\KeywordTok{library}\NormalTok{(foreign) }\CommentTok{# argumento não precisa das aspas}
\CommentTok{# Para carregar Base de Dados dos aplicativos:}
\CommentTok{# Minitab, S, SAS, SPSS, Stata, Systat, Weka, dBase ...}

\CommentTok{#[=========================================================================]}
\CommentTok{#[                    Pacotes da User Library                              ]}
\CommentTok{#[=========================================================================]}

\CommentTok{#P/instalar um pacote da web (CRAN) basta executar install.packages() 1 vez}
\CommentTok{#install.packages("data.table") # Para carregar BD's de grandes dimensões}
\KeywordTok{library}\NormalTok{(data.table) }\CommentTok{# (p.53-53 do livro R_CS); argumento não precisa das aspas}
\end{Highlighting}
\end{Shaded}

\begin{verbatim}
## Warning: package 'data.table' was built under R version 3.4.4
\end{verbatim}

\begin{Shaded}
\begin{Highlighting}[]
\CommentTok{# 1- converter o arquivo para .csv usando a função fwf2csv () do pacote descr}
\CommentTok{# 2- carregar o BD com a função fread() do pacote data.table, que usa menos}
\CommentTok{#    memória que a função read.fwf() do pacote ...}
\CommentTok{#install.packages("sqldf") # p/carregar partes de BD's de grandes dimensões}
\KeywordTok{library}\NormalTok{(sqldf) }\CommentTok{# R_SC: (p. 53-54)}
\end{Highlighting}
\end{Shaded}

\begin{verbatim}
## Warning: package 'sqldf' was built under R version 3.4.4
\end{verbatim}

\begin{verbatim}
## Loading required package: gsubfn
\end{verbatim}

\begin{verbatim}
## Warning: package 'gsubfn' was built under R version 3.4.4
\end{verbatim}

\begin{verbatim}
## Loading required package: proto
\end{verbatim}

\begin{verbatim}
## Warning: package 'proto' was built under R version 3.4.4
\end{verbatim}

\begin{verbatim}
## Loading required package: RSQLite
\end{verbatim}

\begin{verbatim}
## Warning: package 'RSQLite' was built under R version 3.4.4
\end{verbatim}

\begin{Shaded}
\begin{Highlighting}[]
\CommentTok{#install.packages("descr")#Um pacote tem de ser instalado 1 vez no seu micro}
\KeywordTok{library}\NormalTok{(descr) }\CommentTok{# Ativado o pacote, suas funções são disponibilizadas p/uso}
\CommentTok{# "descr" é um pacote com funções voltadas para Estatística Descritiva}

\CommentTok{#install.packages("gdata")}
\KeywordTok{library}\NormalTok{(gdata) }\CommentTok{# pacote para manipulação de dados (BD's) (p. 45)}
\end{Highlighting}
\end{Shaded}

\begin{verbatim}
## Warning: execução do comando '"C:\PROGRA~2\LYX2~1.2\Perl\bin\perl.exe" "C:/
## Users/M/Documents/R/win-library/3.4/gdata/perl/supportedFormats.pl"' teve
## status 2
\end{verbatim}

\begin{verbatim}
## gdata: Unable to load perl libaries needed by read.xls()
## gdata: to support 'XLX' (Excel 97-2004) files.
\end{verbatim}

\begin{verbatim}
## 
\end{verbatim}

\begin{verbatim}
## gdata: Unable to load perl libaries needed by read.xls()
## gdata: to support 'XLSX' (Excel 2007+) files.
\end{verbatim}

\begin{verbatim}
## 
\end{verbatim}

\begin{verbatim}
## gdata: Run the function 'installXLSXsupport()'
## gdata: to automatically download and install the perl
## gdata: libaries needed to support Excel XLS and XLSX formats.
\end{verbatim}

\begin{verbatim}
## 
## Attaching package: 'gdata'
\end{verbatim}

\begin{verbatim}
## The following objects are masked from 'package:data.table':
## 
##     first, last
\end{verbatim}

\begin{verbatim}
## The following object is masked from 'package:stats':
## 
##     nobs
\end{verbatim}

\begin{verbatim}
## The following object is masked from 'package:utils':
## 
##     object.size
\end{verbatim}

\begin{verbatim}
## The following object is masked from 'package:base':
## 
##     startsWith
\end{verbatim}

\begin{Shaded}
\begin{Highlighting}[]
               \CommentTok{# No Windows poderá ser necessário instalar ActivePerl}
               \CommentTok{# ou outro interpretador da linguagem perl.}

\CommentTok{#install.packages("igraph") # Océu é o limite!!!!!!!!!!!!!!!!!!!!!!!!!!!!!!}
\KeywordTok{library}\NormalTok{(igraph) }\CommentTok{# pacote para Network Analysis and Visualization}
\end{Highlighting}
\end{Shaded}

\begin{verbatim}
## Warning: package 'igraph' was built under R version 3.4.4
\end{verbatim}

\begin{verbatim}
## 
## Attaching package: 'igraph'
\end{verbatim}

\begin{verbatim}
## The following objects are masked from 'package:stats':
## 
##     decompose, spectrum
\end{verbatim}

\begin{verbatim}
## The following object is masked from 'package:base':
## 
##     union
\end{verbatim}

\begin{Shaded}
\begin{Highlighting}[]
                \CommentTok{# R_CS: cap. 12- Análise de Redes Sociais (com grafos)}

\CommentTok{#install.packages("knitr")}
\KeywordTok{library}\NormalTok{(knitr) }\CommentTok{# pacote para geração de Relatório Dinâmico em R (p. 119)}
\end{Highlighting}
\end{Shaded}

\begin{verbatim}
## Warning: package 'knitr' was built under R version 3.4.4
\end{verbatim}

\begin{Shaded}
\begin{Highlighting}[]
\CommentTok{#install.packages("memisc") # para surveys}
\KeywordTok{library}\NormalTok{(memisc) }\CommentTok{# pacote para manipulação de pesquisa de dados (p. 66, 89)}
\end{Highlighting}
\end{Shaded}

\begin{verbatim}
## Loading required package: lattice
\end{verbatim}

\begin{verbatim}
## Loading required package: MASS
\end{verbatim}

\begin{verbatim}
## 
## Attaching package: 'memisc'
\end{verbatim}

\begin{verbatim}
## The following objects are masked from 'package:stats':
## 
##     contr.sum, contr.treatment, contrasts
\end{verbatim}

\begin{verbatim}
## The following object is masked from 'package:base':
## 
##     as.array
\end{verbatim}

\begin{Shaded}
\begin{Highlighting}[]
                \CommentTok{# e para apresentação de análises de seus resultados}

\CommentTok{#install.packages("rgdal") # para exibição de Mapas e dados espacializados}
\KeywordTok{library}\NormalTok{(rgdal) }\CommentTok{# R_SC: cap. 11- Mapas (p. 134-139)}
\end{Highlighting}
\end{Shaded}

\begin{verbatim}
## Warning: package 'rgdal' was built under R version 3.4.4
\end{verbatim}

\begin{verbatim}
## Loading required package: sp
\end{verbatim}

\begin{verbatim}
## Warning: package 'sp' was built under R version 3.4.4
\end{verbatim}

\begin{verbatim}
## rgdal: version: 1.2-18, (SVN revision 718)
##  Geospatial Data Abstraction Library extensions to R successfully loaded
##  Loaded GDAL runtime: GDAL 2.2.3, released 2017/11/20
##  Path to GDAL shared files: C:/Users/M/Documents/R/win-library/3.4/rgdal/gdal
##  GDAL binary built with GEOS: TRUE 
##  Loaded PROJ.4 runtime: Rel. 4.9.3, 15 August 2016, [PJ_VERSION: 493]
##  Path to PROJ.4 shared files: C:/Users/M/Documents/R/win-library/3.4/rgdal/proj
##  Linking to sp version: 1.2-7
\end{verbatim}

\begin{Shaded}
\begin{Highlighting}[]
\CommentTok{# Requer a instalação do pacote sp}
\CommentTok{#install.packages("sp")}
\KeywordTok{library}\NormalTok{(sp)}

\CommentTok{#install.packages("rmarkdown") # para instalação do RMarkdown}
\KeywordTok{library}\NormalTok{(rmarkdown) }\CommentTok{#R_SC: geração Relatórios Dinâmicos no RStudio(p. 115-19)}
\end{Highlighting}
\end{Shaded}

\begin{verbatim}
## Warning: package 'rmarkdown' was built under R version 3.4.4
\end{verbatim}

\begin{Shaded}
\begin{Highlighting}[]
\CommentTok{# Requer instalação de outros pacotes p/rodar o RMarkdown dentro do RStudio}
\CommentTok{#install.packages("htmltools") - esse não precisou, veio c/o RMarkdown}
\KeywordTok{library}\NormalTok{(htmltools) }\CommentTok{# Ferramentas para HTML}
\end{Highlighting}
\end{Shaded}

\begin{verbatim}
## Warning: package 'htmltools' was built under R version 3.4.4
\end{verbatim}

\begin{verbatim}
## 
## Attaching package: 'htmltools'
\end{verbatim}

\begin{verbatim}
## The following object is masked from 'package:memisc':
## 
##     css
\end{verbatim}

\begin{Shaded}
\begin{Highlighting}[]
\CommentTok{#install.packages("caTools")#   - esse precisou e instalou o bitops}
\KeywordTok{library}\NormalTok{(caTools)}\CommentTok{#Tools: moving windows statistics, GIF, Base64, ROC AUC etc.}
\end{Highlighting}
\end{Shaded}

\begin{verbatim}
## Warning: package 'caTools' was built under R version 3.4.4
\end{verbatim}

\begin{Shaded}
\begin{Highlighting}[]
\CommentTok{#install.packages(c("bindr", "bindrcpp", "Rcpp", "stringi")) é uma função composta}
\KeywordTok{library}\NormalTok{(bindr)}\CommentTok{# library deve ter package com comprimento 1}
\KeywordTok{library}\NormalTok{(bindrcpp)}\CommentTok{#}
\KeywordTok{library}\NormalTok{(Rcpp)}\CommentTok{#}
\KeywordTok{library}\NormalTok{(stringi)}\CommentTok{#}
\end{Highlighting}
\end{Shaded}

\begin{verbatim}
## Warning: package 'stringi' was built under R version 3.4.4
\end{verbatim}

\begin{Shaded}
\begin{Highlighting}[]
\CommentTok{#install.packages(c("cluster", "Matrix"), lib="C:/Users/cleuler-bn/Documents/R/R-3.4.4/library")}
\KeywordTok{library}\NormalTok{(cluster)}\CommentTok{#}
\KeywordTok{library}\NormalTok{(Matrix)}\CommentTok{#}

\CommentTok{#install.packages(c("financial", "FinancialInstrument", "FinancialMath"))}
\KeywordTok{library}\NormalTok{(financial)}\CommentTok{#}
\KeywordTok{library}\NormalTok{(FinancialInstrument)}\CommentTok{#}
\end{Highlighting}
\end{Shaded}

\begin{verbatim}
## Warning: package 'FinancialInstrument' was built under R version 3.4.4
\end{verbatim}

\begin{verbatim}
## Loading required package: quantmod
\end{verbatim}

\begin{verbatim}
## Warning: package 'quantmod' was built under R version 3.4.4
\end{verbatim}

\begin{verbatim}
## Loading required package: xts
\end{verbatim}

\begin{verbatim}
## Warning: package 'xts' was built under R version 3.4.4
\end{verbatim}

\begin{verbatim}
## Loading required package: zoo
\end{verbatim}

\begin{verbatim}
## Warning: package 'zoo' was built under R version 3.4.4
\end{verbatim}

\begin{verbatim}
## 
## Attaching package: 'zoo'
\end{verbatim}

\begin{verbatim}
## The following objects are masked from 'package:base':
## 
##     as.Date, as.Date.numeric
\end{verbatim}

\begin{verbatim}
## 
## Attaching package: 'xts'
\end{verbatim}

\begin{verbatim}
## The following objects are masked from 'package:gdata':
## 
##     first, last
\end{verbatim}

\begin{verbatim}
## The following objects are masked from 'package:data.table':
## 
##     first, last
\end{verbatim}

\begin{verbatim}
## Loading required package: TTR
\end{verbatim}

\begin{verbatim}
## Version 0.4-0 included new data defaults. See ?getSymbols.
\end{verbatim}

\begin{Shaded}
\begin{Highlighting}[]
\KeywordTok{library}\NormalTok{(FinancialMath)}\CommentTok{#}
\end{Highlighting}
\end{Shaded}

\begin{verbatim}
## 
## Attaching package: 'FinancialMath'
\end{verbatim}

\begin{verbatim}
## The following object is masked from 'package:FinancialInstrument':
## 
##     bond
\end{verbatim}

\begin{Shaded}
\begin{Highlighting}[]
\CommentTok{#install.packages("tinytex")#   - foi preciso instalar para gerar arquivo .pdf direto do RMarkdown}
\CommentTok{#library(tinytex)# para carregar o pacote tinytex, que gera arquivo .tex e certamente converte para .pdf}
\CommentTok{#                  Mas isso gerou uma v2.pdf no formato de uma janela do PDF, sem os marcadores do lado esquerdo!!!!!}
\CommentTok{#                  Do Jeito antigo estava melhor e gravava um .pdf na pasta R_CS/Aula1 que ao abrir no Adobe}
\CommentTok{#                  apresentou na parte esquerda da tela do Adobe todos os marcadores das secções do arquivo (melhor)!}

\CommentTok{# Um *look* na sua **Estação de Trabalho** desta sessão do **R** versão 3.4.3}
\KeywordTok{sessionInfo}\NormalTok{()}
\end{Highlighting}
\end{Shaded}

\begin{verbatim}
## R version 3.4.3 (2017-11-30)
## Platform: x86_64-w64-mingw32/x64 (64-bit)
## Running under: Windows 10 x64 (build 16299)
## 
## Matrix products: default
## 
## locale:
## [1] LC_COLLATE=Portuguese_Brazil.1252  LC_CTYPE=Portuguese_Brazil.1252   
## [3] LC_MONETARY=Portuguese_Brazil.1252 LC_NUMERIC=C                      
## [5] LC_TIME=Portuguese_Brazil.1252    
## 
## attached base packages:
## [1] stats     graphics  grDevices utils     datasets  methods   base     
## 
## other attached packages:
##  [1] FinancialMath_0.1.1       FinancialInstrument_1.3.1
##  [3] quantmod_0.4-13           TTR_0.23-3               
##  [5] xts_0.10-2                zoo_1.8-1                
##  [7] financial_0.2             Matrix_1.2-12            
##  [9] cluster_2.0.6             stringi_1.1.7            
## [11] Rcpp_0.12.15              bindrcpp_0.2             
## [13] bindr_0.1                 caTools_1.17.1           
## [15] htmltools_0.3.6           rmarkdown_1.9            
## [17] rgdal_1.2-18              sp_1.2-7                 
## [19] memisc_0.99.14.9          MASS_7.3-47              
## [21] lattice_0.20-35           knitr_1.20               
## [23] igraph_1.2.1              gdata_2.18.0             
## [25] descr_1.1.4               sqldf_0.4-11             
## [27] RSQLite_2.0               gsubfn_0.7               
## [29] proto_1.0.0               data.table_1.10.4-3      
## [31] foreign_0.8-69           
## 
## loaded via a namespace (and not attached):
##  [1] compiler_3.4.3  bitops_1.0-6    tools_3.4.3     digest_0.6.15  
##  [5] bit_1.1-12      evaluate_0.10.1 memoise_1.1.0   pkgconfig_2.0.1
##  [9] DBI_0.8         curl_3.2        yaml_2.1.18     repr_0.12.0    
## [13] stringr_1.3.0   gtools_3.5.0    rprojroot_1.3-2 bit64_0.9-7    
## [17] grid_3.4.3      tcltk_3.4.3     blob_1.1.1      magrittr_1.5   
## [21] backports_1.1.2 xtable_1.8-2    chron_2.3-52
\end{verbatim}

\begin{Shaded}
\begin{Highlighting}[]
\CommentTok{# Os interessados em assinar a *Lista Brasileira do R* -- [R-br] da **UFPR** devem [acessar](http://listas.inf.ufpr.br/cgi-bin/mailman/listinfo/r-br)}

\CommentTok{#[=========================================================================]}
\CommentTok{#[                                                                         ]}
\CommentTok{#[=========================================================================]}
\end{Highlighting}
\end{Shaded}

\section{GERANDO UMA BD - AS ALTURAS E PESOS DA
TURMA}\label{gerando-uma-bd---as-alturas-e-pesos-da-turma}

\subsection{\texorpdfstring{Criando vetores: um tipo
}{Criando vetores: um  tipo }}\label{criando-vetores-um-tipo}

 é um fundamental do R; é uma estrutura de dados que permite armazenar
um conjunto de valores de um mesmo tipo sob um mesmo nome de .\\
Seus principais tipos são:\\
: \\
\\
\\
\\
O valor NA pode ser armazenado como valor NULL em qualquer um desses
tipos. A função \texttt{vector()} cria um vetor do R.\\
Seus \emph{argumentos} são: \emph{mode} (modo ou \emph{storage mode of
an Object}) e \emph{length} (comprimento do ).

Exemplos de uso dessa função.

\section{Criando vetores vazios de vários tipos básicos e de um tipo
especial}\label{criando-vetores-vazios-de-varios-tipos-basicos-e-de-um-tipo-especial}

\begin{Shaded}
\begin{Highlighting}[]
\CommentTok{#getwd()}
\CommentTok{#setwd("~/../R_CS/Aula3")}

\NormalTok{x <-}\StringTok{ }\KeywordTok{vector}\NormalTok{(}\DataTypeTok{mode =} \StringTok{"character"}\NormalTok{, }\DataTypeTok{length =} \DecValTok{5}\NormalTok{)}
\NormalTok{y <-}\StringTok{ }\KeywordTok{vector}\NormalTok{(}\DataTypeTok{mode =} \StringTok{"numeric"}\NormalTok{, }\DataTypeTok{length =} \DecValTok{7}\NormalTok{)}
\NormalTok{z <-}\StringTok{ }\KeywordTok{vector}\NormalTok{(}\DataTypeTok{mode =} \StringTok{"logical"}\NormalTok{, }\DataTypeTok{length =} \DecValTok{4}\NormalTok{)}

\NormalTok{x}
\end{Highlighting}
\end{Shaded}

\begin{verbatim}
## [1] "" "" "" "" ""
\end{verbatim}

\begin{Shaded}
\begin{Highlighting}[]
\NormalTok{y}
\end{Highlighting}
\end{Shaded}

\begin{verbatim}
## [1] 0 0 0 0 0 0 0
\end{verbatim}

\begin{Shaded}
\begin{Highlighting}[]
\NormalTok{z}
\end{Highlighting}
\end{Shaded}

\begin{verbatim}
## [1] FALSE FALSE FALSE FALSE
\end{verbatim}

\begin{Shaded}
\begin{Highlighting}[]
\KeywordTok{Sys.Date}\NormalTok{()}
\end{Highlighting}
\end{Shaded}

\begin{verbatim}
## [1] "2018-04-18"
\end{verbatim}

\begin{Shaded}
\begin{Highlighting}[]
\NormalTok{hoje<-}\KeywordTok{Sys.Date}\NormalTok{() }\CommentTok{# cria um objeto do tipo <Date>, que foi funcionalmente criado (CRAN)}
\NormalTok{hoje}
\end{Highlighting}
\end{Shaded}

\begin{verbatim}
## [1] "2018-04-18"
\end{verbatim}

\begin{Shaded}
\begin{Highlighting}[]
\KeywordTok{format}\NormalTok{(hoje, }\StringTok{"%d %b %Y"}\NormalTok{) }\CommentTok{# exibe a data de hoje num formato padrão ISO/ABNT}
\end{Highlighting}
\end{Shaded}

\begin{verbatim}
## [1] "18 abr 2018"
\end{verbatim}

\begin{Shaded}
\begin{Highlighting}[]
\NormalTok{dezsemanas<-}\KeywordTok{seq}\NormalTok{(hoje, }\DataTypeTok{len =} \DecValTok{10}\NormalTok{, }\DataTypeTok{by =} \StringTok{"1 week"}\NormalTok{)}
\NormalTok{dez.semanas<-}\KeywordTok{seq}\NormalTok{(hoje, }\DataTypeTok{len =} \DecValTok{10}\NormalTok{, }\DataTypeTok{by =} \StringTok{"1 week"}\NormalTok{)}
\KeywordTok{rm}\NormalTok{(dezsemanas)}
\NormalTok{dez.semanas}
\end{Highlighting}
\end{Shaded}

\begin{verbatim}
##  [1] "2018-04-18" "2018-04-25" "2018-05-02" "2018-05-09" "2018-05-16"
##  [6] "2018-05-23" "2018-05-30" "2018-06-06" "2018-06-13" "2018-06-20"
\end{verbatim}

\begin{Shaded}
\begin{Highlighting}[]
\NormalTok{dez.semanas<-dez.semanas}\OperatorTok{-}\DecValTok{7}
\NormalTok{dez.semanas}
\end{Highlighting}
\end{Shaded}

\begin{verbatim}
##  [1] "2018-04-11" "2018-04-18" "2018-04-25" "2018-05-02" "2018-05-09"
##  [6] "2018-05-16" "2018-05-23" "2018-05-30" "2018-06-06" "2018-06-13"
\end{verbatim}

\begin{Shaded}
\begin{Highlighting}[]
\CommentTok{#w <- vector(mode = "Date", length = 6) # *Error*}
\CommentTok{# porque não é um parâmetro válido para o argumento *mode* da função vector()}

\CommentTok{# Criando um vetor de datas para servir de rótulos para nossa série temporal experimental}
\CommentTok{#dez_semanas<-seq(c("2018-04-11"), len = 10, by = "1 week") # Error}
\CommentTok{# Porque "2018-04-11" é um tipo básico <char> e não um tipo especial <Date>.}
\CommentTok{# É preciso converter <char> em <Date>. E, claro, há uma função para isso!!!}
\NormalTok{dez_semanas<-}\KeywordTok{seq.Date}\NormalTok{(}\DataTypeTok{from =} \KeywordTok{as.Date}\NormalTok{(}\StringTok{"2018-04-11"}\NormalTok{), }\DataTypeTok{len =} \DecValTok{10}\NormalTok{, }\DataTypeTok{by =} \StringTok{"1 week"}\NormalTok{)}
\NormalTok{dez_semanas<-}\KeywordTok{seq}\NormalTok{(}\DataTypeTok{from =} \KeywordTok{as.Date}\NormalTok{(}\StringTok{"2018-04-11"}\NormalTok{), }\DataTypeTok{len =} \DecValTok{10}\NormalTok{, }\DataTypeTok{by =} \StringTok{"1 week"}\NormalTok{)}
\NormalTok{dez_semanas}
\end{Highlighting}
\end{Shaded}

\begin{verbatim}
##  [1] "2018-04-11" "2018-04-18" "2018-04-25" "2018-05-02" "2018-05-09"
##  [6] "2018-05-16" "2018-05-23" "2018-05-30" "2018-06-06" "2018-06-13"
\end{verbatim}

\begin{Shaded}
\begin{Highlighting}[]
\CommentTok{# o NA é um valor que pode ser atribuído a uma posição de um vetor de qualquer tipo}
\NormalTok{a<-}\KeywordTok{c}\NormalTok{(}\DecValTok{1}\OperatorTok{:}\DecValTok{1000}\NormalTok{) }\CommentTok{# Uma composição da função c() com a função seq(), simbolizada pelos :}
\KeywordTok{sum}\NormalTok{(a)}
\end{Highlighting}
\end{Shaded}

\begin{verbatim}
## [1] 500500
\end{verbatim}

\begin{Shaded}
\begin{Highlighting}[]
\KeywordTok{args}\NormalTok{(sum)}
\end{Highlighting}
\end{Shaded}

\begin{verbatim}
## function (..., na.rm = FALSE) 
## NULL
\end{verbatim}

\begin{Shaded}
\begin{Highlighting}[]
\NormalTok{a[}\DecValTok{1001}\NormalTok{]<-}\OtherTok{NA}
\NormalTok{a}
\end{Highlighting}
\end{Shaded}

\begin{verbatim}
##    [1]    1    2    3    4    5    6    7    8    9   10   11   12   13
##   [14]   14   15   16   17   18   19   20   21   22   23   24   25   26
##   [27]   27   28   29   30   31   32   33   34   35   36   37   38   39
##   [40]   40   41   42   43   44   45   46   47   48   49   50   51   52
##   [53]   53   54   55   56   57   58   59   60   61   62   63   64   65
##   [66]   66   67   68   69   70   71   72   73   74   75   76   77   78
##   [79]   79   80   81   82   83   84   85   86   87   88   89   90   91
##   [92]   92   93   94   95   96   97   98   99  100  101  102  103  104
##  [105]  105  106  107  108  109  110  111  112  113  114  115  116  117
##  [118]  118  119  120  121  122  123  124  125  126  127  128  129  130
##  [131]  131  132  133  134  135  136  137  138  139  140  141  142  143
##  [144]  144  145  146  147  148  149  150  151  152  153  154  155  156
##  [157]  157  158  159  160  161  162  163  164  165  166  167  168  169
##  [170]  170  171  172  173  174  175  176  177  178  179  180  181  182
##  [183]  183  184  185  186  187  188  189  190  191  192  193  194  195
##  [196]  196  197  198  199  200  201  202  203  204  205  206  207  208
##  [209]  209  210  211  212  213  214  215  216  217  218  219  220  221
##  [222]  222  223  224  225  226  227  228  229  230  231  232  233  234
##  [235]  235  236  237  238  239  240  241  242  243  244  245  246  247
##  [248]  248  249  250  251  252  253  254  255  256  257  258  259  260
##  [261]  261  262  263  264  265  266  267  268  269  270  271  272  273
##  [274]  274  275  276  277  278  279  280  281  282  283  284  285  286
##  [287]  287  288  289  290  291  292  293  294  295  296  297  298  299
##  [300]  300  301  302  303  304  305  306  307  308  309  310  311  312
##  [313]  313  314  315  316  317  318  319  320  321  322  323  324  325
##  [326]  326  327  328  329  330  331  332  333  334  335  336  337  338
##  [339]  339  340  341  342  343  344  345  346  347  348  349  350  351
##  [352]  352  353  354  355  356  357  358  359  360  361  362  363  364
##  [365]  365  366  367  368  369  370  371  372  373  374  375  376  377
##  [378]  378  379  380  381  382  383  384  385  386  387  388  389  390
##  [391]  391  392  393  394  395  396  397  398  399  400  401  402  403
##  [404]  404  405  406  407  408  409  410  411  412  413  414  415  416
##  [417]  417  418  419  420  421  422  423  424  425  426  427  428  429
##  [430]  430  431  432  433  434  435  436  437  438  439  440  441  442
##  [443]  443  444  445  446  447  448  449  450  451  452  453  454  455
##  [456]  456  457  458  459  460  461  462  463  464  465  466  467  468
##  [469]  469  470  471  472  473  474  475  476  477  478  479  480  481
##  [482]  482  483  484  485  486  487  488  489  490  491  492  493  494
##  [495]  495  496  497  498  499  500  501  502  503  504  505  506  507
##  [508]  508  509  510  511  512  513  514  515  516  517  518  519  520
##  [521]  521  522  523  524  525  526  527  528  529  530  531  532  533
##  [534]  534  535  536  537  538  539  540  541  542  543  544  545  546
##  [547]  547  548  549  550  551  552  553  554  555  556  557  558  559
##  [560]  560  561  562  563  564  565  566  567  568  569  570  571  572
##  [573]  573  574  575  576  577  578  579  580  581  582  583  584  585
##  [586]  586  587  588  589  590  591  592  593  594  595  596  597  598
##  [599]  599  600  601  602  603  604  605  606  607  608  609  610  611
##  [612]  612  613  614  615  616  617  618  619  620  621  622  623  624
##  [625]  625  626  627  628  629  630  631  632  633  634  635  636  637
##  [638]  638  639  640  641  642  643  644  645  646  647  648  649  650
##  [651]  651  652  653  654  655  656  657  658  659  660  661  662  663
##  [664]  664  665  666  667  668  669  670  671  672  673  674  675  676
##  [677]  677  678  679  680  681  682  683  684  685  686  687  688  689
##  [690]  690  691  692  693  694  695  696  697  698  699  700  701  702
##  [703]  703  704  705  706  707  708  709  710  711  712  713  714  715
##  [716]  716  717  718  719  720  721  722  723  724  725  726  727  728
##  [729]  729  730  731  732  733  734  735  736  737  738  739  740  741
##  [742]  742  743  744  745  746  747  748  749  750  751  752  753  754
##  [755]  755  756  757  758  759  760  761  762  763  764  765  766  767
##  [768]  768  769  770  771  772  773  774  775  776  777  778  779  780
##  [781]  781  782  783  784  785  786  787  788  789  790  791  792  793
##  [794]  794  795  796  797  798  799  800  801  802  803  804  805  806
##  [807]  807  808  809  810  811  812  813  814  815  816  817  818  819
##  [820]  820  821  822  823  824  825  826  827  828  829  830  831  832
##  [833]  833  834  835  836  837  838  839  840  841  842  843  844  845
##  [846]  846  847  848  849  850  851  852  853  854  855  856  857  858
##  [859]  859  860  861  862  863  864  865  866  867  868  869  870  871
##  [872]  872  873  874  875  876  877  878  879  880  881  882  883  884
##  [885]  885  886  887  888  889  890  891  892  893  894  895  896  897
##  [898]  898  899  900  901  902  903  904  905  906  907  908  909  910
##  [911]  911  912  913  914  915  916  917  918  919  920  921  922  923
##  [924]  924  925  926  927  928  929  930  931  932  933  934  935  936
##  [937]  937  938  939  940  941  942  943  944  945  946  947  948  949
##  [950]  950  951  952  953  954  955  956  957  958  959  960  961  962
##  [963]  963  964  965  966  967  968  969  970  971  972  973  974  975
##  [976]  976  977  978  979  980  981  982  983  984  985  986  987  988
##  [989]  989  990  991  992  993  994  995  996  997  998  999 1000   NA
\end{verbatim}

\begin{Shaded}
\begin{Highlighting}[]
\KeywordTok{sum}\NormalTok{(a)}
\end{Highlighting}
\end{Shaded}

\begin{verbatim}
## [1] NA
\end{verbatim}

\begin{Shaded}
\begin{Highlighting}[]
\KeywordTok{sum}\NormalTok{(a, }\DataTypeTok{na.rm =} \OtherTok{TRUE}\NormalTok{) }\CommentTok{# 1 NA será removido para não indeterminar a soma de 1000 parcelas}
\end{Highlighting}
\end{Shaded}

\begin{verbatim}
## [1] 500500
\end{verbatim}

\begin{Shaded}
\begin{Highlighting}[]
\KeywordTok{mean}\NormalTok{(a)}
\end{Highlighting}
\end{Shaded}

\begin{verbatim}
## [1] NA
\end{verbatim}

\begin{Shaded}
\begin{Highlighting}[]
\KeywordTok{mean}\NormalTok{(a, }\DataTypeTok{na.rm =} \OtherTok{TRUE}\NormalTok{)}\CommentTok{# nem sua média}
\end{Highlighting}
\end{Shaded}

\begin{verbatim}
## [1] 500.5
\end{verbatim}

\begin{Shaded}
\begin{Highlighting}[]
\KeywordTok{sd}\NormalTok{(a)}
\end{Highlighting}
\end{Shaded}

\begin{verbatim}
## [1] NA
\end{verbatim}

\begin{Shaded}
\begin{Highlighting}[]
\KeywordTok{sd}\NormalTok{(a, }\DataTypeTok{na.rm =} \OtherTok{TRUE}\NormalTok{)  }\CommentTok{# nem seu desvio padrão (standart desviation)}
\end{Highlighting}
\end{Shaded}

\begin{verbatim}
## [1] 288.8194
\end{verbatim}

\begin{Shaded}
\begin{Highlighting}[]
\NormalTok{###########################################################################################}
\CommentTok{# CUIDADO PORQUE UM ÚNICO **NA** NUMA BD PROPAGA SUA CAPACIDADE DE IMPEDIR QUE CÁLCULOS DE ESTATÍSTICA SEJAM PROCESSADOS}
\NormalTok{###########################################################################################}
\end{Highlighting}
\end{Shaded}

\section{Criando vetores para receber variáveis de uma
BD}\label{criando-vetores-para-receber-variaveis-de-uma-bd}

\begin{Shaded}
\begin{Highlighting}[]
\NormalTok{nomes<-}\KeywordTok{c}\NormalTok{(}\StringTok{"Bernard"}\NormalTok{,}
\StringTok{"Carlos"}\NormalTok{,}
\StringTok{"Cleuler"}\NormalTok{,}
\StringTok{"Helber"}\NormalTok{,}
\StringTok{"Larissa"}\NormalTok{,}
\StringTok{"Mateus"}\NormalTok{,}
\StringTok{"Michell"}\NormalTok{,}
\StringTok{"Nayana"}\NormalTok{,}
\StringTok{"Paula"}\NormalTok{,}
\StringTok{"Rafael"}\NormalTok{,}
\StringTok{"Tatiane"}\NormalTok{,}
\StringTok{"Thiago"}\NormalTok{,}
\StringTok{"Wesley"}\NormalTok{)}
\NormalTok{h<-}\KeywordTok{c}\NormalTok{(}\FloatTok{1.74}\NormalTok{,}\FloatTok{1.63}\NormalTok{,}\FloatTok{1.77}\NormalTok{,}\FloatTok{1.75}\NormalTok{,}\OtherTok{NA}\NormalTok{,}\FloatTok{1.85}\NormalTok{,}\FloatTok{1.6}\NormalTok{,}\OtherTok{NA}\NormalTok{,}\FloatTok{1.55}\NormalTok{,}\FloatTok{1.7}\NormalTok{,}\FloatTok{1.63}\NormalTok{,}\FloatTok{1.7}\NormalTok{,}\FloatTok{1.75}\NormalTok{)}
\NormalTok{p<-}\KeywordTok{c}\NormalTok{(}\FloatTok{63.8}\NormalTok{,}
\FloatTok{79.5}\NormalTok{,}
\FloatTok{81.6}\NormalTok{,}
\FloatTok{81.3}\NormalTok{,}
\DecValTok{49}\NormalTok{,}
\FloatTok{82.7}\NormalTok{,}
\FloatTok{57.6}\NormalTok{,}
\FloatTok{56.3}\NormalTok{,}
\FloatTok{72.4}\NormalTok{,}
\FloatTok{62.1}\NormalTok{,}
\FloatTok{52.6}\NormalTok{,}
\FloatTok{82.1}\NormalTok{,}
\FloatTok{81.9}\NormalTok{)}

\NormalTok{dez_semanas[}\DecValTok{1}\NormalTok{]}
\end{Highlighting}
\end{Shaded}

\begin{verbatim}
## [1] "2018-04-11"
\end{verbatim}

\begin{Shaded}
\begin{Highlighting}[]
\NormalTok{nomes}
\end{Highlighting}
\end{Shaded}

\begin{verbatim}
##  [1] "Bernard" "Carlos"  "Cleuler" "Helber"  "Larissa" "Mateus"  "Michell"
##  [8] "Nayana"  "Paula"   "Rafael"  "Tatiane" "Thiago"  "Wesley"
\end{verbatim}

\begin{Shaded}
\begin{Highlighting}[]
\NormalTok{h}
\end{Highlighting}
\end{Shaded}

\begin{verbatim}
##  [1] 1.74 1.63 1.77 1.75   NA 1.85 1.60   NA 1.55 1.70 1.63 1.70 1.75
\end{verbatim}

\begin{Shaded}
\begin{Highlighting}[]
\NormalTok{p}
\end{Highlighting}
\end{Shaded}

\begin{verbatim}
##  [1] 63.8 79.5 81.6 81.3 49.0 82.7 57.6 56.3 72.4 62.1 52.6 82.1 81.9
\end{verbatim}

\section{Exercícios - Para Resolução em
Sala}\label{exercicios---para-resolucao-em-sala}

Refletir e responder às seguintes questões \emph{pragmáticas}:\\
1) Qual a altura média da sua turma de R?\\
2) Qual o peso médio da sua turma de R na aula do dia 11 abr. 2018?\\
3) Qual o número médio de caracteres dos prenomes dos alunos da turma de
R que mediram seus pesos no dia 11 abr. 2018? 4) Qual o número médio de
caracteres dos prenomes dos alunos matriculados nesta turma de R? 5)
Qual o desvio padrão das médias encontradas?\\
6) Quem está abaixo e acima da média mais ou menos 1 Desvio Padrão? 7)
Calcule o IMC de cada observação do dia 11 abr. 2018.

\begin{Shaded}
\begin{Highlighting}[]
\CommentTok{# Invocando as funções mean() e sd() para uma <var> <vector> <num>}
\CommentTok{#1) Média e Desvio Padrão (#5) das alturas dos alunos:}
\NormalTok{hm<-}\StringTok{ }\KeywordTok{mean}\NormalTok{(h, }\DataTypeTok{na.rm=}\OtherTok{TRUE}\NormalTok{)}
\NormalTok{hDP<-}\KeywordTok{sd}\NormalTok{(h, }\DataTypeTok{na.rm=}\OtherTok{TRUE}\NormalTok{) }\CommentTok{# Desvio padrão da altura é uma medida de dispersão dessa variável}
\CommentTok{# É uma turma com 8.7 cm de dispersão em torno da altura média de 1.70 m}
\CommentTok{# São brasileiros de estatura mediana, gostam muito de..., mas...}
\CommentTok{#6) Quem está abaixo e acima da média mais ou menos 1 Desvio Padrão?}
\NormalTok{hm}
\end{Highlighting}
\end{Shaded}

\begin{verbatim}
## [1] 1.697273
\end{verbatim}

\begin{Shaded}
\begin{Highlighting}[]
\NormalTok{hDP}
\end{Highlighting}
\end{Shaded}

\begin{verbatim}
## [1] 0.08730303
\end{verbatim}

\begin{Shaded}
\begin{Highlighting}[]
\NormalTok{h[h}\OperatorTok{<}\NormalTok{hm}\OperatorTok{-}\NormalTok{hDP }\OperatorTok{|}\StringTok{ }\NormalTok{h}\OperatorTok{>}\NormalTok{hm}\OperatorTok{+}\NormalTok{hDP]}
\end{Highlighting}
\end{Shaded}

\begin{verbatim}
## [1]   NA 1.85 1.60   NA 1.55
\end{verbatim}

\begin{Shaded}
\begin{Highlighting}[]
\NormalTok{nomes[h}\OperatorTok{<}\NormalTok{hm}\OperatorTok{-}\NormalTok{hDP }\OperatorTok{|}\StringTok{ }\NormalTok{h}\OperatorTok{>}\NormalTok{hm}\OperatorTok{+}\NormalTok{hDP] }\CommentTok{# Eis os outliers da estatura de nossa turma.}
\end{Highlighting}
\end{Shaded}

\begin{verbatim}
## [1] NA        "Mateus"  "Michell" NA        "Paula"
\end{verbatim}

\begin{Shaded}
\begin{Highlighting}[]
\CommentTok{#2) Média e Desvio Padrão (#5) dos pesos dos alunos:}
\NormalTok{pm<-}\StringTok{ }\KeywordTok{mean}\NormalTok{(p, }\DataTypeTok{na.rm=}\OtherTok{TRUE}\NormalTok{) }\CommentTok{# É uma uma turma de magros!!! Conclusão precipitada?}
\NormalTok{pDP<-}\KeywordTok{sd}\NormalTok{(p, }\DataTypeTok{na.rm=}\OtherTok{TRUE}\NormalTok{)}
\CommentTok{# A média do peso da turma no dia 11 abr. 2018 é de 69.5 Kg}
\CommentTok{# O Desvio Padrão dessas 13 obsevrações de peso   = 12.9 Kg}
\CommentTok{#6) Quem está abaixo e acima da média mais ou menos 1 Desvio Padrão?}
\NormalTok{p[p}\OperatorTok{<}\NormalTok{pm}\OperatorTok{-}\NormalTok{pDP }\OperatorTok{|}\StringTok{ }\NormalTok{p}\OperatorTok{>}\NormalTok{pm}\OperatorTok{+}\NormalTok{pDP]}
\end{Highlighting}
\end{Shaded}

\begin{verbatim}
## [1] 49.0 82.7 56.3 52.6
\end{verbatim}

\begin{Shaded}
\begin{Highlighting}[]
\NormalTok{nomes[p}\OperatorTok{<}\NormalTok{pm}\OperatorTok{-}\NormalTok{pDP }\OperatorTok{|}\StringTok{ }\NormalTok{p}\OperatorTok{>}\NormalTok{pm}\OperatorTok{+}\NormalTok{pDP] }\CommentTok{# Eis os outliers da nossa turma.}
\end{Highlighting}
\end{Shaded}

\begin{verbatim}
## [1] "Larissa" "Mateus"  "Nayana"  "Tatiane"
\end{verbatim}

\begin{Shaded}
\begin{Highlighting}[]
\CommentTok{#3) Número médio de caracteres dos prenomes dos alunos da turma de R que mediram seus pesos no dia 11 abr. 2018 e seu Desvio Padrão (#5)}
\KeywordTok{mean}\NormalTok{(}\KeywordTok{length}\NormalTok{(nomes))}
\end{Highlighting}
\end{Shaded}

\begin{verbatim}
## [1] 13
\end{verbatim}

\begin{Shaded}
\begin{Highlighting}[]
\KeywordTok{sd}\NormalTok{(}\KeywordTok{mean}\NormalTok{(}\KeywordTok{length}\NormalTok{(nomes))) }\CommentTok{# retorna um NA. Por que?}
\end{Highlighting}
\end{Shaded}

\begin{verbatim}
## [1] NA
\end{verbatim}

\begin{Shaded}
\begin{Highlighting}[]
\NormalTok{nomes}
\end{Highlighting}
\end{Shaded}

\begin{verbatim}
##  [1] "Bernard" "Carlos"  "Cleuler" "Helber"  "Larissa" "Mateus"  "Michell"
##  [8] "Nayana"  "Paula"   "Rafael"  "Tatiane" "Thiago"  "Wesley"
\end{verbatim}

\begin{Shaded}
\begin{Highlighting}[]
\NormalTok{tam_nomes<-}\KeywordTok{length}\NormalTok{(nomes) }\CommentTok{# cuidado porque retorna o comprimento do vetor names = 13!!!}
\NormalTok{tam_nomes<-}\KeywordTok{nchar}\NormalTok{(nomes, }\DataTypeTok{keepNA =} \OtherTok{NA}\NormalTok{)}
\NormalTok{tam_nomes}
\end{Highlighting}
\end{Shaded}

\begin{verbatim}
##  [1] 7 6 7 6 7 6 7 6 5 6 7 6 6
\end{verbatim}

\begin{Shaded}
\begin{Highlighting}[]
\NormalTok{tam_nomes_media<-}\KeywordTok{mean}\NormalTok{(tam_nomes)}
\NormalTok{tam_nomes_media}
\end{Highlighting}
\end{Shaded}

\begin{verbatim}
## [1] 6.307692
\end{verbatim}

\begin{Shaded}
\begin{Highlighting}[]
\NormalTok{tam_nomes_DP   <-}\KeywordTok{sd}\NormalTok{(tam_nomes)}
\NormalTok{tam_nomes_DP}
\end{Highlighting}
\end{Shaded}

\begin{verbatim}
## [1] 0.6304252
\end{verbatim}

\begin{Shaded}
\begin{Highlighting}[]
\CommentTok{#6) Quem está abaixo e acima da média mais ou menos 1 Desvio Padrão?}
\NormalTok{tam_nomes[tam_nomes}\OperatorTok{<}\NormalTok{tam_nomes_media}\OperatorTok{-}\NormalTok{tam_nomes_DP }\OperatorTok{|}\StringTok{ }\NormalTok{tam_nomes}\OperatorTok{>}\NormalTok{tam_nomes_media}\OperatorTok{+}\NormalTok{tam_nomes_DP]}
\end{Highlighting}
\end{Shaded}

\begin{verbatim}
## [1] 7 7 7 7 5 7
\end{verbatim}

\begin{Shaded}
\begin{Highlighting}[]
\NormalTok{nomes[tam_nomes}\OperatorTok{<}\NormalTok{tam_nomes_media}\OperatorTok{-}\NormalTok{tam_nomes_DP }\OperatorTok{|}\StringTok{ }\NormalTok{tam_nomes}\OperatorTok{>}\NormalTok{tam_nomes_media}\OperatorTok{+}\NormalTok{tam_nomes_DP] }\CommentTok{# Eis os nomes daqueles com tamanhos de nomes outliers da nossa turma.}
\end{Highlighting}
\end{Shaded}

\begin{verbatim}
## [1] "Bernard" "Cleuler" "Larissa" "Michell" "Paula"   "Tatiane"
\end{verbatim}

\begin{Shaded}
\begin{Highlighting}[]
\CommentTok{#7) Cálculo do IMC de cada observação do dia 11 abr. 2018.}
\CommentTok{#O cálculo do IMC é feito dividindo o peso (em quilogramas) pela altura (em metros) elevada ao quadrado.}
\NormalTok{IMC<-p}\OperatorTok{/}\NormalTok{h}\OperatorTok{^}\DecValTok{2}
\NormalTok{IMC}
\end{Highlighting}
\end{Shaded}

\begin{verbatim}
##  [1] 21.07280 29.92209 26.04616 26.54694       NA 24.16362 22.50000
##  [8]       NA 30.13528 21.48789 19.79751 28.40830 26.74286
\end{verbatim}

\begin{Shaded}
\begin{Highlighting}[]
\NormalTok{IMC_m<-}\StringTok{ }\KeywordTok{mean}\NormalTok{(IMC, }\DataTypeTok{na.rm=}\OtherTok{TRUE}\NormalTok{) }\CommentTok{# É uma uma turma de magros!!! Conclusão precipitada?}
\NormalTok{IMC_m }\CommentTok{# = 25.17 Kg/m2     # O IMC médio da turma indica ligeiramente acima do peso normal}
\end{Highlighting}
\end{Shaded}

\begin{verbatim}
## [1] 25.16577
\end{verbatim}

\begin{Shaded}
\begin{Highlighting}[]
\NormalTok{IMC_DP<-}\KeywordTok{sd}\NormalTok{(IMC, }\DataTypeTok{na.rm=}\OtherTok{TRUE}\NormalTok{)}
\NormalTok{IMC_DP}\CommentTok{# =  3.61 Kg/m2}
\end{Highlighting}
\end{Shaded}

\begin{verbatim}
## [1] 3.608471
\end{verbatim}

\begin{Shaded}
\begin{Highlighting}[]
\NormalTok{IMC[IMC}\OperatorTok{<}\FloatTok{18.5} \OperatorTok{|}\StringTok{ }\NormalTok{IMC}\OperatorTok{>=}\DecValTok{25}\NormalTok{]}
\end{Highlighting}
\end{Shaded}

\begin{verbatim}
## [1] 29.92209 26.04616 26.54694       NA       NA 30.13528 28.40830 26.74286
\end{verbatim}

\begin{Shaded}
\begin{Highlighting}[]
\NormalTok{nomes[IMC}\OperatorTok{<}\FloatTok{18.5} \OperatorTok{|}\StringTok{ }\NormalTok{IMC}\OperatorTok{>=}\DecValTok{25}\NormalTok{] }\CommentTok{# Revelar ou não revelar. Eis a questão!!!!!!!!!!!!!!!!!!!!!!!!!!}
\end{Highlighting}
\end{Shaded}

\begin{verbatim}
## [1] "Carlos"  "Cleuler" "Helber"  NA        NA        "Paula"   "Thiago" 
## [8] "Wesley"
\end{verbatim}

\begin{Shaded}
\begin{Highlighting}[]
\CommentTok{# cut() Convert Numeric <num> to Factor <fctr>}
\NormalTok{###########################################################################################}
\CommentTok{# CUIDADO PORQUE UM ÚNICO ERRO DE SINTAXE FAZ COM QUE O COMPILADOR INTERROMPA A EXECUÇÃO DO SRCIPT (CÓDIGO FONTE)  #}
\NormalTok{###########################################################################################}
\end{Highlighting}
\end{Shaded}

Exercíco da Aula n. 2: 8) Redija e salve um script para a função linear
em \textbf{R}. Gere um gráfico para essa função no intervalo {[}0,5{]} e
salve-o no formato .pdf.

\begin{Shaded}
\begin{Highlighting}[]
\NormalTok{###########################################################################################}
\CommentTok{# TENTATIVAS E  ERROS PARA GERAR UM GRÁFICO y=f(x)=a.x + b }
\CommentTok{#}
\CommentTok{# IMPORTANDO UM ARQUIVO FEITO PELO BERNARD E CONVERTENDO-O DE UTF-8 PARA WINDOWS-1252}
\CommentTok{#}
\NormalTok{###########################################################################################}

\KeywordTok{library}\NormalTok{(descr)}
\KeywordTok{library}\NormalTok{(stats)}
\KeywordTok{getwd}\NormalTok{()}
\end{Highlighting}
\end{Shaded}

\begin{verbatim}
## [1] "C:/Users/M/Documents/R_CS/Aula3"
\end{verbatim}

\begin{Shaded}
\begin{Highlighting}[]
\CommentTok{#linhas<-readLines("Plottar_grafico.R")}
\CommentTok{#linhas<-fromUTF8(linhas)}
\CommentTok{#writeLines(linhas, "Plotar_grafico-win.R")}

\CommentTok{#Script desenvolvido para criar graficos}
\CommentTok{#Criar função da equação da reta > y = ax+b}
\CommentTok{#------------------Parametros------------------}
\CommentTok{#     a = Coeficiente linear}
\CommentTok{#     b = Coeficiente Angular}
\CommentTok{#     x = Vetor de valores no Eixo X}
\CommentTok{#     y = Vetor de valores no Eixo Y}

\CommentTok{#Cria funcao que representa a equacao da reta}
\NormalTok{linear <-}\StringTok{ }\ControlFlowTok{function}\NormalTok{(a,b,x)\{}
\NormalTok{  y <-}\StringTok{ }\NormalTok{a}\OperatorTok{*}\NormalTok{x }\OperatorTok{+}\StringTok{ }\NormalTok{b}
  \KeywordTok{return}\NormalTok{(y)}
\NormalTok{\}}

\NormalTok{a <-}\StringTok{ }\FloatTok{1.5} \CommentTok{#Coeficiente linear}
\NormalTok{b <-}\StringTok{ }\FloatTok{0.5} \CommentTok{#Coeficiente angular}

\CommentTok{#x < 1:10 #Vetor de valores do Eixo X # HAVIA UM ERROR. NÃO DE SINTAXE, MAS DE PROGRAMAÇÃO}
\NormalTok{x <-}\StringTok{ }\DecValTok{1}\OperatorTok{:}\DecValTok{10} \CommentTok{#Vetor de valores do Eixo X}

\CommentTok{#y <- linear(x) # OCORREU OUTRO ERROR. AO CHAMAR A FUNÇÃO linear()}
\NormalTok{y <-}\StringTok{ }\KeywordTok{linear}\NormalTok{(a,b,x) }\CommentTok{# é preciso repassar os parâmetros dos argumentos a e b da função}

\KeywordTok{print}\NormalTok{(y) }\CommentTok{#Mostrar os valores do Eixo Y}
\end{Highlighting}
\end{Shaded}

\begin{verbatim}
##  [1]  2.0  3.5  5.0  6.5  8.0  9.5 11.0 12.5 14.0 15.5
\end{verbatim}

\begin{Shaded}
\begin{Highlighting}[]
\CommentTok{#Parametros do plot}
\CommentTok{#     main = Titulo do grafico}
\CommentTok{#     ylab = Nome do Eixo Y}
\CommentTok{#     xlab = Nome do Eixo X}
\CommentTok{#     type = Tipo de plotagem > l = linha,p = pontos, h = histograma}
\KeywordTok{plot}\NormalTok{(x,y,}\DataTypeTok{main=}\StringTok{'Gráfico Curso R'}\NormalTok{,}\DataTypeTok{ylab=}\StringTok{'Eixo y'}\NormalTok{,}\DataTypeTok{xlab=}\StringTok{'Eixo x'}\NormalTok{,}\DataTypeTok{type=}\StringTok{'o'}\NormalTok{)}
\end{Highlighting}
\end{Shaded}

\includegraphics{PGE-aula3_files/figure-latex/reta-1.pdf}

\begin{Shaded}
\begin{Highlighting}[]
\CommentTok{# Redesenhando o mesmo Gráfico}
\NormalTok{x <-}\StringTok{ }\DecValTok{0}\OperatorTok{:}\DecValTok{10} \CommentTok{#Vetor de valores do Eixo X}
\NormalTok{y <-}\StringTok{ }\KeywordTok{linear}\NormalTok{(a,b,x) }\CommentTok{# é preciso repassar os parâmetros dos argumentos a e b da função}
\KeywordTok{print}\NormalTok{(y) }\CommentTok{#Mostrar os valores do Eixo Y}
\end{Highlighting}
\end{Shaded}

\begin{verbatim}
##  [1]  0.5  2.0  3.5  5.0  6.5  8.0  9.5 11.0 12.5 14.0 15.5
\end{verbatim}

\begin{Shaded}
\begin{Highlighting}[]
\NormalTok{y1 <-}\StringTok{ }\KeywordTok{linear}\NormalTok{(}\DecValTok{2}\NormalTok{,}\DecValTok{0}\NormalTok{,x)}
\NormalTok{y2 <-}\StringTok{ }\KeywordTok{linear}\NormalTok{(}\FloatTok{0.5}\NormalTok{,}\DecValTok{0}\NormalTok{,x)}

\KeywordTok{plot.new}\NormalTok{()}
\KeywordTok{plot}\NormalTok{(x,y,}\DataTypeTok{xlim=}\KeywordTok{c}\NormalTok{(}\DecValTok{0}\NormalTok{,}\DecValTok{10}\NormalTok{),}\DataTypeTok{ylim=}\KeywordTok{c}\NormalTok{(}\DecValTok{0}\NormalTok{,}\DecValTok{16}\NormalTok{),}\DataTypeTok{main=}\StringTok{'Gráfico Curso R'}\NormalTok{,}\DataTypeTok{ylab=}\StringTok{'y'}\NormalTok{,}\DataTypeTok{xlab=}\StringTok{'x'}\NormalTok{,}\DataTypeTok{type=}\StringTok{'o'}\NormalTok{)}
\end{Highlighting}
\end{Shaded}

\includegraphics{PGE-aula3_files/figure-latex/reta-2.pdf}

\begin{Shaded}
\begin{Highlighting}[]
\CommentTok{#lines(x,y1, col="blue") # Error: plot.new has not been called yet}
\CommentTok{#lines(x,y2, col="red")}
\end{Highlighting}
\end{Shaded}

Exercícios remanescentes da Aula n. 01:\\
9) Apresente duas funções lineraes que sejam inversas. Plote-as
juntamente com a função identidade. 10) Descrever os tipos de variáveis
geradas na Job Area e suas características.

\textbf{Trabalho Final do curso: } Na primeira aula registrar a altura
declarada e medir e registrar o peso de cada aluno, que poderá
identificar-se com um apelido.\\
Em cada aula medir e registrar numa BD o peso de cada aluno numa
sequencia aleatória.\\
Calcular o IMC de cada observação e apontar para os IMC's abaixo ou
acima da faixa recomendada pela literatura médica.\\
Calcular a média e o desvio padrão do IMC da população observada;
detectar os pontos \emph{outliers}.\\
Gerar um série temporal, com período de 7 dias, ao longo dos nossos 10
encontros.\\
Tratar eventuais NA's.\\
\emph{Descrever} a variação do IMC médio da turma ao longo do curso,
dado que será feito um apelo geral para aqueles acima da média para
tentarem reduzi-lo nas próximas 10 semanas.\\
Fazer uma regressão linear do IMC médio em função do tempo analisando se
ele sofreu alguma variação estatisticamente significativa.\\
\emph{Inferir} qual resultado seria alcançado se o curso durasse 20
semanas.

\section{Objetos}\label{objetos}

\subsection{Vetores}\label{vetores}

Conjunto de elementos do mesmo tipo (logical, numeric, integer, double,
character)

\begin{enumerate}
\def\labelenumi{\arabic{enumi}.}
\tightlist
\item
  A forma mais simples de se criar um vetor é usar a função de
  concatenação \texttt{c()}.
\end{enumerate}

\begin{Shaded}
\begin{Highlighting}[]
\NormalTok{value.num  =}\StringTok{ }\KeywordTok{c}\NormalTok{(}\DecValTok{3}\NormalTok{,}\DecValTok{4}\NormalTok{,}\DecValTok{2}\NormalTok{,}\DecValTok{6}\NormalTok{,}\DecValTok{20}\NormalTok{)}

\NormalTok{value.num}
\end{Highlighting}
\end{Shaded}

\begin{verbatim}
## [1]  3  4  2  6 20
\end{verbatim}

\begin{Shaded}
\begin{Highlighting}[]
\NormalTok{value.char =}\StringTok{ }\KeywordTok{c}\NormalTok{(}\StringTok{"koala"}\NormalTok{, }\StringTok{"kangaroo"}\NormalTok{)}

\NormalTok{value.char}
\end{Highlighting}
\end{Shaded}

\begin{verbatim}
## [1] "koala"    "kangaroo"
\end{verbatim}

\begin{Shaded}
\begin{Highlighting}[]
\NormalTok{value.logical =}\StringTok{ }\KeywordTok{c}\NormalTok{(}\OtherTok{FALSE}\NormalTok{, }\OtherTok{FALSE}\NormalTok{, }\OtherTok{TRUE}\NormalTok{, }\OtherTok{TRUE}\NormalTok{)}

\NormalTok{value.logical}
\end{Highlighting}
\end{Shaded}

\begin{verbatim}
## [1] FALSE FALSE  TRUE  TRUE
\end{verbatim}

\begin{enumerate}
\def\labelenumi{\arabic{enumi}.}
\setcounter{enumi}{1}
\tightlist
\item
  Segunda maneira de criar vetor no R: usando a função \texttt{scan}
\end{enumerate}

\begin{Shaded}
\begin{Highlighting}[]
\NormalTok{values =}\StringTok{ }\KeywordTok{scan}\NormalTok{(}\DataTypeTok{text=}\StringTok{"}
\StringTok{2}
\StringTok{3}
\StringTok{4}
\StringTok{5"}
\NormalTok{)}

\NormalTok{values}
\end{Highlighting}
\end{Shaded}

\begin{verbatim}
## [1] 2 3 4 5
\end{verbatim}

\begin{enumerate}
\def\labelenumi{\arabic{enumi}.}
\setcounter{enumi}{2}
\tightlist
\item
  Outra opção usando comando \texttt{rep}
\end{enumerate}

\begin{Shaded}
\begin{Highlighting}[]
\KeywordTok{rep}\NormalTok{(}\DecValTok{1}\NormalTok{,}\DecValTok{5}\NormalTok{)}
\end{Highlighting}
\end{Shaded}

\begin{verbatim}
## [1] 1 1 1 1 1
\end{verbatim}

\begin{Shaded}
\begin{Highlighting}[]
\KeywordTok{rep}\NormalTok{(}\KeywordTok{c}\NormalTok{(}\DecValTok{1}\NormalTok{,}\DecValTok{2}\NormalTok{),}\DecValTok{3}\NormalTok{)}
\end{Highlighting}
\end{Shaded}

\begin{verbatim}
## [1] 1 2 1 2 1 2
\end{verbatim}

\begin{Shaded}
\begin{Highlighting}[]
\KeywordTok{rep}\NormalTok{(}\KeywordTok{c}\NormalTok{(}\DecValTok{1}\NormalTok{,}\DecValTok{6}\NormalTok{),}\DataTypeTok{each=}\DecValTok{3}\NormalTok{)}
\end{Highlighting}
\end{Shaded}

\begin{verbatim}
## [1] 1 1 1 6 6 6
\end{verbatim}

\begin{Shaded}
\begin{Highlighting}[]
\KeywordTok{rep}\NormalTok{(}\KeywordTok{c}\NormalTok{(}\DecValTok{1}\NormalTok{,}\DecValTok{6}\NormalTok{),}\KeywordTok{c}\NormalTok{(}\DecValTok{3}\NormalTok{,}\DecValTok{5}\NormalTok{))}
\end{Highlighting}
\end{Shaded}

\begin{verbatim}
## [1] 1 1 1 6 6 6 6 6
\end{verbatim}

\begin{enumerate}
\def\labelenumi{\arabic{enumi}.}
\setcounter{enumi}{3}
\tightlist
\item
  Outra opção usando comando \texttt{seq}
\end{enumerate}

\begin{Shaded}
\begin{Highlighting}[]
\KeywordTok{seq}\NormalTok{(}\DataTypeTok{from=}\DecValTok{1}\NormalTok{,}\DataTypeTok{to=}\DecValTok{5}\NormalTok{)}
\end{Highlighting}
\end{Shaded}

\begin{verbatim}
## [1] 1 2 3 4 5
\end{verbatim}

\begin{Shaded}
\begin{Highlighting}[]
\KeywordTok{seq}\NormalTok{(}\DataTypeTok{from=}\DecValTok{1}\NormalTok{, }\DataTypeTok{to=}\DecValTok{5}\NormalTok{, }\DataTypeTok{by=}\FloatTok{0.1}\NormalTok{)}
\end{Highlighting}
\end{Shaded}

\begin{verbatim}
##  [1] 1.0 1.1 1.2 1.3 1.4 1.5 1.6 1.7 1.8 1.9 2.0 2.1 2.2 2.3 2.4 2.5 2.6
## [18] 2.7 2.8 2.9 3.0 3.1 3.2 3.3 3.4 3.5 3.6 3.7 3.8 3.9 4.0 4.1 4.2 4.3
## [35] 4.4 4.5 4.6 4.7 4.8 4.9 5.0
\end{verbatim}

\begin{Shaded}
\begin{Highlighting}[]
\KeywordTok{seq}\NormalTok{(}\DataTypeTok{from=}\DecValTok{1}\NormalTok{, }\DataTypeTok{to=}\DecValTok{5}\NormalTok{, }\DataTypeTok{length=}\DecValTok{10}\NormalTok{)}
\end{Highlighting}
\end{Shaded}

\begin{verbatim}
##  [1] 1.000000 1.444444 1.888889 2.333333 2.777778 3.222222 3.666667
##  [8] 4.111111 4.555556 5.000000
\end{verbatim}

\begin{Shaded}
\begin{Highlighting}[]
\KeywordTok{rep}\NormalTok{(}\KeywordTok{seq}\NormalTok{(}\DataTypeTok{from=}\DecValTok{1}\NormalTok{, }\DataTypeTok{to=}\DecValTok{5}\NormalTok{, }\DataTypeTok{length=}\DecValTok{10}\NormalTok{),}\DataTypeTok{each=}\DecValTok{3}\NormalTok{)}
\end{Highlighting}
\end{Shaded}

\begin{verbatim}
##  [1] 1.000000 1.000000 1.000000 1.444444 1.444444 1.444444 1.888889
##  [8] 1.888889 1.888889 2.333333 2.333333 2.333333 2.777778 2.777778
## [15] 2.777778 3.222222 3.222222 3.222222 3.666667 3.666667 3.666667
## [22] 4.111111 4.111111 4.111111 4.555556 4.555556 4.555556 5.000000
## [29] 5.000000 5.000000
\end{verbatim}

\begin{enumerate}
\def\labelenumi{\arabic{enumi}.}
\setcounter{enumi}{4}
\tightlist
\item
  Outra opção usando comando \texttt{:}
\end{enumerate}

\begin{Shaded}
\begin{Highlighting}[]
\DecValTok{1}\OperatorTok{:}\DecValTok{5}
\end{Highlighting}
\end{Shaded}

\begin{verbatim}
## [1] 1 2 3 4 5
\end{verbatim}

\begin{Shaded}
\begin{Highlighting}[]
\KeywordTok{c}\NormalTok{(}\DecValTok{1}\OperatorTok{:}\DecValTok{5}\NormalTok{,}\DecValTok{10}\NormalTok{)}
\end{Highlighting}
\end{Shaded}

\begin{verbatim}
## [1]  1  2  3  4  5 10
\end{verbatim}

\subsection{Operações com Vetores}\label{operacoes-com-vetores}

\begin{Shaded}
\begin{Highlighting}[]
\NormalTok{x =}\StringTok{ }\DecValTok{1}\OperatorTok{:}\DecValTok{4}

\NormalTok{y =}\StringTok{ }\DecValTok{5}\OperatorTok{:}\DecValTok{8}

\NormalTok{x }\OperatorTok{+}\StringTok{ }\NormalTok{y}
\end{Highlighting}
\end{Shaded}

\begin{verbatim}
## [1]  6  8 10 12
\end{verbatim}

\begin{Shaded}
\begin{Highlighting}[]
\DecValTok{2}\OperatorTok{*}\NormalTok{x }\OperatorTok{+}\DecValTok{1}
\end{Highlighting}
\end{Shaded}

\begin{verbatim}
## [1] 3 5 7 9
\end{verbatim}

\begin{Shaded}
\begin{Highlighting}[]
\NormalTok{x }\OperatorTok{*}\StringTok{ }\NormalTok{y}
\end{Highlighting}
\end{Shaded}

\begin{verbatim}
## [1]  5 12 21 32
\end{verbatim}

\begin{Shaded}
\begin{Highlighting}[]
\NormalTok{x }\OperatorTok{/}\StringTok{ }\NormalTok{y}
\end{Highlighting}
\end{Shaded}

\begin{verbatim}
## [1] 0.2000000 0.3333333 0.4285714 0.5000000
\end{verbatim}

\begin{Shaded}
\begin{Highlighting}[]
\KeywordTok{log}\NormalTok{(x)}
\end{Highlighting}
\end{Shaded}

\begin{verbatim}
## [1] 0.0000000 0.6931472 1.0986123 1.3862944
\end{verbatim}

\begin{Shaded}
\begin{Highlighting}[]
\KeywordTok{log}\NormalTok{(x,}\DecValTok{10}\NormalTok{)}
\end{Highlighting}
\end{Shaded}

\begin{verbatim}
## [1] 0.0000000 0.3010300 0.4771213 0.6020600
\end{verbatim}

\begin{Shaded}
\begin{Highlighting}[]
\KeywordTok{sum}\NormalTok{(x)}
\end{Highlighting}
\end{Shaded}

\begin{verbatim}
## [1] 10
\end{verbatim}

\begin{Shaded}
\begin{Highlighting}[]
\KeywordTok{mean}\NormalTok{(x)}
\end{Highlighting}
\end{Shaded}

\begin{verbatim}
## [1] 2.5
\end{verbatim}

\begin{Shaded}
\begin{Highlighting}[]
\KeywordTok{prod}\NormalTok{(x)}
\end{Highlighting}
\end{Shaded}

\begin{verbatim}
## [1] 24
\end{verbatim}

\begin{Shaded}
\begin{Highlighting}[]
\KeywordTok{var}\NormalTok{(x)}
\end{Highlighting}
\end{Shaded}

\begin{verbatim}
## [1] 1.666667
\end{verbatim}

\begin{Shaded}
\begin{Highlighting}[]
\CommentTok{# O que é um vetor do tipo factor}
\CommentTok{# usado para variáveis categóricas}
\CommentTok{# Que apresenta vávios Levels (níveis)}
\CommentTok{# Comumente cada nível recebe um nome gerando um conjunto denominado Labels}

\CommentTok{# Exemplo: No nosso estudo de caso seria interessante separar os dados amostrado segundo o sexo biológico de cada obsevração, porque biologicamente seres humanos XY tam estatura média superioe às dos XX. Logo também o IMC deve ser tratado para cada subgrupo separadamente.}

\NormalTok{s<-}\KeywordTok{c}\NormalTok{(}\StringTok{"m"}\NormalTok{,}\StringTok{"m"}\NormalTok{,}\StringTok{"m"}\NormalTok{,}\StringTok{"m"}\NormalTok{,}\StringTok{"f"}\NormalTok{,}\StringTok{"m"}\NormalTok{,}\StringTok{"m"}\NormalTok{,}\StringTok{"f"}\NormalTok{,}\StringTok{"f"}\NormalTok{,}\StringTok{"m"}\NormalTok{,}\StringTok{"f"}\NormalTok{,}\StringTok{"m"}\NormalTok{,}\StringTok{"m"}\NormalTok{)}
\NormalTok{s }\CommentTok{# um <vctr> do tipo <chr>}
\end{Highlighting}
\end{Shaded}

\begin{verbatim}
##  [1] "m" "m" "m" "m" "f" "m" "m" "f" "f" "m" "f" "m" "m"
\end{verbatim}

\begin{Shaded}
\begin{Highlighting}[]
\KeywordTok{mode}\NormalTok{(s)}
\end{Highlighting}
\end{Shaded}

\begin{verbatim}
## [1] "character"
\end{verbatim}

\begin{Shaded}
\begin{Highlighting}[]
\KeywordTok{class}\NormalTok{(s)}
\end{Highlighting}
\end{Shaded}

\begin{verbatim}
## [1] "character"
\end{verbatim}

\begin{Shaded}
\begin{Highlighting}[]
\KeywordTok{length}\NormalTok{(s)}
\end{Highlighting}
\end{Shaded}

\begin{verbatim}
## [1] 13
\end{verbatim}

\begin{Shaded}
\begin{Highlighting}[]
\KeywordTok{summary}\NormalTok{(s)}
\end{Highlighting}
\end{Shaded}

\begin{verbatim}
##    Length     Class      Mode 
##        13 character character
\end{verbatim}

\begin{Shaded}
\begin{Highlighting}[]
\KeywordTok{str}\NormalTok{(s)}
\end{Highlighting}
\end{Shaded}

\begin{verbatim}
##  chr [1:13] "m" "m" "m" "m" "f" "m" "m" "f" "f" "m" "f" "m" "m"
\end{verbatim}

\begin{Shaded}
\begin{Highlighting}[]
\KeywordTok{dput}\NormalTok{(s)}
\end{Highlighting}
\end{Shaded}

\begin{verbatim}
## c("m", "m", "m", "m", "f", "m", "m", "f", "f", "m", "f", "m", 
## "m")
\end{verbatim}

\begin{Shaded}
\begin{Highlighting}[]
\CommentTok{# Transformando numa variável factor <fctr>}
\NormalTok{s<-}\KeywordTok{as.factor}\NormalTok{(s) }\CommentTok{# Destroi <chr> e recria o vetor s como um <fctr>}
\NormalTok{s}
\end{Highlighting}
\end{Shaded}

\begin{verbatim}
##  [1] m m m m f m m f f m f m m
## Levels: f m
\end{verbatim}

\begin{Shaded}
\begin{Highlighting}[]
\KeywordTok{mode}\NormalTok{(s) }\CommentTok{# é um vetor do tipo <numeric>}
\end{Highlighting}
\end{Shaded}

\begin{verbatim}
## [1] "numeric"
\end{verbatim}

\begin{Shaded}
\begin{Highlighting}[]
\KeywordTok{class}\NormalTok{(s) }\CommentTok{# é um factor <fctr>, que é um caso especial de <numeric> indexado a Labels}
\end{Highlighting}
\end{Shaded}

\begin{verbatim}
## [1] "factor"
\end{verbatim}

\begin{Shaded}
\begin{Highlighting}[]
\KeywordTok{length}\NormalTok{(s)}
\end{Highlighting}
\end{Shaded}

\begin{verbatim}
## [1] 13
\end{verbatim}

\begin{Shaded}
\begin{Highlighting}[]
\KeywordTok{summary}\NormalTok{(s)}
\end{Highlighting}
\end{Shaded}

\begin{verbatim}
## f m 
## 4 9
\end{verbatim}

\begin{Shaded}
\begin{Highlighting}[]
\KeywordTok{str}\NormalTok{(s) }\CommentTok{# investigando a structure da variável s do tipo <fctr>}
\end{Highlighting}
\end{Shaded}

\begin{verbatim}
##  Factor w/ 2 levels "f","m": 2 2 2 2 1 2 2 1 1 2 ...
\end{verbatim}

\begin{Shaded}
\begin{Highlighting}[]
\KeywordTok{dput}\NormalTok{(s)}
\end{Highlighting}
\end{Shaded}

\begin{verbatim}
## structure(c(2L, 2L, 2L, 2L, 1L, 2L, 2L, 1L, 1L, 2L, 1L, 2L, 2L
## ), .Label = c("f", "m"), class = "factor")
\end{verbatim}

\begin{Shaded}
\begin{Highlighting}[]
\KeywordTok{table}\NormalTok{(s) }\CommentTok{# retorna um vetor tipo <fctr> com a frequência de cada um dos níveis (Levels) ou categorias do vetor que é repassado como parâmetro do argumento da função table()}
\end{Highlighting}
\end{Shaded}

\begin{verbatim}
## s
## f m 
## 4 9
\end{verbatim}

\begin{Shaded}
\begin{Highlighting}[]
\CommentTok{# Essa mesma função é usada para retornar tabulações cruzadas (cross table) de duas variáveis categóricas}
\KeywordTok{max}\NormalTok{(h, }\DataTypeTok{na.rm =} \OtherTok{TRUE}\NormalTok{)}
\end{Highlighting}
\end{Shaded}

\begin{verbatim}
## [1] 1.85
\end{verbatim}

\begin{Shaded}
\begin{Highlighting}[]
\NormalTok{hcat <-}\StringTok{ }\KeywordTok{cut}\NormalTok{(h,}\KeywordTok{c}\NormalTok{(}\DecValTok{0}\NormalTok{,}\FloatTok{1.6}\NormalTok{,}\FloatTok{1.7}\NormalTok{,}\FloatTok{2.0}\NormalTok{),}\DataTypeTok{labels =} \KeywordTok{c}\NormalTok{(}\StringTok{"Baixo"}\NormalTok{,}\StringTok{"Médio"}\NormalTok{,}\StringTok{"Alto"}\NormalTok{))}
\CommentTok{# função cat() Convert Numeric to Factor}
\KeywordTok{str}\NormalTok{(hcat)}
\end{Highlighting}
\end{Shaded}

\begin{verbatim}
##  Factor w/ 3 levels "Baixo","Médio",..: 3 2 3 3 NA 3 1 NA 1 2 ...
\end{verbatim}

\begin{Shaded}
\begin{Highlighting}[]
\KeywordTok{dput}\NormalTok{(hcat)}
\end{Highlighting}
\end{Shaded}

\begin{verbatim}
## structure(c(3L, 2L, 3L, 3L, NA, 3L, 1L, NA, 1L, 2L, 2L, 2L, 3L
## ), .Label = c("Baixo", "Médio", "Alto"), class = "factor")
\end{verbatim}

\begin{Shaded}
\begin{Highlighting}[]
\KeywordTok{table}\NormalTok{(hcat,s)}
\end{Highlighting}
\end{Shaded}

\begin{verbatim}
##        s
## hcat    f m
##   Baixo 1 1
##   Médio 1 3
##   Alto  0 5
\end{verbatim}

\begin{Shaded}
\begin{Highlighting}[]
\NormalTok{hm}
\end{Highlighting}
\end{Shaded}

\begin{verbatim}
## [1] 1.697273
\end{verbatim}

\begin{Shaded}
\begin{Highlighting}[]
\NormalTok{ct<-}\KeywordTok{table}\NormalTok{(hcat,s)}
\KeywordTok{prop.table}\NormalTok{(ct,}\DecValTok{1}\NormalTok{)}
\end{Highlighting}
\end{Shaded}

\begin{verbatim}
##        s
## hcat       f    m
##   Baixo 0.50 0.50
##   Médio 0.25 0.75
##   Alto  0.00 1.00
\end{verbatim}

\begin{Shaded}
\begin{Highlighting}[]
\KeywordTok{prop.table}\NormalTok{(ct,}\DecValTok{2}\NormalTok{)}
\end{Highlighting}
\end{Shaded}

\begin{verbatim}
##        s
## hcat            f         m
##   Baixo 0.5000000 0.1111111
##   Médio 0.5000000 0.3333333
##   Alto  0.0000000 0.5555556
\end{verbatim}

\begin{Shaded}
\begin{Highlighting}[]
\KeywordTok{prop.table}\NormalTok{(ct)}
\end{Highlighting}
\end{Shaded}

\begin{verbatim}
##        s
## hcat             f          m
##   Baixo 0.09090909 0.09090909
##   Médio 0.09090909 0.27272727
##   Alto  0.00000000 0.45454545
\end{verbatim}

\begin{Shaded}
\begin{Highlighting}[]
\DecValTok{100}\OperatorTok{*}\KeywordTok{prop.table}\NormalTok{(ct)}
\end{Highlighting}
\end{Shaded}

\begin{verbatim}
##        s
## hcat            f         m
##   Baixo  9.090909  9.090909
##   Médio  9.090909 27.272727
##   Alto   0.000000 45.454545
\end{verbatim}

\begin{Shaded}
\begin{Highlighting}[]
\CommentTok{# Analisando o resultados dessas cross tables p.u. vê-se que o IMC deve ser categorizado em feminino (XX) e masculino (XY)}

\CommentTok{# Calculando a altura média das observações s == f}
\NormalTok{s}\OperatorTok{==}\StringTok{"f"}
\end{Highlighting}
\end{Shaded}

\begin{verbatim}
##  [1] FALSE FALSE FALSE FALSE  TRUE FALSE FALSE  TRUE  TRUE FALSE  TRUE
## [12] FALSE FALSE
\end{verbatim}

\begin{Shaded}
\begin{Highlighting}[]
\NormalTok{h[s}\OperatorTok{==}\StringTok{"f"}\NormalTok{]}
\end{Highlighting}
\end{Shaded}

\begin{verbatim}
## [1]   NA   NA 1.55 1.63
\end{verbatim}

\begin{Shaded}
\begin{Highlighting}[]
\KeywordTok{mean}\NormalTok{(h[s}\OperatorTok{==}\StringTok{"f"}\NormalTok{], }\DataTypeTok{na.rm=}\OtherTok{TRUE}\NormalTok{) }\CommentTok{# é média da estatura do sexo feminino  = 1.59 m}
\end{Highlighting}
\end{Shaded}

\begin{verbatim}
## [1] 1.59
\end{verbatim}

\begin{Shaded}
\begin{Highlighting}[]
\KeywordTok{mean}\NormalTok{(h[s}\OperatorTok{==}\StringTok{"m"}\NormalTok{], }\DataTypeTok{na.rm=}\OtherTok{TRUE}\NormalTok{) }\CommentTok{# é média da estatura do sexo masculino = 1.72 m}
\end{Highlighting}
\end{Shaded}

\begin{verbatim}
## [1] 1.721111
\end{verbatim}

\begin{Shaded}
\begin{Highlighting}[]
\CommentTok{# Exibindo essa diferença graficamente}
\KeywordTok{boxplot}\NormalTok{(h}\OperatorTok{~}\NormalTok{s) }\CommentTok{# homens são, em média, mais alto que as mulheres}
\end{Highlighting}
\end{Shaded}

\includegraphics{PGE-aula3_files/figure-latex/unnamed-chunk-7-1.pdf}

\begin{Shaded}
\begin{Highlighting}[]
\KeywordTok{boxplot}\NormalTok{(p}\OperatorTok{~}\NormalTok{s) }\CommentTok{# homens são, em média, mais pesados que as mulheres}
\end{Highlighting}
\end{Shaded}

\includegraphics{PGE-aula3_files/figure-latex/unnamed-chunk-7-2.pdf}

\begin{Shaded}
\begin{Highlighting}[]
\CommentTok{# Esses gráficos corroboram uma Hipótese de estratificação f & m para analisar o IMC?????}
\CommentTok{# Duvidar é preciso.}
\CommentTok{# Transformar sua dúvida nums hipótese testável.}
\CommentTok{# E testar adequadamente a Hipótese **contra** as observações colhidas no campo.}

\KeywordTok{boxplot}\NormalTok{(IMC}\OperatorTok{~}\NormalTok{s)}
\end{Highlighting}
\end{Shaded}

\includegraphics{PGE-aula3_files/figure-latex/unnamed-chunk-7-3.pdf}

\subsection{Matriz}\label{matriz}

Conjunto de elementos dispostos em linhas e colunas, em que todos os
elementos são do mesmo tipo

\begin{Shaded}
\begin{Highlighting}[]
\NormalTok{mat.num  =}\StringTok{ }\KeywordTok{matrix}\NormalTok{(}\KeywordTok{c}\NormalTok{(}\DecValTok{1}\OperatorTok{:}\DecValTok{16}\NormalTok{),}\DecValTok{4}\NormalTok{,}\DecValTok{4}\NormalTok{)}

\NormalTok{mat.num}
\end{Highlighting}
\end{Shaded}

\begin{verbatim}
##      [,1] [,2] [,3] [,4]
## [1,]    1    5    9   13
## [2,]    2    6   10   14
## [3,]    3    7   11   15
## [4,]    4    8   12   16
\end{verbatim}

\begin{Shaded}
\begin{Highlighting}[]
\NormalTok{mat.char =}\StringTok{ }\KeywordTok{matrix}\NormalTok{(LETTERS[}\DecValTok{1}\OperatorTok{:}\DecValTok{4}\NormalTok{],}\DecValTok{2}\NormalTok{,}\DecValTok{2}\NormalTok{)}

\NormalTok{mat.char}
\end{Highlighting}
\end{Shaded}

\begin{verbatim}
##      [,1] [,2]
## [1,] "A"  "C" 
## [2,] "B"  "D"
\end{verbatim}

\section{Manipulando Matrizes}\label{manipulando-matrizes}

\begin{Shaded}
\begin{Highlighting}[]
\CommentTok{#Criando nomes para as linhas de uma matriz}

\KeywordTok{rownames}\NormalTok{(mat.num) =}\StringTok{ }\KeywordTok{c}\NormalTok{(}\StringTok{"Sao Paulo"}\NormalTok{, }\StringTok{"Americana"}\NormalTok{, }\StringTok{"Piracicaba"}\NormalTok{, }\StringTok{"Madson"}\NormalTok{ )}

\KeywordTok{colnames}\NormalTok{(mat.num) =}\StringTok{ }\DecValTok{1}\OperatorTok{:}\DecValTok{4}

\NormalTok{mat.num}
\end{Highlighting}
\end{Shaded}

\begin{verbatim}
##            1 2  3  4
## Sao Paulo  1 5  9 13
## Americana  2 6 10 14
## Piracicaba 3 7 11 15
## Madson     4 8 12 16
\end{verbatim}

\begin{Shaded}
\begin{Highlighting}[]
\CommentTok{#Multiplicação elemento a elemento}

\NormalTok{mat.num2 =}\StringTok{ }\KeywordTok{diag}\NormalTok{(}\KeywordTok{seq}\NormalTok{(}\DecValTok{10}\NormalTok{,}\DecValTok{40}\NormalTok{,}\DataTypeTok{by=}\DecValTok{10}\NormalTok{))}

\NormalTok{mat.num2}
\end{Highlighting}
\end{Shaded}

\begin{verbatim}
##      [,1] [,2] [,3] [,4]
## [1,]   10    0    0    0
## [2,]    0   20    0    0
## [3,]    0    0   30    0
## [4,]    0    0    0   40
\end{verbatim}

\begin{Shaded}
\begin{Highlighting}[]
\NormalTok{mat.num3 =}\StringTok{ }\NormalTok{mat.num }\OperatorTok{*}\StringTok{ }\NormalTok{mat.num2}

\NormalTok{mat.num3}
\end{Highlighting}
\end{Shaded}

\begin{verbatim}
##             1   2   3   4
## Sao Paulo  10   0   0   0
## Americana   0 120   0   0
## Piracicaba  0   0 330   0
## Madson      0   0   0 640
\end{verbatim}

\begin{Shaded}
\begin{Highlighting}[]
\CommentTok{#Multiplicação de Matrizes}

\NormalTok{iden =}\StringTok{ }\KeywordTok{diag}\NormalTok{(}\DecValTok{4}\NormalTok{)}

\NormalTok{iden}
\end{Highlighting}
\end{Shaded}

\begin{verbatim}
##      [,1] [,2] [,3] [,4]
## [1,]    1    0    0    0
## [2,]    0    1    0    0
## [3,]    0    0    1    0
## [4,]    0    0    0    1
\end{verbatim}

\begin{Shaded}
\begin{Highlighting}[]
\NormalTok{mat.num}\OperatorTok\NormalTok{iden}
\end{Highlighting}
\end{Shaded}

\begin{verbatim}
##            [,1] [,2] [,3] [,4]
## Sao Paulo     1    5    9   13
## Americana     2    6   10   14
## Piracicaba    3    7   11   15
## Madson        4    8   12   16
\end{verbatim}

\begin{Shaded}
\begin{Highlighting}[]
\CommentTok{#Acessando elementos das matrizes }

\CommentTok{#Um elemento}
\NormalTok{mat.num[}\DecValTok{1}\NormalTok{,}\DecValTok{1}\NormalTok{]}
\end{Highlighting}
\end{Shaded}

\begin{verbatim}
## [1] 1
\end{verbatim}

\begin{Shaded}
\begin{Highlighting}[]
\CommentTok{#Linhas }
\NormalTok{mat.num[}\DecValTok{1}\NormalTok{,]}
\end{Highlighting}
\end{Shaded}

\begin{verbatim}
##  1  2  3  4 
##  1  5  9 13
\end{verbatim}

\begin{Shaded}
\begin{Highlighting}[]
\CommentTok{#Colunas}
\NormalTok{mat.num[,}\DecValTok{3}\NormalTok{]}
\end{Highlighting}
\end{Shaded}

\begin{verbatim}
##  Sao Paulo  Americana Piracicaba     Madson 
##          9         10         11         12
\end{verbatim}

\begin{Shaded}
\begin{Highlighting}[]
\CommentTok{#Sub Matrizes}

\NormalTok{mat.num[}\KeywordTok{c}\NormalTok{(}\DecValTok{1}\NormalTok{,}\DecValTok{3}\NormalTok{,}\DecValTok{4}\NormalTok{), }\KeywordTok{c}\NormalTok{(}\DecValTok{2}\NormalTok{,}\DecValTok{1}\NormalTok{,}\DecValTok{4}\NormalTok{)]}
\end{Highlighting}
\end{Shaded}

\begin{verbatim}
##            2 1  4
## Sao Paulo  5 1 13
## Piracicaba 7 3 15
## Madson     8 4 16
\end{verbatim}

\begin{Shaded}
\begin{Highlighting}[]
\NormalTok{mat.num[}\KeywordTok{c}\NormalTok{(T,F,T,T), }\KeywordTok{c}\NormalTok{(T,T,F,T)]}
\end{Highlighting}
\end{Shaded}

\begin{verbatim}
##            1 2  4
## Sao Paulo  1 5 13
## Piracicaba 3 7 15
## Madson     4 8 16
\end{verbatim}

\begin{Shaded}
\begin{Highlighting}[]
\NormalTok{mat.num[}\OperatorTok{-}\KeywordTok{c}\NormalTok{(}\DecValTok{1}\NormalTok{,}\DecValTok{3}\NormalTok{,}\DecValTok{4}\NormalTok{), }\OperatorTok{-}\KeywordTok{c}\NormalTok{(}\DecValTok{2}\NormalTok{,}\DecValTok{1}\NormalTok{,}\DecValTok{4}\NormalTok{)]}
\end{Highlighting}
\end{Shaded}

\begin{verbatim}
## [1] 10
\end{verbatim}

\section{Data.frames}\label{data.frames}

São similares às matrizes no entanto permite que as colunas tenham
diferentes tipos

\begin{Shaded}
\begin{Highlighting}[]
\KeywordTok{data}\NormalTok{(iris)}

\NormalTok{iris}
\end{Highlighting}
\end{Shaded}

\begin{verbatim}
##     Sepal.Length Sepal.Width Petal.Length Petal.Width    Species
## 1            5.1         3.5          1.4         0.2     setosa
## 2            4.9         3.0          1.4         0.2     setosa
## 3            4.7         3.2          1.3         0.2     setosa
## 4            4.6         3.1          1.5         0.2     setosa
## 5            5.0         3.6          1.4         0.2     setosa
## 6            5.4         3.9          1.7         0.4     setosa
## 7            4.6         3.4          1.4         0.3     setosa
## 8            5.0         3.4          1.5         0.2     setosa
## 9            4.4         2.9          1.4         0.2     setosa
## 10           4.9         3.1          1.5         0.1     setosa
## 11           5.4         3.7          1.5         0.2     setosa
## 12           4.8         3.4          1.6         0.2     setosa
## 13           4.8         3.0          1.4         0.1     setosa
## 14           4.3         3.0          1.1         0.1     setosa
## 15           5.8         4.0          1.2         0.2     setosa
## 16           5.7         4.4          1.5         0.4     setosa
## 17           5.4         3.9          1.3         0.4     setosa
## 18           5.1         3.5          1.4         0.3     setosa
## 19           5.7         3.8          1.7         0.3     setosa
## 20           5.1         3.8          1.5         0.3     setosa
## 21           5.4         3.4          1.7         0.2     setosa
## 22           5.1         3.7          1.5         0.4     setosa
## 23           4.6         3.6          1.0         0.2     setosa
## 24           5.1         3.3          1.7         0.5     setosa
## 25           4.8         3.4          1.9         0.2     setosa
## 26           5.0         3.0          1.6         0.2     setosa
## 27           5.0         3.4          1.6         0.4     setosa
## 28           5.2         3.5          1.5         0.2     setosa
## 29           5.2         3.4          1.4         0.2     setosa
## 30           4.7         3.2          1.6         0.2     setosa
## 31           4.8         3.1          1.6         0.2     setosa
## 32           5.4         3.4          1.5         0.4     setosa
## 33           5.2         4.1          1.5         0.1     setosa
## 34           5.5         4.2          1.4         0.2     setosa
## 35           4.9         3.1          1.5         0.2     setosa
## 36           5.0         3.2          1.2         0.2     setosa
## 37           5.5         3.5          1.3         0.2     setosa
## 38           4.9         3.6          1.4         0.1     setosa
## 39           4.4         3.0          1.3         0.2     setosa
## 40           5.1         3.4          1.5         0.2     setosa
## 41           5.0         3.5          1.3         0.3     setosa
## 42           4.5         2.3          1.3         0.3     setosa
## 43           4.4         3.2          1.3         0.2     setosa
## 44           5.0         3.5          1.6         0.6     setosa
## 45           5.1         3.8          1.9         0.4     setosa
## 46           4.8         3.0          1.4         0.3     setosa
## 47           5.1         3.8          1.6         0.2     setosa
## 48           4.6         3.2          1.4         0.2     setosa
## 49           5.3         3.7          1.5         0.2     setosa
## 50           5.0         3.3          1.4         0.2     setosa
## 51           7.0         3.2          4.7         1.4 versicolor
## 52           6.4         3.2          4.5         1.5 versicolor
## 53           6.9         3.1          4.9         1.5 versicolor
## 54           5.5         2.3          4.0         1.3 versicolor
## 55           6.5         2.8          4.6         1.5 versicolor
## 56           5.7         2.8          4.5         1.3 versicolor
## 57           6.3         3.3          4.7         1.6 versicolor
## 58           4.9         2.4          3.3         1.0 versicolor
## 59           6.6         2.9          4.6         1.3 versicolor
## 60           5.2         2.7          3.9         1.4 versicolor
## 61           5.0         2.0          3.5         1.0 versicolor
## 62           5.9         3.0          4.2         1.5 versicolor
## 63           6.0         2.2          4.0         1.0 versicolor
## 64           6.1         2.9          4.7         1.4 versicolor
## 65           5.6         2.9          3.6         1.3 versicolor
## 66           6.7         3.1          4.4         1.4 versicolor
## 67           5.6         3.0          4.5         1.5 versicolor
## 68           5.8         2.7          4.1         1.0 versicolor
## 69           6.2         2.2          4.5         1.5 versicolor
## 70           5.6         2.5          3.9         1.1 versicolor
## 71           5.9         3.2          4.8         1.8 versicolor
## 72           6.1         2.8          4.0         1.3 versicolor
## 73           6.3         2.5          4.9         1.5 versicolor
## 74           6.1         2.8          4.7         1.2 versicolor
## 75           6.4         2.9          4.3         1.3 versicolor
## 76           6.6         3.0          4.4         1.4 versicolor
## 77           6.8         2.8          4.8         1.4 versicolor
## 78           6.7         3.0          5.0         1.7 versicolor
## 79           6.0         2.9          4.5         1.5 versicolor
## 80           5.7         2.6          3.5         1.0 versicolor
## 81           5.5         2.4          3.8         1.1 versicolor
## 82           5.5         2.4          3.7         1.0 versicolor
## 83           5.8         2.7          3.9         1.2 versicolor
## 84           6.0         2.7          5.1         1.6 versicolor
## 85           5.4         3.0          4.5         1.5 versicolor
## 86           6.0         3.4          4.5         1.6 versicolor
## 87           6.7         3.1          4.7         1.5 versicolor
## 88           6.3         2.3          4.4         1.3 versicolor
## 89           5.6         3.0          4.1         1.3 versicolor
## 90           5.5         2.5          4.0         1.3 versicolor
## 91           5.5         2.6          4.4         1.2 versicolor
## 92           6.1         3.0          4.6         1.4 versicolor
## 93           5.8         2.6          4.0         1.2 versicolor
## 94           5.0         2.3          3.3         1.0 versicolor
## 95           5.6         2.7          4.2         1.3 versicolor
## 96           5.7         3.0          4.2         1.2 versicolor
## 97           5.7         2.9          4.2         1.3 versicolor
## 98           6.2         2.9          4.3         1.3 versicolor
## 99           5.1         2.5          3.0         1.1 versicolor
## 100          5.7         2.8          4.1         1.3 versicolor
## 101          6.3         3.3          6.0         2.5  virginica
## 102          5.8         2.7          5.1         1.9  virginica
## 103          7.1         3.0          5.9         2.1  virginica
## 104          6.3         2.9          5.6         1.8  virginica
## 105          6.5         3.0          5.8         2.2  virginica
## 106          7.6         3.0          6.6         2.1  virginica
## 107          4.9         2.5          4.5         1.7  virginica
## 108          7.3         2.9          6.3         1.8  virginica
## 109          6.7         2.5          5.8         1.8  virginica
## 110          7.2         3.6          6.1         2.5  virginica
## 111          6.5         3.2          5.1         2.0  virginica
## 112          6.4         2.7          5.3         1.9  virginica
## 113          6.8         3.0          5.5         2.1  virginica
## 114          5.7         2.5          5.0         2.0  virginica
## 115          5.8         2.8          5.1         2.4  virginica
## 116          6.4         3.2          5.3         2.3  virginica
## 117          6.5         3.0          5.5         1.8  virginica
## 118          7.7         3.8          6.7         2.2  virginica
## 119          7.7         2.6          6.9         2.3  virginica
## 120          6.0         2.2          5.0         1.5  virginica
## 121          6.9         3.2          5.7         2.3  virginica
## 122          5.6         2.8          4.9         2.0  virginica
## 123          7.7         2.8          6.7         2.0  virginica
## 124          6.3         2.7          4.9         1.8  virginica
## 125          6.7         3.3          5.7         2.1  virginica
## 126          7.2         3.2          6.0         1.8  virginica
## 127          6.2         2.8          4.8         1.8  virginica
## 128          6.1         3.0          4.9         1.8  virginica
## 129          6.4         2.8          5.6         2.1  virginica
## 130          7.2         3.0          5.8         1.6  virginica
## 131          7.4         2.8          6.1         1.9  virginica
## 132          7.9         3.8          6.4         2.0  virginica
## 133          6.4         2.8          5.6         2.2  virginica
## 134          6.3         2.8          5.1         1.5  virginica
## 135          6.1         2.6          5.6         1.4  virginica
## 136          7.7         3.0          6.1         2.3  virginica
## 137          6.3         3.4          5.6         2.4  virginica
## 138          6.4         3.1          5.5         1.8  virginica
## 139          6.0         3.0          4.8         1.8  virginica
## 140          6.9         3.1          5.4         2.1  virginica
## 141          6.7         3.1          5.6         2.4  virginica
## 142          6.9         3.1          5.1         2.3  virginica
## 143          5.8         2.7          5.1         1.9  virginica
## 144          6.8         3.2          5.9         2.3  virginica
## 145          6.7         3.3          5.7         2.5  virginica
## 146          6.7         3.0          5.2         2.3  virginica
## 147          6.3         2.5          5.0         1.9  virginica
## 148          6.5         3.0          5.2         2.0  virginica
## 149          6.2         3.4          5.4         2.3  virginica
## 150          5.9         3.0          5.1         1.8  virginica
\end{verbatim}

\begin{Shaded}
\begin{Highlighting}[]
\NormalTok{iris}\OperatorTok{$}\NormalTok{Sepal.Length}
\end{Highlighting}
\end{Shaded}

\begin{verbatim}
##   [1] 5.1 4.9 4.7 4.6 5.0 5.4 4.6 5.0 4.4 4.9 5.4 4.8 4.8 4.3 5.8 5.7 5.4
##  [18] 5.1 5.7 5.1 5.4 5.1 4.6 5.1 4.8 5.0 5.0 5.2 5.2 4.7 4.8 5.4 5.2 5.5
##  [35] 4.9 5.0 5.5 4.9 4.4 5.1 5.0 4.5 4.4 5.0 5.1 4.8 5.1 4.6 5.3 5.0 7.0
##  [52] 6.4 6.9 5.5 6.5 5.7 6.3 4.9 6.6 5.2 5.0 5.9 6.0 6.1 5.6 6.7 5.6 5.8
##  [69] 6.2 5.6 5.9 6.1 6.3 6.1 6.4 6.6 6.8 6.7 6.0 5.7 5.5 5.5 5.8 6.0 5.4
##  [86] 6.0 6.7 6.3 5.6 5.5 5.5 6.1 5.8 5.0 5.6 5.7 5.7 6.2 5.1 5.7 6.3 5.8
## [103] 7.1 6.3 6.5 7.6 4.9 7.3 6.7 7.2 6.5 6.4 6.8 5.7 5.8 6.4 6.5 7.7 7.7
## [120] 6.0 6.9 5.6 7.7 6.3 6.7 7.2 6.2 6.1 6.4 7.2 7.4 7.9 6.4 6.3 6.1 7.7
## [137] 6.3 6.4 6.0 6.9 6.7 6.9 5.8 6.8 6.7 6.7 6.3 6.5 6.2 5.9
\end{verbatim}

\begin{Shaded}
\begin{Highlighting}[]
\NormalTok{iris}\OperatorTok{$}\NormalTok{Renato =}\StringTok{ }\OtherTok{TRUE}

\NormalTok{iris}
\end{Highlighting}
\end{Shaded}

\begin{verbatim}
##     Sepal.Length Sepal.Width Petal.Length Petal.Width    Species Renato
## 1            5.1         3.5          1.4         0.2     setosa   TRUE
## 2            4.9         3.0          1.4         0.2     setosa   TRUE
## 3            4.7         3.2          1.3         0.2     setosa   TRUE
## 4            4.6         3.1          1.5         0.2     setosa   TRUE
## 5            5.0         3.6          1.4         0.2     setosa   TRUE
## 6            5.4         3.9          1.7         0.4     setosa   TRUE
## 7            4.6         3.4          1.4         0.3     setosa   TRUE
## 8            5.0         3.4          1.5         0.2     setosa   TRUE
## 9            4.4         2.9          1.4         0.2     setosa   TRUE
## 10           4.9         3.1          1.5         0.1     setosa   TRUE
## 11           5.4         3.7          1.5         0.2     setosa   TRUE
## 12           4.8         3.4          1.6         0.2     setosa   TRUE
## 13           4.8         3.0          1.4         0.1     setosa   TRUE
## 14           4.3         3.0          1.1         0.1     setosa   TRUE
## 15           5.8         4.0          1.2         0.2     setosa   TRUE
## 16           5.7         4.4          1.5         0.4     setosa   TRUE
## 17           5.4         3.9          1.3         0.4     setosa   TRUE
## 18           5.1         3.5          1.4         0.3     setosa   TRUE
## 19           5.7         3.8          1.7         0.3     setosa   TRUE
## 20           5.1         3.8          1.5         0.3     setosa   TRUE
## 21           5.4         3.4          1.7         0.2     setosa   TRUE
## 22           5.1         3.7          1.5         0.4     setosa   TRUE
## 23           4.6         3.6          1.0         0.2     setosa   TRUE
## 24           5.1         3.3          1.7         0.5     setosa   TRUE
## 25           4.8         3.4          1.9         0.2     setosa   TRUE
## 26           5.0         3.0          1.6         0.2     setosa   TRUE
## 27           5.0         3.4          1.6         0.4     setosa   TRUE
## 28           5.2         3.5          1.5         0.2     setosa   TRUE
## 29           5.2         3.4          1.4         0.2     setosa   TRUE
## 30           4.7         3.2          1.6         0.2     setosa   TRUE
## 31           4.8         3.1          1.6         0.2     setosa   TRUE
## 32           5.4         3.4          1.5         0.4     setosa   TRUE
## 33           5.2         4.1          1.5         0.1     setosa   TRUE
## 34           5.5         4.2          1.4         0.2     setosa   TRUE
## 35           4.9         3.1          1.5         0.2     setosa   TRUE
## 36           5.0         3.2          1.2         0.2     setosa   TRUE
## 37           5.5         3.5          1.3         0.2     setosa   TRUE
## 38           4.9         3.6          1.4         0.1     setosa   TRUE
## 39           4.4         3.0          1.3         0.2     setosa   TRUE
## 40           5.1         3.4          1.5         0.2     setosa   TRUE
## 41           5.0         3.5          1.3         0.3     setosa   TRUE
## 42           4.5         2.3          1.3         0.3     setosa   TRUE
## 43           4.4         3.2          1.3         0.2     setosa   TRUE
## 44           5.0         3.5          1.6         0.6     setosa   TRUE
## 45           5.1         3.8          1.9         0.4     setosa   TRUE
## 46           4.8         3.0          1.4         0.3     setosa   TRUE
## 47           5.1         3.8          1.6         0.2     setosa   TRUE
## 48           4.6         3.2          1.4         0.2     setosa   TRUE
## 49           5.3         3.7          1.5         0.2     setosa   TRUE
## 50           5.0         3.3          1.4         0.2     setosa   TRUE
## 51           7.0         3.2          4.7         1.4 versicolor   TRUE
## 52           6.4         3.2          4.5         1.5 versicolor   TRUE
## 53           6.9         3.1          4.9         1.5 versicolor   TRUE
## 54           5.5         2.3          4.0         1.3 versicolor   TRUE
## 55           6.5         2.8          4.6         1.5 versicolor   TRUE
## 56           5.7         2.8          4.5         1.3 versicolor   TRUE
## 57           6.3         3.3          4.7         1.6 versicolor   TRUE
## 58           4.9         2.4          3.3         1.0 versicolor   TRUE
## 59           6.6         2.9          4.6         1.3 versicolor   TRUE
## 60           5.2         2.7          3.9         1.4 versicolor   TRUE
## 61           5.0         2.0          3.5         1.0 versicolor   TRUE
## 62           5.9         3.0          4.2         1.5 versicolor   TRUE
## 63           6.0         2.2          4.0         1.0 versicolor   TRUE
## 64           6.1         2.9          4.7         1.4 versicolor   TRUE
## 65           5.6         2.9          3.6         1.3 versicolor   TRUE
## 66           6.7         3.1          4.4         1.4 versicolor   TRUE
## 67           5.6         3.0          4.5         1.5 versicolor   TRUE
## 68           5.8         2.7          4.1         1.0 versicolor   TRUE
## 69           6.2         2.2          4.5         1.5 versicolor   TRUE
## 70           5.6         2.5          3.9         1.1 versicolor   TRUE
## 71           5.9         3.2          4.8         1.8 versicolor   TRUE
## 72           6.1         2.8          4.0         1.3 versicolor   TRUE
## 73           6.3         2.5          4.9         1.5 versicolor   TRUE
## 74           6.1         2.8          4.7         1.2 versicolor   TRUE
## 75           6.4         2.9          4.3         1.3 versicolor   TRUE
## 76           6.6         3.0          4.4         1.4 versicolor   TRUE
## 77           6.8         2.8          4.8         1.4 versicolor   TRUE
## 78           6.7         3.0          5.0         1.7 versicolor   TRUE
## 79           6.0         2.9          4.5         1.5 versicolor   TRUE
## 80           5.7         2.6          3.5         1.0 versicolor   TRUE
## 81           5.5         2.4          3.8         1.1 versicolor   TRUE
## 82           5.5         2.4          3.7         1.0 versicolor   TRUE
## 83           5.8         2.7          3.9         1.2 versicolor   TRUE
## 84           6.0         2.7          5.1         1.6 versicolor   TRUE
## 85           5.4         3.0          4.5         1.5 versicolor   TRUE
## 86           6.0         3.4          4.5         1.6 versicolor   TRUE
## 87           6.7         3.1          4.7         1.5 versicolor   TRUE
## 88           6.3         2.3          4.4         1.3 versicolor   TRUE
## 89           5.6         3.0          4.1         1.3 versicolor   TRUE
## 90           5.5         2.5          4.0         1.3 versicolor   TRUE
## 91           5.5         2.6          4.4         1.2 versicolor   TRUE
## 92           6.1         3.0          4.6         1.4 versicolor   TRUE
## 93           5.8         2.6          4.0         1.2 versicolor   TRUE
## 94           5.0         2.3          3.3         1.0 versicolor   TRUE
## 95           5.6         2.7          4.2         1.3 versicolor   TRUE
## 96           5.7         3.0          4.2         1.2 versicolor   TRUE
## 97           5.7         2.9          4.2         1.3 versicolor   TRUE
## 98           6.2         2.9          4.3         1.3 versicolor   TRUE
## 99           5.1         2.5          3.0         1.1 versicolor   TRUE
## 100          5.7         2.8          4.1         1.3 versicolor   TRUE
## 101          6.3         3.3          6.0         2.5  virginica   TRUE
## 102          5.8         2.7          5.1         1.9  virginica   TRUE
## 103          7.1         3.0          5.9         2.1  virginica   TRUE
## 104          6.3         2.9          5.6         1.8  virginica   TRUE
## 105          6.5         3.0          5.8         2.2  virginica   TRUE
## 106          7.6         3.0          6.6         2.1  virginica   TRUE
## 107          4.9         2.5          4.5         1.7  virginica   TRUE
## 108          7.3         2.9          6.3         1.8  virginica   TRUE
## 109          6.7         2.5          5.8         1.8  virginica   TRUE
## 110          7.2         3.6          6.1         2.5  virginica   TRUE
## 111          6.5         3.2          5.1         2.0  virginica   TRUE
## 112          6.4         2.7          5.3         1.9  virginica   TRUE
## 113          6.8         3.0          5.5         2.1  virginica   TRUE
## 114          5.7         2.5          5.0         2.0  virginica   TRUE
## 115          5.8         2.8          5.1         2.4  virginica   TRUE
## 116          6.4         3.2          5.3         2.3  virginica   TRUE
## 117          6.5         3.0          5.5         1.8  virginica   TRUE
## 118          7.7         3.8          6.7         2.2  virginica   TRUE
## 119          7.7         2.6          6.9         2.3  virginica   TRUE
## 120          6.0         2.2          5.0         1.5  virginica   TRUE
## 121          6.9         3.2          5.7         2.3  virginica   TRUE
## 122          5.6         2.8          4.9         2.0  virginica   TRUE
## 123          7.7         2.8          6.7         2.0  virginica   TRUE
## 124          6.3         2.7          4.9         1.8  virginica   TRUE
## 125          6.7         3.3          5.7         2.1  virginica   TRUE
## 126          7.2         3.2          6.0         1.8  virginica   TRUE
## 127          6.2         2.8          4.8         1.8  virginica   TRUE
## 128          6.1         3.0          4.9         1.8  virginica   TRUE
## 129          6.4         2.8          5.6         2.1  virginica   TRUE
## 130          7.2         3.0          5.8         1.6  virginica   TRUE
## 131          7.4         2.8          6.1         1.9  virginica   TRUE
## 132          7.9         3.8          6.4         2.0  virginica   TRUE
## 133          6.4         2.8          5.6         2.2  virginica   TRUE
## 134          6.3         2.8          5.1         1.5  virginica   TRUE
## 135          6.1         2.6          5.6         1.4  virginica   TRUE
## 136          7.7         3.0          6.1         2.3  virginica   TRUE
## 137          6.3         3.4          5.6         2.4  virginica   TRUE
## 138          6.4         3.1          5.5         1.8  virginica   TRUE
## 139          6.0         3.0          4.8         1.8  virginica   TRUE
## 140          6.9         3.1          5.4         2.1  virginica   TRUE
## 141          6.7         3.1          5.6         2.4  virginica   TRUE
## 142          6.9         3.1          5.1         2.3  virginica   TRUE
## 143          5.8         2.7          5.1         1.9  virginica   TRUE
## 144          6.8         3.2          5.9         2.3  virginica   TRUE
## 145          6.7         3.3          5.7         2.5  virginica   TRUE
## 146          6.7         3.0          5.2         2.3  virginica   TRUE
## 147          6.3         2.5          5.0         1.9  virginica   TRUE
## 148          6.5         3.0          5.2         2.0  virginica   TRUE
## 149          6.2         3.4          5.4         2.3  virginica   TRUE
## 150          5.9         3.0          5.1         1.8  virginica   TRUE
\end{verbatim}

\section{Arrays}\label{arrays}

São vetores formados por dataframes, ``matrizes que permitem que suas
colunas tenham diferentes tipos de variáveis etc.

\begin{Shaded}
\begin{Highlighting}[]
\CommentTok{# Construindo um exemplo}
\NormalTok{a<-}\StringTok{ }\KeywordTok{array}\NormalTok{(}\DecValTok{1}\OperatorTok{:}\DecValTok{50}\NormalTok{, }\DataTypeTok{dim =} \KeywordTok{c}\NormalTok{(}\DecValTok{2}\NormalTok{,}\DecValTok{5}\NormalTok{,}\DecValTok{5}\NormalTok{))}
\NormalTok{a}
\end{Highlighting}
\end{Shaded}

\begin{verbatim}
## , , 1
## 
##      [,1] [,2] [,3] [,4] [,5]
## [1,]    1    3    5    7    9
## [2,]    2    4    6    8   10
## 
## , , 2
## 
##      [,1] [,2] [,3] [,4] [,5]
## [1,]   11   13   15   17   19
## [2,]   12   14   16   18   20
## 
## , , 3
## 
##      [,1] [,2] [,3] [,4] [,5]
## [1,]   21   23   25   27   29
## [2,]   22   24   26   28   30
## 
## , , 4
## 
##      [,1] [,2] [,3] [,4] [,5]
## [1,]   31   33   35   37   39
## [2,]   32   34   36   38   40
## 
## , , 5
## 
##      [,1] [,2] [,3] [,4] [,5]
## [1,]   41   43   45   47   49
## [2,]   42   44   46   48   50
\end{verbatim}

\section{List}\label{list}

Generalização dos vetores no sentido de que uma lista é uma coleção de
objetos de tipos os mais variados

\begin{Shaded}
\begin{Highlighting}[]
\NormalTok{dados<-}\KeywordTok{c}\NormalTok{(}\KeywordTok{rep}\NormalTok{(}\DecValTok{1}\OperatorTok{:}\DecValTok{4}\NormalTok{, }\DecValTok{2}\NormalTok{, }\DataTypeTok{each =} \DecValTok{2}\NormalTok{))}
\NormalTok{A =}\StringTok{ }\KeywordTok{list}\NormalTok{(}\DataTypeTok{x =} \DecValTok{1}\OperatorTok{:}\DecValTok{4}\NormalTok{, }\DataTypeTok{y =} \KeywordTok{matrix}\NormalTok{(}\DecValTok{1}\OperatorTok{:}\DecValTok{4}\NormalTok{,}\DecValTok{2}\NormalTok{,}\DecValTok{2}\NormalTok{), }\DataTypeTok{w =}\NormalTok{ dados, }\DataTypeTok{v =} \KeywordTok{list}\NormalTok{(}\DataTypeTok{B=}\DecValTok{4}\NormalTok{,}\DataTypeTok{C=}\DecValTok{5}\NormalTok{))}

\NormalTok{A}
\end{Highlighting}
\end{Shaded}

\begin{verbatim}
## $x
## [1] 1 2 3 4
## 
## $y
##      [,1] [,2]
## [1,]    1    3
## [2,]    2    4
## 
## $w
##  [1] 1 1 2 2 3 3 4 4 1 1 2 2 3 3 4 4
## 
## $v
## $v$B
## [1] 4
## 
## $v$C
## [1] 5
\end{verbatim}

\begin{Shaded}
\begin{Highlighting}[]
\NormalTok{A[[}\DecValTok{1}\NormalTok{]]}
\end{Highlighting}
\end{Shaded}

\begin{verbatim}
## [1] 1 2 3 4
\end{verbatim}

\begin{Shaded}
\begin{Highlighting}[]
\NormalTok{A[[}\DecValTok{4}\NormalTok{]]}
\end{Highlighting}
\end{Shaded}

\begin{verbatim}
## $B
## [1] 4
## 
## $C
## [1] 5
\end{verbatim}

\begin{Shaded}
\begin{Highlighting}[]
\NormalTok{A}\OperatorTok{$}\NormalTok{x}
\end{Highlighting}
\end{Shaded}

\begin{verbatim}
## [1] 1 2 3 4
\end{verbatim}

\begin{Shaded}
\begin{Highlighting}[]
\NormalTok{A}\OperatorTok{$}\NormalTok{y}
\end{Highlighting}
\end{Shaded}

\begin{verbatim}
##      [,1] [,2]
## [1,]    1    3
## [2,]    2    4
\end{verbatim}

\begin{Shaded}
\begin{Highlighting}[]
\NormalTok{A}\OperatorTok{$}\NormalTok{w}
\end{Highlighting}
\end{Shaded}

\begin{verbatim}
##  [1] 1 1 2 2 3 3 4 4 1 1 2 2 3 3 4 4
\end{verbatim}

\begin{Shaded}
\begin{Highlighting}[]
\NormalTok{A}\OperatorTok{$}\NormalTok{v}
\end{Highlighting}
\end{Shaded}

\begin{verbatim}
## $B
## [1] 4
## 
## $C
## [1] 5
\end{verbatim}

\begin{Shaded}
\begin{Highlighting}[]
\NormalTok{B =}\StringTok{ }\KeywordTok{list}\NormalTok{(}\DataTypeTok{s =} \DecValTok{1}\OperatorTok{:}\DecValTok{5}\NormalTok{, }\DataTypeTok{r =} \DecValTok{2}\NormalTok{)}

\NormalTok{Q =}\StringTok{ }\KeywordTok{c}\NormalTok{(A,B)}

\NormalTok{Q}
\end{Highlighting}
\end{Shaded}

\begin{verbatim}
## $x
## [1] 1 2 3 4
## 
## $y
##      [,1] [,2]
## [1,]    1    3
## [2,]    2    4
## 
## $w
##  [1] 1 1 2 2 3 3 4 4 1 1 2 2 3 3 4 4
## 
## $v
## $v$B
## [1] 4
## 
## $v$C
## [1] 5
## 
## 
## $s
## [1] 1 2 3 4 5
## 
## $r
## [1] 2
\end{verbatim}


\end{document}
